\subsection{The Axiom of Replacement}

\exercise{1}{
  Let $\prop{P}(x,y)$ be a property such that for every $x$ there is at most one $y$ for which $\prop{P}(x,y)$ holds.
  Then for every set $A$ there is a set $B$ such that, for all $x \in A$, if $\prop{P}(x,y)$ holds for some $y$, then $\prop{P}(x,y)$ holds for some $y \in B$.
}
\sol{
  \qproof{
    Define a property $\prop{R}(x,y)$ such that $\prop{R}(x,y)$ holds if and only if
    \begin{enumerate}
    \item $\prop{P}(x,y)$ holds, or
    \item $y = \es$ and there is not a $z$ such that $\prop{P}(x,z)$ holds.
    \end{enumerate}
    Clearly this property is such that for every $x$ there is a unique $y$ for which $\prop{R}(x,y)$ holds.

    Now consider any set $A$.
    Then by the Axiom Schema of Replacement there is a set $B$ such that, for every $x \in A$, there is a $y \in B$ for which $\prop{R}(x,y)$ holds.
    Consider any $x \in A$.
    Then by the above there is a $y \in B$ such that $\prop{R}(x,y)$ holds.
    Now suppose that $\prop{P}(x,z)$ holds for some $z$.
    Then option 2 above cannot be the case so that, $\prop{P}(x,y)$ holds (option 1) since $\prop{R}(x,y)$ does.
    Thus $\prop{P}(x,y)$ holds for some $y \in B$ as we were required to show.
  }
}

\exercise{2}{
  Use Theorem~6.3.6 to prove the existence of

  (a) The set $\braces{\es, \braces{\es}, \braces{\braces{\es}}, \braces{\braces{\braces{\es}}}, \ldots}$.

  (b) The set $\braces{\nats, \pset{\nats}, \pset{\pset{\nats}}, \ldots}$.

  (c) The set $\w + \w = \w \cup \braces{\w, \w+1, (\w+1)+1, \ldots}$.
}
\sol{
  (a)
  \qproof{
    Define the operation $\prop{G}(x,n)$ for set a $x$ and $n \in \nats$ by
    $$
    \prop{G}(x,n) = \braces{x} \,.
    $$
    Then by Theorem~6.3.6 there is a unique sequence $\angles{a_n \where n \in \nats}$ where
    \ali{
      a_0 &= \es \\
      a_{n+1} &= \prop{G}(a_n, n) = \braces{a_n}
    }
    for all $n \in \nats$.
    Clearly the range of $\angles{a_n}$ is the set we seek.
  }

  (b)
  \qproof{
    Similarly define the operation $\prop{G}(x,n)$ for a set $x$ and $n \in \nats$ by
    $$
    \prop{G}(x,n) = \pset{x} \,,
    $$
    noting that this set exists by the Axiom of Power Set.
    Then by Theorem~6.3.6 there is a sequence $\angles{a_n \where n \in \nats}$ defined by
    \ali{
      a_0 &= \nats \\
      a_{n+1} &= \prop{G}(a_n, n) = \pset{a_n} \,,
    }
    noting that $a_0 = \nats$ exists by the Axiom of Infinity.
    Clearly then the range of $\angles{a_n}$ is the set we seek.
  }

  (c)
  \qproof{
    Define the operation $\prop{G}(x,n)$ for a set $x$ and $n \in \nats$ by
    $$
    \prop{G}(x,n) = S(x) = x \cup \braces{x} \,,
    $$
    Then by Theorem~6.3.6 there is a sequence $\angles{a_n \where n \in \nats}$ defined by
    \ali{
      a_0 &= \w \\
      a_{n+1} &= \prop{G}(a_n, n) = S(a_n) = a_n+1 \,,
    }
    noting that $a_0 = \w = \nats$ exists by the Axiom of Infinity.
    Clearly then the range of $\angles{a_n}$ is $A = \braces{\w, \w+1, \w+2, \ldots}$.
    It then follows that  $\w + \w = \w \cup A$ is the set we seek.
  }
}

\exercise{3}{
  Use Theorem~6.3.6 to define
  \ali{
    V_0 &= \es; \\
    V_{n+1} &= \pset{V_n} \hspace{0.5cm} (n \in \w); \\
    V_\w &= \bigcup_{n \in \w} V_n \,.
  }
}
\sol{
  \qproof{
    Define the operation $\prop{G}(x,n)$ for a set $x$ and $n \in \nats$ by
    $$
    \prop{G}(x,n) = \pset{x} \,,
    $$
    noting that this set exists by the Axiom of Power Set.
    Then by Theorem~6.3.6 there is a sequence $\angles{V_n \where n \in \nats}$ defined by
    \ali{
      V_0 &= \es \\
      V_{n+1} &= \prop{G}(V_n, n) = \pset{V_n} \,,
    }
    noting that $V_0 = \es$ exists by the Axiom of Existence.
    Then we let
    $$
    V_\w = \bigcup_{n \in \w} V_n \,,
    $$
    noting that $\w = \nats$.
    This set exists by the Axiom of Union.
  }
}

\exercise{4}{
  (a) Every $x \in V_\w$ is finite.

  (b) $V_\w$ is transitive.

  (c) $V_\w$ is an inductive set.
}
\sol{

  (a)
  \qproof{
    First we show by induction that every $V_n$ is finite (for $n \in \nats$).
    For $n=0$ we have $V_n = V_0 = \es$, which is clearly finite.
    Now suppose that $V_n$ is finite then we have $V_{n+1} = \pset{V_n}$, which is finite by Theorem~4.2.8.

    Now consider any $x \in V_\w = \bigcup_{n \in \w} V_n$ so that there is an $n \in \w$ such that $x \in V_n$.
    We note that $n \neq 0$ since $V_0 = \es$ so it cannot be that $x \in V_0 = \es$.
    Hence $V_{n-1}$ is a set and moreover $V_n = \pset{V_{n-1}}$.
    So since $x \in V_n$ it follows that $x \in \pset{V_{n-1}}$ so that $x \ss V_{n-1}$.
    Thus it follows that $|x| \leq |V_{n-1}|$ so that clearly $x$ is finite since $V_{n-1}$ is (shown above).
  }

  (b)
  \qproof{
    Consider any $x \in V_w$.
    Then by the same argument as in part (a) above it follows that $x \in V_n$ where $n \neq 0$.
    Hence again $V_{n-1}$ is a set and $V_n = \pset{V_{n-1}}$ so that $x \ss V_{n-1}$.
    Then for any $y \in x$ we have that $y \in V_{n-1}$, from which it follows that clearly $y \in \bigcup_{k \in \w} V_k = V_\w$.
    Hence since $y$ was arbitrary $x \ss V_\w$, and since $x$ was arbitrary this shows that $V_\w$ is transitive by definition.
  }

  (c)
  \qproof{
    First we show by induction that each $V_n$ (where $n \in \w$) is transitive.
    For $n=0$ we have $V_n = V_0 = \es$, which is clearly vacuously transitive.
    Now suppose that $V_n$ is transitive and consider any $x \in V_{n+1} = \pset{V_n}$ so that $x \ss V_n$.
    Now consider any $y \in x$ so that also $y \in V_n$.
    But since $V_n$ is transitive $y \ss V_n$ so that $y \in \pset{V_n} = V_{n+1}$.
    Hence since $y$ was arbitrary this shows that $x \ss V_{n+1}$ and since $x$ was arbitrary this shows by definition that $V_{n+1}$ is transitive, thereby completing the inductive proof.

    Now we show that $V_\w$ is inductive.
    So  first note that $V_1 = \pset{V_0} = \pset{\es} = \braces{\es}$ so that $0 = \es \in V_1$.
    From this is clearly follows that $0 \in \bigcup_{n \in \w} V_n = V_\w$.

    Now suppose that $n \in V_\w = \bigcup_{k \in \w} V_k$ so that there is an $m \in \w$ such that $n \in V_m$.
    Since it was shown above that $V_m$ is transitive we have that $n \ss V_m$ as well.
    So consider any $x \in n+1 = n \cup \braces{n}$.
    If $x \in n$ then also $x \in V_m$ since $n \ss V_m$.
    On the other hand if $x \in \braces{n}$ then $x = n \in V_m$.
    Since $x$ was arbitrary this shows that $n+1 \ss V_m$ so that $n+1 \in \pset{V_m} = V_{m+1}$.
    From this it clearly follows that $n+1 \in \bigcup_{k \in \w} V_k = V_\w$.
    This shows that $V_\w$ is inductive by definition.
  }
}

\exercise{5}{
  (a) If $x \in V_\w$ and $y \in V_\w$, then $\braces{x,y} \in V_\w$.

  (b) If $X \in V_\w$, then $\bigcup X \in V_\w$ and $\pset{X} \in V_\w$.

  (c) If $A \in V_\w$ and $f$ is a function on $A$ such that $f(x) \in V_\w$ for each $x \in A$, then $f[X] \in V_\w$.

  (d) If $X$ is a finite subset of $V_\w$, then $X \in V_\w$.

  Note that part (c) differs slightly from the book; see the Errata List.
}
\sol{

  (a)
  \qproof{
    First we show that if $x \in V_n$ for some $n \in \w$ then $x \in V_m$ for all $m \geq n$.
    We show this by induction on $m$.
    So for $m=n$ clearly $x \in V_n = V_m$.
    Now suppose that $x \in V_m$.
    Then it was shown in Exercise~6.3.4 part (c) that $V_m$ is transitive so that $x \ss V_m$.
    Hence $x \in \pset{V_m} = V_{m+1}$, thereby completing the inductive proof.

    Now suppose that $x,y \in V_\w$.
    Then  there are $n,m \in \w$ such that $x \in V_n$ and $y \in V_m$.
    Without loss of generality we can assume that $n \leq m$ (since if this is not the case then we simply reverse the roles of $x$ and $y$).
    So since $m \geq n$ it follows from what was shown above that $x \in V_m$ as well.
    Hence we have that clearly $\braces{x,y} \ss V_m$ since both $x \in V_m$  and $y \in V_m$.
    Then $\braces{x,y} \in \pset{V_m} = V_{m+1}$ from which it clearly follows that $\braces{x,y} \in \bigcup_{k \in \w} V_k = V_\w$.
  }

  (b)
  \qproof{
    Suppose that $X \in V_\w = \bigcup_{k \in \w} V_k$.
    Then there is an $n \in \w$ such that $X \in V_n$.
    It was shown in Exercise 6.3.4 part (c) that $V_n$ is transitive so that $X \ss V_n$.

    First we show that $\bigcup X \in V_\w$.
    So consider any $x \in \bigcup X$ so that there is a $Y \in X$ such that $x \in Y$.
    Then since $X \ss V_n$ we have that $Y \in V_n$.
    Since again $V_n$ is transitive we have that $Y \ss V_n$ so that $x \in V_n$ since $x \in Y$.
    Since $x$ was arbitrary it follows that $\bigcup X \ss V_n$ so that $\bigcup X \in \pset{V_n} = V_{n+1}$.
    From this it clearly follows that  $\bigcup X \in \bigcup_{k \in \w} V_k = V_\w$.

    Next we show that $\pset{X} \in V_\w$.
    So consider any $Y \in \pset{X}$ so that $Y \ss X$.
    Now consider any $y \in Y$ so that also $y \in X$.
    Since $X \ss V_n$ we have that $y \in V_n$.
    But since $y \in Y$ was arbitrary it follows that $Y \ss V_n$ so that $Y \in \pset{V_n} = V_{n+1}$.
    Then since $Y \in \pset{X}$ was arbitrary it follows that $\pset{X} \ss V_{n+1}$ so that $\pset{X} \in \pset{V_{n+1}} = V_{n+2}$.
    From this it clearly follows that  $\pset{X} \in \bigcup_{k \in \w} V_k = V_\w$.
  }

  (c)
  \qproof{
    Note the issue with this part in the errata list.
    Since $A \in V_\w$ we have by Exercise~6.3.4 part (a) that $A$ is finite.
    Then by Theorem~2.2.5 it follows that $f[A]$ is finite.
    Also clearly $f[A]$ is a subset of $V_\w$ and hence is a finite subset.
    Therefore by part (d) below $f[A] \in V_\w$.
  }

  (d)
  \qproof{
    Consider any finite $X \ss V_\w$.
    Suppose then that $|X| = n$ for some $n \in \nats$.
    Then for each $x_k \in X$, where $k \in n$, we have that $x_k \in V_\w = \bigcup_{m \in \w} V_m$ so that there is an $m_k \in \w$ where $x_k \in V_{m_k}$.
    Now let $m = \max_{k \in n} m_k$, which exists since $n$ is finite.
    Then, for any $k \in n$, by what was shown in Exercise~6.3.5 part (a) we have $x_k \in V_m$ since $x_k \in V_{m_k}$ and $m \geq m_k$.
    Hence it follows that $X \ss V_m$ so that $X \in \pset{V_m} = V_{m+1}$.
    Clearly then $X \in \bigcup_{k \in \w} V_k = V_\w$.
  }
}
