\subsection{The Axiom of Choice and its Equivalents}

\exercise{1}{
  Prove: If a set $A$ can be linearly ordered, then every system of finite subsets of $A$ has a choice function.
  (It does not follow from the Zermelo-Fraenkel axioms that every set can be linearly ordered.)
}
\sol{
  \begin{lem}\label{lem:aoc:finitelo}
    Any linear ordering of a finite set is a well-ordering.
  \end{lem}
  \qproof{
    We show this by strong induction on the cardinality of the set.
    So consider natural number $n$ and suppose that all linear ordered sets of cardinality $k  < n$ are well-orderings.
    Also suppose that $(A, \preceq)$ is any linearly ordered set with $\abs{A} = n$.
    Consider any nonempty $B \ss A$ so that there is a $b \in B$.
    Then clearly $C = B - \braces{b}$ is also a finite set with $C \pss B \ss A$ so that $\abs{C} < n$.
    Clearly also $C$ is linearly ordered by $\preceq$ so that, by the induction hypothesis, $C$ is well-ordered by $\preceq$.
    Now, if $C = \es$, then it follows that $B = \braces{b}$, which clearly has least element $b$.
    On the other hand, if $C \neq \es$, then it has a least element $c$ since it is well-ordered.
    Since $\preceq$ is a \emph{linear} ordering, it has to be that either $c \preceq b$ or $b \preceq c$.

    Case: $c \preceq b$.
    Then consider any $x \in B$.
    If $x = b$ then obvisouly $c \preceq b = x$.
    If $x \neq b$ then $x \in C = B - \braces{b}$ so that again $c \preceq x$ since $c$ is the least element of $C$.
    This shows that $c$ is the least element of $B$ since $x$ was arbitrary.

    Case: $b \preceq c$.
    Then consider any $x \in B$.
    If $x = b$ then obviously $b \preceq b = x$.
    If $x \neq b$ then $x \in C = B - \braces{b}$ so that $b \preceq c \preceq x$ since $c$ is the least element of $C$.
    This shows that $b$ is the least element of $B$ since $x$ was arbitrary.

    Hence in all cases we have that $B$ has a $\preceq$-least element.
    This shows that $\preceq$ is a well-ordering of $A$ since $B \ss A$ was arbitrary.
    This completes the inductive proof.
  }

  \mainprob
  \qproof{
    Suppose that $A$ is a set that can be linearly ordered and suppose that $\preceq$ is a such a linear ordering.
    Suppose also that $S$ is a system of finite subsets of $A$.
    Then clearly any $B \in S$ is finite and linearly ordered by $\preceq$.
    We then have that $\preceq$ is a well-ordering of $B$ by Lemma~\ref{lem:aoc:finitelo}.
    So we then set
    \gath{
      f(B) = \begin{cases}
        \text{the least element of $B$ according to $\preceq$} & B \neq \es \\
        \es & B = \es
      \end{cases}
    }
    for any $B \in S$.
    Clearly then $f$ is a choice function for $S$.
  }
}
