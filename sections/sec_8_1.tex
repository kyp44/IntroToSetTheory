\subsection{The Axiom of Choice and its Equivalents}

\exercise{1}{
  Prove: If a set $A$ can be linearly ordered, then every system of finite subsets of $A$ has a choice function.
  (It does not follow from the Zermelo-Fraenkel axioms that every set can be linearly ordered.)
}
\sol{
  \begin{lem}\label{lem:aoc:finitelo}
    Any linear ordering of a finite set is a well-ordering.
  \end{lem}
  \qproof{
    We show this by strong induction on the cardinality of the set.
    So consider natural number $n$ and suppose that all linear ordered sets of cardinality $k  < n$ are well-orderings.
    Also suppose that $(A, \prece)$ is any linearly ordered set with $\abs{A} = n$.
    Consider any nonempty $B \ss A$ so that there is a $b \in B$.
    Then clearly $C = B - \braces{b}$ is also a finite set with $C \pss B \ss A$ so that $\abs{C} < n$.
    Clearly also $C$ is linearly ordered by $\prece$ so that, by the induction hypothesis, $C$ is well-ordered by $\prece$.
    Now, if $C = \es$, then it follows that $B = \braces{b}$, which clearly has least element $b$.
    On the other hand, if $C \neq \es$, then it has a least element $c$ since it is well-ordered.
    Since $\prece$ is a \emph{linear} ordering, it has to be that either $c \prece b$ or $b \prece c$.

    Case: $c \prece b$.
    Then consider any $x \in B$.
    If $x = b$ then obviously $c \prece b = x$.
    If $x \neq b$ then $x \in C = B - \braces{b}$ so that again $c \prece x$ since $c$ is the least element of $C$.
    This shows that $c$ is the least element of $B$ since $x$ was arbitrary.

    Case: $b \prece c$.
    Then consider any $x \in B$.
    If $x = b$ then obviously $b \prece b = x$.
    If $x \neq b$ then $x \in C = B - \braces{b}$ so that $b \prece c \prece x$ since $c$ is the least element of $C$.
    This shows that $b$ is the least element of $B$ since $x$ was arbitrary.

    Hence in all cases we have that $B$ has a $\prece$-least element.
    This shows that $\prece$ is a well-ordering of $A$ since $B \ss A$ was arbitrary.
    This completes the inductive proof.
  }

  \mainprob
  \qproof{
    Suppose that $A$ is a set that can be linearly ordered and suppose that $\prece$ is a such a linear ordering.
    Suppose also that $S$ is a system of finite subsets of $A$.
    Then clearly any $B \in S$ is finite and linearly ordered by $\prece$.
    We then have that $\prece$ is a well-ordering of $B$ by Lemma~\ref{lem:aoc:finitelo}.
    So we then set
    \gath{
      f(B) = \begin{cases}
        \text{the least element of $B$ according to $\prece$} & B \neq \es \\
        \es                                                   & B = \es
      \end{cases}
    }
    for any $B \in S$.
    Clearly then $f$ is a choice function for $S$.
  }
}

\exercise{2}{
  If $A$ can be well-ordered, then $\pset{A}$ can be linearly ordered.
    [Hint: Let $<$ be a well-ordering of $A$; for $X,Y \ss A$ define $X \prec Y$ if any only if the $<$-least element of $X \sd Y$ belongs to $X$.]
}
\sol{
  \qproof{
    Suppose that $<$ is a well-ordering of $A$.
    Then, following the hint, defined the relation $X \prec Y$ if and only if the $<$-least element of $X \sd Y$ is in $X$ for any $X$ and $Y$ in $\pset{A}$.
    Note that, for any $x \in X \sd Y = (X - Y) \cup (Y - X)$, we clearly have that $x \in X$ or $x \in Y$ so that $x \in A$ since $X \ss A$ and $Y \ss A$.
    Hence $X \sd Y \ss A$ so that it is also well-ordered by $<$.

    First we show that $\prec$ is a (strict) order on $\pset{A}$.
    Hence we must show that it is asymmetric and transitive.
    So consider any $X$ and $Y$ in $\pset{A}$ where $X \prec Y$.
    Then by definition the $<$-least element $x$ in $X \sd Y$ is in $X$.
    Suppose also that $Y \prec X$ so that, since $Y \sd X = X \sd Y$, $x$ is also in $Y$.
    Then clearly $x$ can be neither in $X - Y$ nor $Y - X$, but then it cannot be that $x \in (X - Y) \cup (Y - X) = X \sd Y$.
    This is a contradiction since $x$ was defined to be in $X \sd Y$.
    Hence it cannot be that $Y \prec X$ as well, which shows that $\prec$ is asymmetric since $X$ and $Y$ were arbitrary.

    To see that $\prec$ is transitive, consider any $X, Y, Z \in \pset{A}$ where $X \prec Y$ and $Y \prec Z$.
    Then the least element $x$ of $X \sd Y$ is in $X$ and the least element $y$ of $Y \sd Z$ is in $Y$.
    Thus it has to be that $x \in X - Y$ and $y \in Y - Z$ so that $x \in X$, $x \notin Y$, $y \in Y$, and $y \notin Z$.
    Note that, in particular, this means that $x \neq y$.
    Since clearly $\braces{x, y} \ss A$, it follows that it has a $<$-least element $a$.
    Thus either $a = x$ or $a = y$.
    For each case we show that
    \begin{enumerate}
      \item $a \in X$
      \item $a \in X \sd Z$
      \item $a$ is a lower bound of $X \sd Z$
    \end{enumerate}

    Case: $a = x$.
    Then clearly $a = x \leq y$ so that $a < y$ since $a = x \neq y$.
    \begin{enumerate}
      \item Clearly $a \in X$ since $a = x$.
      \item Suppose that $a \in Z$.
            Then, since $a = x \notin Y$, we have that $a \in Z - Y$ so that $a \in Y \sd Z$.
            Hence $y \leq a$ since $y$ is the least element of $Y \sd Z$, but this contradicts the fact that $y > a$.
            So it must be that in fact $a \notin Z$ so that $a \in X - Z$.
            Thus $a \in X \sd Z$.
      \item Now consider any $z \in X \sd Z$.

            Case: $z \in X - Z$.
            Then, if $z \in Y$, we have that $z \in Y - Z$ so that $z \in Y \sd Z$.
            It then follows that $a = x \leq y \leq z$ since $y$ is the least element of $Y \sd Z$.
            On the other hand, if $z \notin Y$, then we have $z \in X - Y$ so that $z \in X \sd Y$.
            Hence $a = x \leq z$ since $x$ is the least element of $X \sd Y$.

            Case: $x \in Z - X$.
            Then, if $z \in Y$, we have $x \in Y - X$ so $z \in X \sd Y$.
            Then $a = x \leq z$ since $x$ is the least element of $X \sd Y$.
            On the other hand, if $z \notin Y$, then we have $z \in Z - Y$ so that $z \in Y \sd Z$.
            Then, as before, $a = x \leq y \leq z$ since $y$ is the least element of $Y \sd Z$.

            Hence in all cases $a \leq z$ so that $a$ is a lower bound since $z$ was arbitrary.
    \end{enumerate}

    Case: $a = y$.
    Then clearly $a = y \leq x$ so that $a < x$ since $a = y \neq x$.
    \begin{enumerate}
      \item Suppose that $a \notin X$.
            Then since $a = y \in Y$ we have that $a \in Y - X$ so that also $a \in X \sd Y$.
            But then $x \leq a$ since $x$ is the least element of $X \sd Y$, which contradicts the fact that $x > a$.
            Hence it has to be that $a \in X$.
      \item We already know that $a = y \notin Z$ so that $a \in X - Z$ since we just showed that $a \in X$.
            Hence $a \in X \sd Z$.
      \item Consider any $z \in X \sd Z$.

            Case: $z \in X - Z$.
            If also $z \in Y$ then clearly $z \in Y - Z$ so that $z \in Y \sd Z$.
            It then follows that $a = y \leq z$ since $y$ is the least element of $Y \sd Z$.
            On the other hand, if $z \notin Y$, then clearly $z \in X - Y$ so that $z \in X \sd Y$.
            Hence $a = y \leq x \leq z$ since $x$ is the least element of $X \sd Y$.

            Case: $z \in Z - X$.
            Then, if $z \in Y$, clearly $z \in Y - X$ so that $z \in X \sd Y$.
            We then have that $a = y \leq x \leq z$ since $x$ is the least element of $X \sd Y$.
            On the other hand, if $z \notin Y$, then $z \in Z - Y$ so that $z \in Y \sd Z$.
            Then clearly $a = y \leq z$ since $y$ is the least element of $Y \sd Z$.

            Hence in all cases $a \leq z$ so that $a$ is a lower bound since $z$ was arbitrary.
    \end{enumerate}

    Thus in all cases we have that $a$ is the least element of $X \sd Z$ (since it is in $X \sd Z$ and also is a lower bound) and $a \in X$.
    By definition, this shows that $X \prec Z$ so that $\prec$ is transitive.
    This also shows that $\prec$ is a (strict) order.

    Lastly, we show that $\prec$ is a linear ordering.
    So consider any $X, Y \in \pset{A}$.
    Assume that $X \neq Y$ so that $X - Y \neq \es$ or $Y - X \neq \es$ (or both).
    From this it follows that $X \sd Y \neq \es$.
    Since also clearly $X \sd Y \ss A$, it has a least element $a$.
    If $a \in X - Y$ then $a \in X$ so that $X \prec Y$.
    Similarly, if $a \in Y - X$ then $a \in Y$ so that $Y \prec X$.
    Hence we have shown that either $X = Y$, $X \prec Y$, or $Y \prec X$ so that $\prec$ is in fact linear since $X$ and $Y$ were arbitrary.

    This completes the proof since we have shown that $\prec$ is a linear ordering of $\pset{A}$.
  }
}

\exerciseapp{3}{*}{
  Let $(A, \leq)$ be an ordered set in which every chain has an upper bound.
  Then for every $a \in A$, there is a $\leq$-maximal element of $x$ of $A$ such that $a \leq x$.
}
\sol{
  \input{shared/lem_aoc_notin}

  \mainprob

  The proof of this is similar to the proof of Zorn's Lemma from the Axiom of Choice (part of Theorem~8.1.13 in the text).
  \qproof{
    First, by Lemma~\ref{lem:aoc:notin}, there is a $b \notin A$.
    Also, by the Axiom of Choice, there is a choice function $g$ on $\pset{A}$.
    Now consider any $a \in A$.
    We then define a transfinite sequence $\angles{a_\a \where \a < h(A)}$ by transfinite recursion as follows.
    Set $a_0 = a$.
    Then, having constructed the sequence $\angles{a_\x \where \x < \a}$ for $0 < \a < h(A)$, we define the set $A_\a = \braces{x \in A \where a_\x < x \text{ for all } \x < \a}$.
    We then set
    \gath{
      a_\a = \begin{cases}
        g(A_\a) & \text{if $a_\x \neq b$ for all $\x < \a$ and $A_\a \neq \es$} \\
        b       & \text{otherwise}.
      \end{cases}
    }

    We claim that there is an $\a < h(A)$ such that $a_\a = b$.
    To see this, suppose to the contrary that $a_\a \neq b$ for all $\a < h(A)$ so that it has to be that each $a_\a \in A$.
    Consider now any $\a < h(A)$ and $\b < h(A)$ where $\a \neq \b$.
    Without loss of generality we can assume that $\a < \b$.
    Clearly then, by definition, we have that $a_\b \in A_\b$ so that $a_\x < a_\b$ for all $\x < \b$.
    But since $\a < \b$, we have that $a_\a < a_\b$ so that $a_\a \neq a_\b$.
    Since $\a$ and $\b$ were arbitrary, this shows that the sequence is an injective function from $h(A)$ to $A$.
    However, this would mean that $h(A)$ is equipotent to some subset of $A$, which contradicts the definition of the Hartogs number.
    Hence it has to be that $a_\a = b$ for some $\a < h(A)$.

    So let $\l < h(A)$ be the least ordinal such that $a_\l = b$ and let $C = \braces{a_\x \where \x < \l}$.
    We claim that $C$ is a chain in $(A, \leq)$.
    So consider any $a_\a$ and $a_\b$ in $C$ so that $\a < \l$ and $\b < \l$
    Without loss of generality we can assume that $\a \leq \b$.
    If $\a = \b$ then obviously $a_\a = a_\b$ so that $a_a \leq a_\b$ clearly holds.
    If $\a < \b$ then, by what was shown above, we have that $a_\a < a_\b$ so that $a_\a \leq a_\b$ again holds.
    Hence, in every case, $a_\a$ and $a_\b$ are comparable in $\leq$, which shows that $C$ is a chain since $a_\a$ and $a_\b$ were arbitrary.

    Thus, since $C$ is a chain of $A$, it has an upper bound $c \in A$.
    We claim that $c$ is also a maximal element of $A$.
    To show this, suppose that there is an $x \in A$ such that $c < x$.
    Now consider any $\x < \l$.
    Then, since $c$ is an upper bound of $C$, we have that $a_\x \leq c < x$ so that $a_\x < x$ since orders are transitive.
    It then follows from the definition of $A_\l$ that $x \in A_\l$ so that $A_\l \neq \es$.
    Also note that, by the definition of $\l$, we have that $a_\x \neq b$ for any $\x < \l$.
    Thus, by the recursive definition of the sequence, it follows that $a_\l = g(A_\l) \neq b$, which contradicts the definition of $\l$ (as the least ordinal such that $a_\l = b$).
    So it has to be that there is no such element $x$, which shows that $c$ is in fact a maximal element of $A$.

    Now, it has to be that $0 \neq \l$ since $a_0 = a \neq b = a_\l$.
    It then follows that $0 < \l$ since $\l$ is an ordinal.
    Hence $a = a_0 \in C$ by the definition of $C$.
    Then, since $c$ is an upper bound of $C$, we have that $c$ is a maximal element of $A$ where $a \leq c$.
    Since $a$ was arbitrary this shows the desired result.
  }
}

\exercise{4}{
  Prove that Zorn's Lemma is equivalent to the statement: For all $(A, \leq)$, the set of all chains of $(A, \leq)$ has an $\ss$-maximal element.
}
\sol{
  \qproof{
    \bicproof{
      First, suppose that Zorn's Lemma is true, and let $C$ be the set of all chains of $(A, \leq)$.
      First, it is trivial to show that $\ss$ is a partial order on $C$, i.e. that it is reflexive, antisymmetric, and transitive.
      Let $B \ss C$ be any $\ss$-chain, and let $U = \bigcup B$.

      First we claim that that $U \in C$, which requires that we show that $U$ is a chain of $(A, \leq)$.
      So consider any $x,y \in U = \bigcup B$ so that there are sets $X,Y \in B$ such that $x \in X$ and $y \in Y$.
      Since $B$ is a $\ss$-chain it follows that either $X \ss Y$ or $Y \ss X$.
      In the case of $X \ss Y$ then clearly both $x$ and $y$ are in $Y$ (since $x \in X$ and $X \ss Y$).
      Then, since $Y \in C$ (since $Y \in B$ and $B \ss C$), we have that $Y$ is a $\leq$-chain.
      Hence $x$ and $y$ are comparable in $\leq$.
      The case in which $Y \ss X$ is analogous.
      Since $x,y \in U$ were arbitrary, this shows that $U$ is a $\leq$-chain so that $U \in C$.

      We also claim that $U$ is an upper bound (with respect to $\ss$) of $B$.
      To show this, consider any $X \in B$ and any $x \in X$.
      Then clearly $x \in \bigcup B = U$.
      Hence $X \ss U$ since $x$ was arbitrary.
      Since also $X \in B$ was arbitrary, this shows that $U$ is an upper bound of $B$.

      Thus, since $B$ was an arbitrary $\ss$-chain, this shows that every chain of $(C, \ss)$ has an upper bound.
      It then follows from Zorn's Lemma that $C$ has a $\ss$-maximal element as desired.
    }{
      Suppose that the set of all chains of $(A, \leq)$ has a $\ss$-maximal element for any $(A, \leq)$.
      So consider any such ordered set $(A, \leq)$ where every chain has a upper bound.
      Let $C$ be the set of all chains of $(A, \leq)$ so that $C$ has a $\ss$-maximal element $M$ by our initial supposition.
      Then, since $M \in C$, it is a chain so it has an upper bound $a \in A$.
      We claim that $a$ is a maximal element of $(A, \leq)$.

      To show this, assume to the contrary that there is a $b \in A$ such that $a < b$, and let $M' = M \cup \braces{b}$.
      Consider any $x,y \in M'$.
      If $x,y \in M$ then clearly $x$ and $y$ are comparable in $\leq$ since $M$ is a chain.
      On the other hand, if $x \in M$ but $y = b$, then $x \leq a$ since $a$ is an upper bound of $M$.
      We also have that $a < b = y$ so that $x < y$ since orders are transitive.
      The case in which $y \in M$ but $x = b$ similarly leads to $y < x$.
      Lastly, if $x = y = b$ then clearly $x \leq y$ is true.
      Hence in all cases $x$ and $y$ are comparable in $\leq$, which shows that $M'$ is a chain since $x$ and $y$ were arbitrary.
      Therefore $M' \in C$.

      Now, it has to be that $b \notin M$ since, if it were, $a$ could not be an upper bound of $M$ since $a < b$ (and therefore it cannot be that $b \leq a$ since the strict ordering is asymmetric).
      So, since $b \notin M$ it follows that $M \pss M \cup \braces{b} = M'$, which contradicts the fact that $M$ is a $\ss$-maximal element of $C$ since also $M' \in C$.
      So it has to be that there is no such $b$ where $a < b$, which shows that $a$ is in fact a maximal element of $(A, \leq)$.
      This proves Zorn's Lemma since $(A, \leq)$ was arbitrary.
    }
  }
}

\exercise{5}{
  Prove that Zorn's Lemma is equivalent to the statement: If $A$ is a system of sets such that, for each $B \ss A$ which is linearly ordered by $\ss$, $\bigcup B  \in A$, then $A$ has an $\ss$-maximal element.
}
\sol{
  \qproof{
    \bicproof{
      Suppose Zorn's Lemma and let $A$ be a system of sets where $\bigcup B \in A$ for any $B$ that is linearly ordered by $\ss$.
      We know that $\ss$ is a partial order on $A$.
      So let $B$ be any $\ss$-chain of $A$.
      Then we know that $\bigcup B \in A$, and we also claim that $\bigcup B$ is an upper bound of $B$.
      To see this, consider any $X \in B$ and any $x \in X$, so that clearly $x \in \bigcup B$.
      Hence $X \ss B$ since $x$ was arbitrary.
      This shows that $\bigcup B$ is an upper bound of $B$ since $X$ was arbitrary.
      Since $B$ was an arbitrary chain, this shows that $(A, \ss)$ is an ordered set where every chain has an upper bound.
      Thus by Zorn's Lemma there is a $\ss$-maximal element of $A$ as desired.
    }{
      Now suppose that $A$ has a $\ss$-maximal element for any system of sets $A$ such that $\bigcup B \in A$ for any $B \ss A$ where $B$ is linearly ordered by  $\ss$.
      Consider any ordered set $(A, \leq)$ and let $C$ be the set of all chains of $A$.
      Let $B$ be any subset of $C$ that is linearly ordered by $\ss$, and consider any $x$ and $y$ in $\bigcup B$.
      Then there are $X$ and $Y$ in $B$ such that $x \in X$ and $y \in Y$.
      Since $B$ is linearly ordered by $\ss$ we have that either $X \ss Y$ or $Y \ss X$.
      In the former case we have $x \in X \ss Y$ so that both $x$ and $y$ are in $Y$.
      Hence $x$ and $y$ are comparable in $\leq$ since $Y \in B \ss C$ so that $Y$ is a $\leq$-chain.
      A similar argument shows that $x$ and $y$ are comparable if $Y \ss X$.
      Since $x$ and $y$ were arbitrary this shows that $\bigcup B$ is a $\leq$-chain so that $\bigcup B \in C$.

      Thus $C$ is a system of sets that meet the criteria of the initial supposition since $B$ was arbitrary.
      Hence $C$ has a $\ss$-maximal element.
      Since $(A, \leq)$ was an arbitrary ordered set and we have shown that the set of all chains of $(A, \leq)$ has a $\ss$-maximal element, Zorn's Lemma follows from Exercise~8.1.4.
    }
  }
}

\exercise{6}{
  A system of sets $A$ has \emph{finite character} if $X \in A$ if and only if every finite subset of $X$ belongs to $A$.
  Prove that Zorn's Lemma is equivalent to the following (Tukey's Lemma): Every system of sets of finite character has an $\ss$-maximal element.
    [Hint: Use Exercise~8.1.5.]
}
\sol{
  \qproof{
    \bicproof{
      Suppose Zorn's Lemma and let $A$ be an arbitrary system of sets of finite character.
      Suppose that $B$ is any subset of $A$ that is linearly ordered by $\ss$ and let $C$ be any finite subset of $\bigcup B$.
      Now, for each $x \in C$ there is a set $X_x \in B$ such that $x \in X_x$, since $x \in C \ss \bigcup B$.
      Clearly the set $D = \braces{X_x \where x \in C}$ is a subset of $B$ so that $D$ is also linearly ordered by $\ss$.
      Also clearly $D$ is finite since $C$ is.
      Hence $D$ has a $\ss$-greatest element $X$.\
      Note that the Axiom of Choice is not needed in selecting the set $X_x$ for each $x \in C$ since we are only making a finite number of choices.
      So consider any $x \in C$ so that $x \in X_x \ss X$.
      Hence $x \in X$ so that $C \ss X$ since $x$ was arbitrary.
      We also have that $X \in B \ss A$ so $X \in A$.
      Therefore $C$ is a finite subset of $X$, which is an element of $A$, so that $C$ is also in $A$ since $A$ has finite character.
      Since $C$ was an arbitrary finite subset of $\bigcup B$ and $C \in A$ it follows that $\bigcup B \in A$.
      Hence, since $B$ was an arbitrary linearly ordered (by $\ss$) subset of $A$, we have by Exercise~8.1.5 and Zorn's Lemma that $A$ has a $\ss$-maximal element as desired.
    }{
      Now suppose that every system of sets of finite character has a $\ss$-maximal element.
      Let $(A, \leq)$ be any ordered set and let $C$ be the set of all chains of $(A, \leq)$.
      Now suppose that $X \in C$ and let $Y$ be any finite subset of $X$.
      Clearly since $X \in C$, it is linearly ordered by $\leq$ so that $Y$ is as well since $Y \ss X$.
      Hence $Y \in C$.
      Now let $X'$ be any set such that every finite subset of $X'$ is in $C$.
      Consider any $x,y \in X'$.
      Then $\braces{x, y}$ is clearly a finite subset of $X'$ so that it is in $C$ and therefore a $\leq$-chain.
      Hence $x$ and $y$ are comparable in $\leq$, which shows that $X'$ itself is a $\leq$-chain since $x$ and $y$ were arbitrary.
      Hence $X' \in C$.
      Thus we have just shown that $X \in C$ if and only if every finite subset of $X$ is in $C$ so that $C$ has finite character by definition.
      Therefore, by the initial supposition, $C$ has a $\ss$ maximal element.
      Since again $C$ is the set of all chains of the arbitrary $(A, \leq)$, Zorn's Lemma follows from Exercise~8.1.4.
    }
  }
}

\exerciseapp{7}{*}{
  Let $E$ be a binary relation on a set $A$.
  Show that there exists a function $f: A \to A$ such that for all $x \in A$, $(x, f(x)) \in E$ if and only if there is some $y \in A$ such that $(x, y) \in E$.
}
\sol{
  \qproof{
    If $A = \es$ then clearly it must be that $E = \es$ since $E \ss A \times A = \es \times \es = \es$.
    Hence $f = \es$ is vacuously such a function.
    So assume that $A \neq \es$ so that there is an $a \in A$.
    For any $x \in A$ define the set $Y_x = \braces{y \in A \where (x,y) \in E}$, noting that this could certainly be empty if $x$ is not in the domain of $E$.
    Clearly $S = \braces{Y_x \where x \in A}$ is a system of sets, and so has a choice function $g$ by the Axiom of Choice.
    We then define a function $f : A \to A$ by
    \gath{
      f(x) = \begin{cases}
        g(Y_x) & Y_x \neq \es \\
        a      & Y_x = \es
      \end{cases}
    }
    for all $x \in A$.
    We claim that $f$ meets the required criteria, so let $x$ be some element of $A$.

    \bicproof{
      Suppose that $(x, f(x)) \in E$.
      Then clearly for $y = f(x)$ we have that $(x, y) = (x, f(x)) \in E$.
      We note that, if $Y = \es$, then $y = f(x) = a \in A$, and if $Y_x \neq \es$ then $y = f(x) = g(Y_x) \in Y_x$ since $g$ is a choice function so that again $y \in A$ since clearly $Y_x \ss A$.
    }{
      Now suppose that there is a $y \in A$ such that $(x,y) \in E$.
      Then clearly by definition we have $y \in Y_x$ so that $Y_x \neq \es$.
      Thus $f(x) = g(Y_x) \in Y_x$ since $g$ is a choice function.
      We therefore have $(x, f(x)) \in E$ as desired, again by the definition of $Y_x$.
    }
  }
}

\exerciseapp{8}{*}{
  Prove that every uncountable set has a subset of cardinality $\al_1$.
}
\sol{
  This proof is similar to that of Theorem~8.1.4.
  \qproof{
    Let $A$ be an uncountable set.
    By the Well-Ordering Principle (which is equivalent to the Axiom of Choice by Theorem~8.1.13) $A$ can be well ordered, and so can be arranged in a bijective transfinite sequence $\angles{a_\a \where \a < \W}$ for some ordinal $\W$.
    Since $A$ is then equipotent to $\W$ it has to be that $\w_1 \leq \W$ since otherwise $\W$ (and therefore $A$) would be countable or finite.
    So then clearly the range of the transfinite sequence $\angles{a_\a \where \a < \w_1}$ is a subset of $A$ with cardinality $\al_1$.
  }
}

\exerciseapp{9}{*}{
  Every infinite set is equipotent to some of its proper subsets.
  Equivalently, Dedekind finite sets are precisely the finite sets.
}
\sol{
  \qproof{
    By Theorem~8.1.4, any infinite set has a countable subset so that such a set is Dedekind infinite by Exercise~5.1.10.
    Therefore any infinite set is equipotent to a proper subset of itself by the definition of Dedekind infinite as desired.
    In fact, any countable set (and by extension any infinite set) is equipotent to an infinite number of its proper subsets.
    To see this we note that the mapping $f(k) = k + n$ is a bijection from $\nats$ to $\nats - n$, which is clearly a proper subset of $\nats$, for any natural number $n$.

    Of course the contrapositive of this is that, if a set is Dedekind finite (i.e. \emph{not} Dedekind infinite), then the set is finite (i.e. \emph{not} finite).
  }
}

\exerciseapp{10}{*}{
  Let $(A, <)$ be a linearly ordered set.
  A sequence $\angles{a_n \where n \in \w}$ of elements of $A$ is \emph{decreasing} if $a_{n+1} < a_n$ for all $n \in \w$.
  Prove that $(A, <)$ is a well-ordering if and only if there exists no infinite decreasing sequence in $A$.
}
\sol{
  \qproof{
    \bicproof{
      We show the contrapositive of this implication.
      So suppose that there \emph{is} a decreasing sequence $\angles{a_n \where n \in \w}$ in $A$, and let $X$ be the range of the sequence so that clearly $X \ss A$ and $X \neq \es$.
      Now consider any $x \in X$ so that there is a $n \in \w$ such that $x = a_n$.
      We then have that $x = a_n > a_{n+1}$ since the sequence is decreasing, noting that clearly $a_{n+1} \in X$.
      Since $x$ was arbitrary this shows that $X$ has no least element (since we have shown that $\forall x \in X \exists y \in X (x > y)$ and this is logically equivalent to $\lnot \exists x \in X \forall y \in X (x \leq y)$, noting that $\lnot (x > y)$ is equivalent to $x \leq y$ since the ordering is linear).
      Thus, since there is a nonempty subset of $A$ that has no least element, it follows that $(A, <)$ is not a well-ordering.
    }{
      We show the contrapositive of this implication as well.
      So suppose that $(A, <)$ is \emph{not} a well-ordering.
      Then there exists a nonempty subset $X$ of $A$ such that $X$ has no least element.
      By the Axiom of Choice the set $\pset{X}$ has a choice function $g$.
      For any $x \in X$ we define the set $X_x = \braces{y \in X \where y < x}$.

      First we claim that $X_x \neq \es$ for any $x \in X$.
      Suppose to the contrary that there is some $x \in X$ such that $X_x = \es$.
      Consider any other $y \in X$.
      Then it cannot be that $y < x$ for then $y \in X_x$ so that $X_x \neq \es$.
      So, since the ordering is linear, it has to be that $y \geq x$.
      But, since $y$ was arbitrary, this would mean that $x$ is the least element of $X$, which we know cannot be since $X$ has no least element!
      Therefore it has to be that indeed $X_x \neq \es$ for any $x \in X$.

      Now we construct a sequence $\angles{a_n \where n \in \w}$ by recursion:
      \ali{
        a_0 &= g(X) \\
        a_{n+1} &= g(X_{a_n}),
      }
      noting that the recursive step is always valid since $X_{a_n}$ is never empty, as was just shown.
      It is easy to see that this is a decreasing sequence.
      Take any $n \in \w$ so that we have that $a_{n+1} = g(X_{a_n})$.
      Hence $a_{n+1} \in X_{a_n}$ since $g$ is a choice function.
      By the definition of $X_{a_n}$ it then follows that $a_{n+1} < a_n$, which shows that the sequence is decreasing since $n$ was arbitrary.
      Hence we have constructed an infinite decreasing sequence in $A$.
    }
  }
}

\def\FL{\bigcap_{t \in T} \parens{\bigcup_{s \in S} A_{t,s}}}
\def\FR{\bigcup_{f \in S^T} \parens{\bigcap_{t \in T} A_{t, f(t)}}}
\def\SL{\bigcup_{t \in T} \parens{\bigcap_{s \in S} A_{t,s}}}
\def\SR{\bigcap_{f \in S^T} \parens{\bigcup_{t \in T} A_{t, f(t)}}}
\exerciseapp{11}{*}{
  Prove the following distributive laws (see Exercise 3.13 in Chapter 2).
  \gath{
    \FL = \FR \\
    \SL = \SR
  }
}
\sol{
  First we show that
  \gath{
    \FL = \FR.
  }
  \qproof{
    First let
    \ali{
      L &= \FL & R &= \FR
    }
    so that we must show that $L = R$.

    \seteqproof{
      Consider any $x \in L$.
      For any $t \in T$, define the set $S_t = \braces{s \in S \where x \in A_{t,s}}$.
      Since $x \in L$ we have that $x \in \bigcup_{s \in S} A_{t,s}$ for all $t \in T$.
      Hence, for all $t \in T$, there is an $s \in S$ such that $x \in A_{t,s}$.
      From this it follows that $S_t \neq \es$ for any $t \in T$.
      Now, by the Axiom of Choice, the system of sets $\braces{S_t \where t \in T}$ has a choice function $g$.
      We then define a function $f(t) = g(S_t)$ for any $t \in T$, noting that this is defined for all $t \in T$ since $S_t$ is nonempty.
      Clearly then we have, for any such $t \in T$, that $f(t) = g(S_t) \in S_t \ss S$ so that $f(t) \in S$.
      Hence $f$ is a function from $T$ into $S$ so that $f \in S^T$.

      Now consider any specific $t \in T$ so that $f(t) \in S_t$ and hence $x \in A_{t,f(t)}$ by the definition of $S_t$.
      Thus since $t$ was arbitrary this shows that $x \in \bigcap_{t \in T} A_{t, f(t)}$.
      Moreover, since $f \in S^T$ we clearly have that $x \in \FR = R$.
      Thus, since $x$ was arbitrary, this shows that $L \ss R$.
    }{
      Now consider any $x \in R$ so that there is an $f \in S^T$ such that $x \in \bigcap_{t \in T} A_{t,f(t)}$.
      Thus we have that $x \in A_{t,f(t)}$ for all $t \in T$.
      So consider any $t \in T$ and let $s = f(t)$, noting that $s = f(t) \in S$ since $f \in S^T$.
      Thus we have that $x \in A_{t,f(t)} = A_{t,s}$.
      Since we have shown that there is an $s \in S$ such that $x \in A_{t,s}$, it follows that $x \in \bigcup_{s \in S} A_{t,s}$.
      Since $t \in T$ was arbitrary we have $x \in \FL = L$.
      Hence $R \ss L$ since $x$ was arbitrary.
    }
  }

  Now we show that
  \gath{
    \SL = \SR.
  }
  \qproof{
    First let
    \ali{
      L &= \SL & R &= \SR
    }
    so that we must show that $L = R$.

    \seteqproof{
      Consider any $x \in L$ so that there is a $t \in T$ such that $x \in \bigcap_{s \in S} A_{t,s}$.
      Hence we have that $x \in A_{t,s}$ for all $s \in S$.
      Now consider any $f \in S^T$.
      Then we have that $f(t) \in S$ so that $x \in A_{t,f(t)}$.
      Therefore there is a $t \in T$ such that $x \in A_{t,f(t)}$ so that $x \in \bigcup_{t \in T} A_{t,f(t)}$.
      Moreover, since $f$ was arbitrary, we have that $x \in \SR = R$.
      This shows that $L \ss R$ since $x$ was arbitrary.
    }{
      We show this by contrapositive.
      So suppose that $x \notin L$.
      Hence we have
      \gath{
        x \notin \SL \\
        \lnot \exists t \in T \parens{x \in \bigcap_{s \in S} A_{t,s}} \\
        \lnot \exists t \in T \forall s \in S \parens{x \in A_{t,s}} \\
        \forall t \in T \exists s \in S \parens{x \notin A_{t,s}}.
      }
      Now, let $S_t = \braces{s \in S \where x \notin A_{t,s}}$ for each $t \in T$, noting that $S_t \neq \es$ by the above for any $t \in T$.
      Then, by the Axiom of Choice, the system of sets $\braces{S_t \where t \in T}$ has a choice function $g$.
      We then set $f(t) = g(S_t)$ for any $t \in T$.
      Then clearly $f(t) = g(S_t) \in S_t \ss S$ for any $t \in T$ so that $f(t) \in S$ since $g$ is a choice function.
      It then follows that $f$ is a function from $T$ into $S$ so that $f \in S^T$.
      Now consider any $t \in T$ so that $f(t) = g(S_t) \in S_t$.
      Then, by the definition of $S_t$, we have that $x \notin A_{t,f(t)}$.
      Since $t \in T$ was arbitrary, we have thus shown that
      \gath{
        \exists f \in S^T \forall t \in T \parens{x \notin A_{t,f(t)}} \\
        \lnot \forall f \in S^T \exists t \in T \parens{x \in A_{t,f(t)}} \\
        \lnot \forall f \in S^T \parens{x \in \bigcup_{t \in T} A_{t,f(t)}} \\
        \lnot \squares{x \in \SR} \\
        x \notin \SR \\
        x \notin R.
      }
      Since $x$ was arbitrary, this shows by contrapositive that $x \in R$ implies $x \in L$ so that $R \ss L$.
    }
  }
}

\exerciseapp{12}{*}{
  Prove that for every ordering $\prece$ on $A$, there is a linear ordering $\leq$ on $A$ such that $a \prece b$ implies $a \leq b$ for all $a,b \in A$ (i.e., every partial ordering can be extended to a linear ordering).
}
\sol{
  \iffalse
    % These were part of an earlier attempted solution that did not work out
    \begin{defin}\label{def:aoc:tranclos}
      For a relation $R$ on a set $A$, the transitive closure $S$ of $R$ is a relation on $A$ such that
      \begin{enumerate}
        \item $R \ss S$,
        \item $S$ is transitive,
        \item For every relation $T \ss A \times A$, if $R \ss T$ and $T$ is transitive, then $S \ss T$.
      \end{enumerate}
      That is, $S$ is the smallest transitive relation (by $\ss$) such that $R \ss S$.
    \end{defin}

    \begin{thrm}\label{def:aoc:tranclosu}
      Suppose that $R$ is a relation on $A$ and let
      \gath{
        \famF = \braces{T \ss A \times A \where \text{$R \ss T$ and $T$ is transitive}}.
      }
      Then the set $S = \bigcap \famF$ is the unique transitive closure of $R$, noting that $\famF \neq \es$ since clearly $R \ss A \times A$ and $A \times A$ is transitive so that $A \times A \in \famF$.
      For this reason we say \emph{the} transitive closure rather than \emph{a} transitive closure.
      \qproof{
        First we show that $S$ satisfies part (1) in the definition of a transitive closure.
        So consider any $(x,y) \in R$ and let $T$ be an arbitrary element of $\famF$.
        Then $R \ss T$ by the definition of $\famF$ so that $(x,y) \in T$.
        Since $T$ was arbitrary this shows that $(x,y) \in \bigcap \famF = S$, and since $(x,y)$ was arbitrary this shows that $R \ss S$ as desired.

        To show (2) consider any $(x,y)$ and $(y,z)$ in $S = \bigcap \famF$.
        Then both $(x,y)$ and $(y,z)$ are in every $T \in \famF$.
        Hence, for any arbitrary $T \in \famF$, clearly $(x,y) \in T$ and $(y,z) \in T$.
        Then, since $T \in F$ we have that $T$ is transitive (by the definition of $\famF$) so that $(x,z) \in T$.
        Since $T$ was arbitrary this shows that $(x,z) \in \bigcap \famF = S$ so that $S$ is transitive.

        Now, to show part (3), consider any relation $T$ on $A$ where $R \ss T$ and $T$ is transitive.
        Consider then any $(x,y) \in S = \bigcap \famF$ so that $(x,y) \in T'$ for every $T' \in \famF$.
        In particular, $T \in \famF$ since $R \ss T$ and $T$ is transitive so that $(x,y) \in T$.
        Since $(x,y)$ was arbitrary this shows that $S \ss T$ as desired.

        Lastly, to show that the transitive closure is unique let $S$ and $T$ be transitive closures of $R$.
        Then, since $R \ss S$ (by part (1) for $S$) and $S$ is transitive (by part (2) for $S$), we have that $T \ss S$ (by part (3) for $T$).
        By an analogous argument $S \ss T$, which shows that $S = T$ so that the transitive closure is unique.
      }
    \end{thrm}
  \fi

  \qproof{
    \iffalse
      % This was another attempted solution that did not work out
      \def\tleq{\trianglelefteq}
      \newcommand\lord[1]{(#1, \leq_{#1})}
      \def\Cb{\overline{C}}
      \def\Ab{\overline{A}}

      Let $(A, \prece)$ be a partially ordered set.
      We define the set
      \ali{
        L = \{\lord{B} \where & \text{$B \ss A$ and $\leq_B$ is a linear order on $B$ such that} \\
        & \text{$x \prece y$ implies $x \leq_B y$ for all $x$ and $y$ in $B$}\}
      }
      Moreover, we define the following relation $\tleq$ on $L$:
      \gath{
        \text{$\lord{B} \tleq \lord{C}$ if and only if $B \ss C$ and $\leq_B \ss \leq_C$}
      }
      for any $\lord{B}$ and $\lord{C}$ in $L$.

      It is easy to see that $\tleq$ is a partial order on $L$.
      Consider any $\lord{B} \in L$.
      Then, since obviously $B \ss B$ and $\leq_B \ss \leq_B$ we have by definition that $\lord{B} \tleq \lord{B}$ so that $\tleq$ is reflexive since $\lord{B}$ was arbitrary.
      Now consider any $\lord{B}$ and $\lord{C}$ in $L$ where $\lord{B} \tleq \lord{C}$ and $\lord{C} \tleq \lord{B}$.
      Then by definition we have that $B \ss C$ and $\leq_B \ss \leq_C$ as well as $C \ss B$ and $\leq_C \ss \leq_B$.
      From this it clearly follows that $B = C$ and $\leq_B = \leq_C$ so that $\lord{B} = \lord{C}$, which shows that $\tleq$ is antisymmetric.
      Lastly, consider any $\lord{B}$, $\lord{C}$, and $\lord{D}$ in $L$ where $\lord{B} \tleq \lord{C}$ and $\lord{C} \tleq \lord{D}$.
      Then we have $B \ss C$ and $\leq_B \ss \leq_C$ as well as $C \ss D$ and $\leq_C \ss \leq_D$.
      Since we know that $\ss$ is transitive we then have that $B \ss D$ and $\leq_B \ss \leq_D$ so that $\lord{B} \tleq \lord{D}$ by definition.
      This shows that $\tleq$ is transitive as well and therefore a partial order on $L$.

      Now we claim that every $\tleq$-chain of $L$ has an upper bound.
      So let $\Cb$ be such a chain of $(L, \tleq)$ and let
      \ali{
        U &= \bigcup_{\lord{B} \in \Cb} B &
        \leq_U &= \bigcup_{\lord{B} \in \Cb} \leq_B.
      }
      Now consider any $x \in U$ so that there is a $\lord{B} \in \Cb$ such that $x \in B$.
      Then, since $\lord{B} \in \Cb$ and $\Cb \ss L$ we have that $\lord{B} \in L$ so that $B \ss A$ by definition.
      Hence since $x \in B$ it follows that $x \in A$.
      Since $x$ was arbitrary this shows that $U \ss A$.

      Now consider any $x$ and $y$ in $U$.
      Then there is are $\lord{B}$ and $\lord{C}$ in $\Cb$ such that $x \in B$ and $y \in C$.
      Now, since $\Cb$ is a $\tleq$-chain we either have that $\lord{B} \tleq \lord{C}$ or $\lord{C} \tleq \lord{B}$.
      In the former case we have that $B \ss C$ and $\leq_B \ss \leq_C$.
      Then since $x \in B$ and $B \ss C$ we have $x \in C$.
      Since also $y \in C$ and $\leq_C$ is a linear order on $C$ it follows that $x$ and $y$ are comparable in $\leq_C$.
      Hence $(x,y) \in \leq_C$ or $(y,x) \in \leq_C$.
      Since $\lord{C} \in \Cb$ it then follows that $(x,y) \in \leq_U$ or $(y,x) \in \leq_U$.
      This shows that $\leq_U$ is a linear order on $U$ since $x$ and $y$ were arbitrary.
      The latter case in which $\lord{C} \tleq \lord{B}$ is analogous.

      Lastly, consider any $x$ and $y$ in $U$ where $x \prece y$.
      Then, again, there are $\lord{B}$ and $\lord{C}$ in $\Cb$ such that $x \in B$ and $y \in C$.
      Since $\Cb$ is a chain, by the same argument as above, we have that either $x$ and $y$ are both in $B$ or they are both in $C$.
      In the former case, since $\lord{B} \in \Cb$ and $\Cb \ss L$, we have $\lord{B} \in L$ so that by definition $x \leq_B y$ since $x$ and $y$ are both in $B$ and $x \prece y$.
      Hence $(x,y) \in \leq_B$ so that, by the definition of $\leq_U$ we have that $(x,y) \in \leq_U$.

      Thus we have shown that $U \ss A$, $\leq_U$ is a linear order on $U$, and $x \prec y$ implies that $x \leq_U y$ for all $x$ and $y$ in $U$.
      This shows by definition that $\lord{U} \in L$.

      Now consider any $\lord{B}$ in $\Cb$.
      Suppose arbitrary $x \in B$ and $(y,z) \in \leq_B$.
      Then clearly $x \in U$ and $(y,z) \in \leq_U$ by definition.
      Hence $B \ss U$ and $\leq_B \ss \leq_U$ so that by definition $\lord{B} \tleq \lord{U}$.
      Since $\lord{B}$ was arbitrary this shows that $\lord{U}$ is an upper bound of $\Cb$.

      Since $\Cb$ was an arbitrary chain of $(L, \tleq)$, this proves the conditions of Zorn's Lemma so that there is a maximal element of $(L, \tleq)$.
      Call this maximal element $(\Ab, \leq)$.
      Now, since $(\Ab, \leq) \in L$ we have that $\Ab \ss A$.
      For a moment, suppose that $A$ is not a subset of $\Ab$ so that there is an $a \in A$ such that $a \notin \Ab$.
      Clearly $\leq_a = \braces{(a,a)}$ is a trivial linear order on $\braces{a}$.
      Moreover, obviously $\leq_a$ has the property that $x \leq_a y$ for any $x$ and $y$ in $\braces{a}$ where $x \prece y$.\
      Hence it follows that $(\braces{a}, \leq_a) \in L$.
    \fi

    Let $(A, \prece)$ be a partially ordered set.
    Define the set
    \gath{
      P = \braces{R \ss A \times A \where \text{$R$ is a partial order of $A$ and $\prece \ss R$}}.
    }
    We know that $\ss$ partially orders $P$.
    Now consider any $\ss$-chain $C$ of $P$.
    If $C = \es$ then, since clearly we have that $\prece \in P$, we have that $\es \ss \prece$ so that $\prece$ is an upper bound of $C$.
    On the other hand if $C \neq \es$ then there is an $R \in C$, noting that $C \ss P$ so that $R \in P$ as well.
    Then, by the definition of $P$, $R$ is a partial order on $A$ such that $\prece \ss R$.
    Let $U = \bigcup C$. We first show that $U$ is a partial order on $A$.

    So consider any $a \in A$.
    Then $(a,a) \in R$ since $R$ is a partial order of $A$ (and therefore reflexive).
    Hence clearly then $(a,a) \in \bigcup C = U$ since $R \in C$ so that $U$ is reflexive since $a$ was arbitrary.

    Now consider any $x$ and $y$ in $A$ where $(x,y) \in U$ and $(y,x) \in U$.
    Then, since $U = \bigcup C$, there are $S \in C$ and $T \in C$ such that $(x,y) \in S$ and $(y,x) \in T$.
    Since $C$ is a $\ss$-chain, we have that either $S \ss T$ or $T \ss S$.
    In the former case then we have that both $(x,y) \in T$ (since $(x,y) \in S$ and $S \ss T$) and $(y,x) \in T$ so that $x=y$ since $T$ is a partial order of $A$ (since $T \in C$ and $C \ss P$) and is therefore antisymmetric.
    The case in which $T \ss S$ is analogous.
    This shows that $U$ is antisymmetric.

    Lastly, consider $x$, $y$, and $z$ in $A$ such that $(x,y) \in U$ and $(y,z) \in U$.
    Then, again, we have that there are $S \in C$ and $T \in C$ such that $(x,y) \in S$ and $(y,z) \in T$.
    We also again have that $S \ss T$ or $T \ss S$ since $C$ is a $\ss$-chain.
    In the former case we have that $(x,y) \in T$ (since $(x,y) \in S$ and $S \ss T$) and $(y,z) \in T$ so that $(x,z) \in T$ since $T$ is a partial order on $A$ (again since $T \in C$ and $C \ss P$) and therefore transitive.
    Hence, since $T \in C$, clearly we have $(x,z) \in \bigcup C = U$ so that $U$ is transitive.
    The case in which $T \ss S$ is again analogous.

    Therefore we have shown that $U$ is reflexive, antisymmetric, and transitive, and is therefore a partial order on $A$ by definition.
    Now consider any $(x,y) \in \prece$.
    Then $(x,y) \in R$ since $\prece \ss R$.
    Hence $(x,y) \in \bigcup C = U$ since $R \in C$.
    It then follows that $\prece \ss U$ since $(x,y)$ was arbitrary.
    Thus $U$ is a partial order on $A$ and $\prece \ss U$ so that by definition $U \in P$.

    Now consider any $S \in C$ and any $(x,y) \in S$.
    Then clearly $(x,y) \in \bigcup C = U$ so that $S \ss U$ since $(x,y)$ was arbitrary.
    Since $S$ was arbitrary this shows that $U$ is in fact an upper bound of $C$ with respect to $\ss$.

    Since the chain $C$ was arbitrary, this shows that all $\ss$-chains of $P$ have an upper bound so that the conditions of Zorn's Lemma are satisfied.
    We can therefore conclude that $P$ has a $\ss$-maximal element $\leq$.

    We claim that $\leq$ is a linear ordering of $A$.
    So assume to the contrary that $\leq$ is not linear so that there are $a$ and $b$ in $A$ such that $(a,b) \notin \leq$ and $(b,a) \notin \leq$.
    Then define the relations
    \gath{
      R' = \braces{(x,y) \in A \times A \where x \leq a \text{ and } b \leq y}
    }
    and $R = \leq \cup R'$.
    First, notice that $(a,a) \in \leq$ and $(b,b) \in \leq$ since $\leq$ is reflexive so that by definition $(a,b) \in R'$ and hence $(a,b) \in R$.
    We claim that $R \in P$ so that we must first show that $R$ is an order on $A$.

    Consider any $x \in A$ so that clearly $(x,x) \in \leq$ since $\leq$ is an ordering of $A$ and is therefore reflexive.
    Hence clearly $(x,x) \in \leq \cup R' = R$ so that $R$ is reflexive.

    Now suppose any $x$ and $y$ in $A$ where $(x,y) \in R$ and $(y,x) \in R$.
    Then clearly $(x,y) \in \leq$ or $(x,y) \in R'$ and similarly $(y,x) \in \leq$ or $(y,x) \in R'$

    Case: $(x,y) \in \leq$. If $(y,x) \in \leq$ then clearly $x=y$ since $\leq$ is an order on $A$ and therefore is antisymmetric.
    The sub-case in which $(y,x) \in R'$ cannot be since, if it were, then we would have $y \leq a$ and $b \leq x$.
    Hence we would have that $b \leq x \leq y \leq a$ so that $b \leq a$ by transitivity, which we know is not possible by our definition of $a$ and $b$.

    Case: $(x,y) \in R'$.
    Then $x \leq a$ and $b \leq y$.
    Here again the sub-case in which $(y,x) \in \leq$ is impossible since we would then have $b \leq y \leq x \leq a$ so that $b \leq a$ by transitivity.
    Evidently the sub-case in which $(y,x) \in R'$ is also impossible since then we would have $y \leq a$ and $b \leq x$ so that $b \leq x \leq a$ as well as $b \leq y \leq a$ so that $b \leq a$ and $a \leq b$, which we know is not possible.
    Hence this entire case is not possible.

    Thus, in the only valid case, we have that $x=y$ so that $R$ is antisymmetric.

    Now suppose $x$, $y$, and $z$ in $A$ where $(x,y) \in R$ and $(y,z) \in R$.
    Then we have that $(x,y) \in \leq$ or $(x,y) \in R'$ and similarly $(y,z) \in \leq$ or $(y,z) \in R'$.

    Case: $(x,y) \in \leq$.
    If $(y,z) \in \leq$ also then clearly we have $(x,z) \in \leq$ since $\leq$ is an order and therefore transitive.
    If, on the other hand, $(y,z) \in R'$ then we have $y \leq a$ and $b \leq z$.
    Hence we have that $x \leq y \leq a$ so that $x \leq a$ by transitivity.
    We thus have $x \leq a$ and $b \leq z$ so that $(x,z) \in R'$ by definition.

    Case: $(x,y) \in R'$.
    Then we have $x \leq a$ and $b \leq y$.
    If $(y,z) \in \leq$ then we have $b \leq y \leq z$ so that $b \leq z$ by transitivity.
    Hence $x \leq a$ and $b \leq z$ so that $(x,z) \in R'$ by definition.
    On the other hand, if $(y,z) \in R'$, then we have $y \leq a$ and $b \leq z$.
    Hence $b \leq y \leq a$ so that $b \leq a$ by transitivity, which we know is not true.
    Therefore this sub-case is impossible.

    Hence in all valid cases we have that $(x,z) \in \leq$ or $(x,z) \in R'$ so that clearly $(x,z) \in R = \leq \cup R'$, thereby showing that $R$ is transitive.

    Therefore we have shown that $R$ is an ordering of $A$.
    We also clearly have that $\prece \ss R$ since $\prece \ss \leq$ (since $\leq \in P$) and $\leq \ss R$ (since $R = \leq \cup R'$).
    Thus indeed $R \in P$.

    Now, since $R = \leq \cup R'$ we clearly have that $\leq \ss R$.
    We also know that $(a,b) \in R$ but $(a,b) \notin \leq$ so that $R \neq \leq$, and hence $\leq \pss R$.
    However, this contradicts the fact that $\leq$ is a maximal element of $P$ so that it must be that $\leq$ is in fact linear!

    Lastly, consider any $a$ and $b$ in $A$ where $a \prece b$.
    Then $(a,b) \in \prece$ so that $(a,b) \in \leq$ because $\prece \ss \leq$ since $\leq \in P$.
    Thus $a \leq b$ so that $\leq$ is the $\prece$-extended linear ordering of $A$ that we seek.
  }
}

\exerciseapp{13}{*}{
  (Principle of Dependent Choices)
  If $R$ is a binary relation on $M \neq \es$ such that for each $x \in M$ there is a $y \in M$ for which $xRy$, then there is a sequence $\angles{x_n \where n \in \w}$ such that $x_n R x_{n+1}$ holds for all $n \in \w$.
}
\sol{
  \qproof{
    Suppose such a relation $R$ on $M \neq \es$.
    Define the set $X_x = \braces{y \in M \where xRy}$ for each $x \in M$.
    Then, by the given property of $R$, clearly $X_x \neq \es$ for any $x \in M$.
    Then, by the Axiom of Choice, the system of sets $\braces{X_x \where x \in M}$ has a choice function $g$.
    We also have that there is an $a \in M$ since $M \neq \es$.
    We then define a sequence recursively as follows:
    \ali{
      x_0 &= a \\
      x_{n+1} &= g(X_{x_n}),
    }
    noting that this is well defined since each $X_n$ is nonempty.

    To show that the sequence $\angles{x_n \where n \in \w}$ has the desired property, consider any $n \in \w$.
    Then by the recursive definition we have that $x_{n+1} = g(X_{x_n})$ so that $x_{n+1} \in X_{x_n}$ since $g$ is a choice function and $X_{x_n}$ is nonempty.
    Then, from the definition of $X_{x_n}$, it follows that $x_{n} R x_{n+1}$.
    This shows the desired result since $n$ was arbitrary.
  }
}

\exercise{14}{
  Assuming only the Principle of Dependent Choices, prove that every countable system of sets has a choice function (the Axiom of Countable Choice).
}
\sol{
  \qproof{
    Suppose that $S$ is a countable system of sets and let $T = \braces{X \in S \where X \neq \es}$.
    Then it suffices to show that there is a choice function $h$ for $T$.
    To see this suppose that there is such a choice function $h$ and define a function $g$ on $S$ by
    \gath{
      g(X) =
      \begin{cases}
        \es  & X = \es    \\
        h(X) & X \neq \es
      \end{cases},
    }
    for any $X \in S$, noting that clearly $X \in T$ if $X \neq \es$ so that $X$ is in the domain of $h$.
    Then, for any nonempty $X \in S$, we have that $g(X) = h(X) \in X$ since $h$ is a choice function.
    Hence $g$ is a choice function on $S$.

    Now we construct a choice function $h$ for $T$.
    First note that clearly either $T$ is finite or countable since $T \ss S$ and $S$ is countable.
    If $T$ is finite then it has a choice function $h$ by Theorem~8.1.2, so we shall assume that $T$ is also countable.
    It then follows that $T$ can be written as a sequence of sets $\angles{X_n \where n \in \w}$.

    Now define the the binary relation $R$ on $\bigcup T$ by
    \gath{
      R = \braces{(x,y) \in \bigcup T \times \bigcup T \where \text{$x \in X_n$ and $y \in X_{n+1}$ for some $n \in \w$}}.
    }
    First, since $X_0 \in T$, it is nonempty so that there is an $x \in X_0$.
    Then clearly $x \in \bigcup T$ so that $\bigcup T$ is nonempty.
    Now consider any $x \in \bigcup T$ so that there is an $n \in \w$ such that $x \in X_n$.
    We then have that $X_{n+1} \neq \es$ since $X_{n+1} \in T$ so that there is a $y \in X_{n+1}$, noting that clearly also $y \in \bigcup T$.
    It then follows from the definition of $R$ that $xRy$.
    Since $x$ was arbitrary this shows that the relation $R$ on $\bigcup T$ meets the conditions of the Principle of Dependent Choices.
    Thus we can conclude that there is a sequence $\angles{x_n \where n \in \w}$ of elements of $\bigcup T$ where $x_n R x_{n+1}$ holds for any $n \in \w$.

    Now, since we have that $x_0 \in \bigcup T$, there is an $m \in \w$ such that $x_0 \in X_m$.
    We claim that $x_n \in X_{m+n}$ for all $n \in \w$, which we show by induction on $n$.
    First, for $n = 0$, we already know that $x_n = x_0 \in X_m = X_{m+0} = X_{m+n}$.
    Now suppose that $x_n \in X_{m+n}$.
    By the property of the sequence $\angles{x_k}_{k \in \w}$ we know that $x_n R x_{n+1}$ so that, by the definition of $R$, we have that $x_{n+1} \in X_{(m+n)+1} = X_{m + (n+1)}$ since clearly $m+n \in \w$ and $x_n \in X_{m+n}$.
    This completes the induction.

    Now consider the set $T' = \braces{X_n \where n \in m}$, which is clearly a finite system of nonempty sets.
    Hence by Theorem~8.1.2 it has a choice function $h'$.
    Now we define a function $h: T \to \bigcup T$.
    For any $X \in T$ we know that there is an $n \in \w$ such that $X = X_n$.
    So we set
    \gath{
      h(X) = h(X_n) =
      \begin{cases}
        h'(X_n) & n < m    \\
        x_{n-m} & n \geq m
      \end{cases}.
    }
    We claim that $h$ is a choice function on $T$.
    So consider any $X \in T$ (which automatically means that $X \neq \es$) so that again $X = X_n$ for some $n \in \w$.
    If $n < m$ then we have that $h(X) = h'(X_n) \in X_n = X$ since $h'$ is a choice function.
    On the other hand, if $n \geq m$, then we note that $n-m \in \w$ and we have $h(X) = x_{n-m} \in X_{m+(n-m)} = X_n = X$ by what was shown above.
    Since $X$ was arbitrary this shows that $h$ is a choice function on $T$.
    Hence, as shown above this means that there is a choice function $g$ on $S$ as desired.
  }
}

\exercise{15}{
  If every set is equipotent to an ordinal number, then the Axiom of Choice holds.
}
\sol{
  \qproof{
    Let $A$ be any set and $\a$ be an ordinal equipotent to $A$.
    Then there is a bijection $f$ from $A$ to $\a$.
    We can then simply order $A$ according to $f$, that is order it by the relation
    \gath{
      \prece = \braces{(x,y) \in A \times A \where f(x) \leq f(y)}
    }
    Clearly then $(A, \prece)$ is isomorphic to $(\a, \leq)$ so that it is a well-ordering.
    Hence $A$ can be well-ordered.
    Since $A$ was an arbitrary set, this shows the Well-Ordering Principle, which is equivalent to the Axiom of Choice by Theorem~8.1.13.
  }
}

\exercise{16}{
  If for any sets $A$ and $B$ either $\abs{A} \leq \abs{B}$ or $\abs{B} \leq \abs{A}$, then the Axiom of Choice holds.
    [Hint: Compare $A$ and $B = h(A)$.]
}
\sol{
  \qproof{
    Consider any set $A$.
    Then we have that either $\abs{A} \leq \abs{h(A)}$ or $\abs{h(A)} \leq \abs{A}$ (where $h(A)$ denotes the Hartogs number of $A$).
    Now, it cannot be that $\abs{h(A)} \leq \abs{A}$.
    For, if it were, then there would be an injection $f$ from $h(A)$ into $A$.
    Then clearly $\ran{f} \ss A$ and $f$ is a bijection from $h(A)$ to $\ran{f}$.
    Thus $h(A)$ is equipotent to $\ran{f} \ss A$, which violates the definition of the Hartogs number.

    Therefore it must be that $\abs{A} \leq \abs{h(A)}$.
    Hence there is an injection $g$ from $A$ into $h(A)$.
    Since $h(A)$ is an ordinal and $\ran{g} \ss h(A)$, it follows that $\ran{g}$ is a set of ordinals, which is well-ordered by Theorem~6.2.6d.
    So, ordering $A$ according to $g$ (considered as a bijection from $A$ to $\ran{g}$) results in a well-ordering of $A$.
    Since $A$ was an arbitrary set, this shows the Well-Ordering Principle, from which the Axiom of Choice follows by Theorem~8.1.13.
  }
}

\exerciseapp{17}{*}{
  If $B$ is an infinite set and $A$ is a subset of $B$ such that $\abs{A} < \abs{B}$, then $\abs{B-A} = \abs{B}$.
}
\sol{
  \qproof{
    Since $B$ is infinite, it follows Theorem~8.1.5 that there is a unique ordinal $\a$ such that $\abs{B} = \al_\a$, noting that this requires the Axiom of Choice.
    Thus there is a bijection $f$ from $B$ onto $\w_\a$.
    Clearly then, since $A \ss B$, we have that $f[A] \ss \w_\a$.
    Moreover, clearly $f$ is a bijection from $A$ to $f[A]$ so that $\abs{f[A]} = \abs{A} < \abs{B} = \al_\a$.
    It therefore follows from Exercise~7.2.6 that $\abs{\w_\a - f[A]} = \al_\a$.
    We now claim that $f[B - A] = \w_\a - f[A]$.

    \seteqproof{
      So consider any $y \in f[B - A]$ so that there is an $x \in B - A$ such that $y = f(x)$.
      Since clearly $x \in B$, we have that $y = f(x) \in \w_\a$ since $f$ maps $B$ onto $\w_\a$.
      Now suppose that also $y \in f[A]$ so that there is a $z \in A$ where $y = f(z)$.
      Then we have that $y = f(x) = f(z)$ so that $x = z$ since $f$ is injective.
      Hence we have $x = z \in A$ but also $z = x \notin A$ since $z = x \in B - A$, which is a contradiction.
      So it must be that $y \notin f[A]$ so that indeed $y \in \w_\a - f[A]$.
      Therefore $f[B - A] \ss \w_\a - f[A]$ since $y$ was arbitrary.
    }{
      Now consider any $y \in \w_\a - f[A]$ so that $y \in \w_\a$ and $y \notin f[A]$.
      Since $f$ is onto $\w_\a$, there is an $x \in B$ such that $y = f(x)$.
      Moreover, since $y \notin f[A]$ we can be sure that $x \notin A$.
      Therefore $x \in B - A$.
      Since $y = f(x)$ it therefore follows that $y \in f[B - A]$ so that $\w_\a - f[A] \ss f[B - A]$ since $y$ was arbitrary.
    }

    Thus we have shown that $f[B - A] = \w_\a - f[A]$ so that clearly $f$ is a bijection from $B - A$ onto $\w_\a - f[A]$.
    Hence we have that $\abs{B - A} = \abs{\w_\a - f[A]} = \al_\a = \abs{B}$ as desired.
  }
}
