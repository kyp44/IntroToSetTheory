\subsection{The Axioms}

\exercise{1}{
  Show that the set of all $x$ such that $x \in A$ and $x \notin B$ exists.
}
\sol{
  \qproof{
    $A$ is a given an extant set.
    So let $\propP{x, B}$ be the property that $x \in A$ and $x \notin B$ so that clearly $\propP{x, B}$ implies that $x \in A$.
    Then, by was expounded just before Example~1.3.13, the set
    \gath{
      \braces{x \where \propP{x, B}} = \braces{x \where x \in A \land x \notin B}
    }
    uniquely exists, which is of course the set we seek.
  }
}

\exercise{2}{
  Replace the Axiom of Existence by the following weaker postulate:

  Weak Axiom of Existence: Some set exists.

  Prove the Axiom of Existence using the Weak Axiom of Existence and the Comprehension Schema.
  [Hint: Let $A$ be a set known to exist; consider $\braces{x \in A \where x \neq x}$.]
}
\sol{
  \qproof{
    Invoking the Weak Axiom of Existence, suppose that $A$ is a set that exists.
    Let $\propP(x)$ denote the proprty $x \neq x$, noting that clearly $P(x)$ is false no matter the $x$, since otherwise identity would be violated.
    Then the set $B = \braces{x \in A \where \propP{x}}$ uniquely exists by the Axiom Schema of Comprehension and Lemma~1.3.4.

    Now consider any $x$.
    If $x \notin A$, then clearly $x \notin B$ either.
    If $x \in A$, then again $x \notin B$ since the property $\propP{x}$ is false (since it is false for \emph{any} $x$).
    Hence, in both cases $x \notin B$, which shows that
    \gath{
      \forall x (x \notin B) \bic \lnot \exists x (x \in A)
    }
    since $x$ was arbitrary.
    This of course means that $B = \es$ is the unique empty set, proving the Axiom of Existence since $B$ was shown to exist.
  }
}

\exercise{3}{
  (a) Prove that a ``set of all sets'' does not exist.
  [Hint: if $V$ is a set of all sets, consider $\braces{x \in V \where x \notin x}$.]
  
  (b) Prove that for any set $A$ there is some $x \notin A$.
}
\sol{
  (a) \qproof{
    Suppose that a set of all sets exists, and denote such a set by $V$.
    Then, since the property $\propP{x}$ defined by $x \notin x$ is a perfectly valid property, the set $W = \braces{x \in V \where \propP{x}}$ uniquely exists by the Axiom Schema of Comprehension and Lemma~1.3.3.
    Of course, it must be that either $W \in W$ or $W \notin W$, noting that $W \in V$ since $W$ is a set.
    If $W \in W$ then it must be that $\propP{W}$ by the definition of $W$ since $W \in V$.
    However, $\propP{W}$ means that $W \notin W$, which is a contradiction.
    Hence, it must be that $W \notin W$, but in this case $\propP{W}$ is true so that $W \in W$ by definition since $W \in V$, which is again a contradiction.
    Since a contradiction occurs no matter what, it must be that our original supposition that a set of all sets exists is untrue.
  }

  (b) \qproof{
    Suppose that $A$ is a set.
    Jumping ahead for a minute, Exercise~1.3.6 below shows that $\pset{A}$, which uniquely exists by the Axiom of Power Set, is not a subset of $A$.
    Therefore, there must be some $x \in \pset{A}$ such that $x \notin A$, which proves our result.
    This is because, by the definition of subset, we have the following logical equivalences:
    \ali{
      \pset{A} \not\ss A &\bic \lnot \forall x \squares{x \in \pset{A} \imp x \in A} \\
      &\bic \exists x \lnot \squares{x \in \pset{A} \imp x \in A} \\
      &\bic \exists x \lnot \squares{x \notin \pset{A} \lor x \in A} \\
      &\bic \exists x \squares{x \in \pset{A} \land x \notin A}.
    }
  }
}

\exercise{4}{
  Let $A$ and $B$ be sets.
  Show that there exists a unique set $C$ such that $x \in C$ if and only if either $x \in A$ and $x \notin B$ or $x \in B$ and $x \notin B$.
}
\sol{
  TODO
}

\exercise{5}{
  (a) Given $A$, $B$, and $C$, there is a set $P$ such that $x \in P$ if and only if $x = A$ or $x = B$ or $x = C$.

  (b) Generalize to four elements.
}
\sol{
  TODO
}

\exercise{6}{
  Show that $\pset{X} \ss X$ is false for any $X$.
  In particular, $\pset{X} \neq X$ for any $X$.
  This proves again that a ``set of all sets'' does not exist.
  [Hint: Let $Y = \braces{u \in X \where u \notin u}$; $Y \in \pset{X}$ but $Y \notin X$.]
}
\sol{
  TODO
}

\exercise{7}{
  The Axiom of Pair, the Axiom of Union, and the Axiom of Power Set can be replaced by the following weaker versions.

  Weak Axiom of Pair: For any $A$ and $B$, there is a set $C$ such that $A \in C$ and $B \in C$.
  
  Weak Axiom of Union: For any $S$, there exists $U$ such that if $X \in A$ and $A \in S$, then $X \in U$.

  Weak Axiom of Power Set: For any set $S$, there exists $P$ such that $X \ss S$ implies $X \in P$.

  Prove the Axiom of Pair, the Axiom of Union, and the Axiom of Power Set using these weaker versions.
  [Hint: Use also the Comprehension Schema.]
}
\sol{
  TODO
}
