\subsection{Infinite Sums and Products of Cardinal Numbers}

\newcommand\unaj[1]{\bigcup_{j \in J_{#1}} A_j}
\exercise{1}{
  If $J_i$ ($i \in I$) are mutually disjoint sets and  $J = \bigcup_{i \in I} J_i$, and if $\k_j$ ($j \in J$) are cardinals, then
  \gath{
    \sum_{i \in I} \parens{ \sum_{j \in J_i} \k_j } = \sum_{j \in J} \k_j
  }
  (\emph{associativity} of $\sum$)
}
\sol{
  \qproof{
  Suppose that $\angles{A_j \where j \in J}$ are mutually disjoint sets where $\abs{A_j} = \k_j$ for every $j \in J$.
  Then, by definition, we have that
  \begin{gather}
    \sum_{j \in J} \k_j = \abs{\unaj{}} \,. \label{eqn:infsp:sumkjall}
  \end{gather}
  Now let $S = \braces{J_i \where i \in I}$, from which it is trivial to show that $\bigcup S = J$.
  It follows from Exercise~2.3.10 that
  \begin{gather}
    \unaj{} = \bigcup_{a \in \bigcup S} A_a = \bigcup_{C \in S} \parens{\bigcup_{a \in C} A_a}
    = \bigcup_{i \in I} \parens{\unaj{i}} \,. \label{eqn:infsp:dun}
  \end{gather}
  We claim that the sets $\angles{\unaj{i} \where i \in I}$ are mutually disjoint.
  So consider any $i_1$ and $i_2$ in $I$ where $i_1 \neq i_2$, and suppose that $\unaj{i_1}$ and $\unaj{i_2}$ are \emph{not} disjoint so that there is an $x$ where $x \in \unaj{i_1}$ and $x \in \unaj{i_2}$.
  Then there is a $j_1 \in J_{i_1}$ where $x \in A_{j_1}$ and a $j_2 \in J_{i_2}$ where $x \in A_{j_2}$.
  Now, since $\angles{J_i \where i \in I}$ are mutually disjoint and $i_1 \neq i_2$, it follows that $J_{i_1}$ and $J_{i_2}$ are disjoint.
  Therefore it has to be that $j_1 \neq j_2$ (since $j_1 \in J_{i_1}$ and $j_2 \in J_{i_2}$).
  But then $A_{j_1}$ and $A_{j_2}$ are not disjoint (since $x$ is in both) despite the fact that $j_1 \neq j_2$, which contradicts the fact that $\angles{A_j \where j \in J}$ are mutually disjoint.
  So it must be that in that $\unaj{i_1}$ and $\unaj{i_2}$ are disjoint, which proves the result since $i_1$ and $i_2$ were arbitrary.

  Since $\angles{\unaj{i} \where i \in I}$ have been shown to be mutually disjoint, it follows by definition that
  \begin{gather}
    \sum_{i \in I} \abs{\unaj{i}} = \abs{\bigcup_{i \in I} \parens{\unaj{i}}} \,. \label{eqn:infsp:sumunaj}
  \end{gather}
  Lastly, we also clearly have that $\angles{A_j \where j \in J_i}$ are mutually disjoint for any $i \in I$ (since $\angles{A_j \where j \in J}$ are mutually disjoint) so that
  \begin{gather}
    \sum_{j \in J_i} \k_j = \abs{\unaj{i}} \,. \label{eqn:infsp:sumkj}
  \end{gather}

  Putting this all together, we have
  \ali{
    \sum_{j \in J} \k_j &= \abs{\unaj{}} & \text{(by \eqref{eqn:infsp:sumkjall})} \\
    &= \abs{\bigcup_{i \in I} \parens{\unaj{i}}} & \text{(by \eqref{eqn:infsp:dun})} \\
    &= \sum_{i \in I} \abs{\unaj{i}} & \text{(by \eqref{eqn:infsp:sumunaj})} \\
    &= \sum_{i \in I} \parens{\sum_{j \in J_i} \k_j} & \text{(by \eqref{eqn:infsp:sumkj})}
  }
  as desired.
  }
}

\exercise{2}{
  If $\k_i \leq \l_i$ for all $i \in I$ then $\sum_{i \in I} \k_i \leq \sum_{i \in I} \l_i$.
}
\sol{
  \def\un{\bigcup_{i \in I}}
  \qproof{
    Suppose that $\angles{A_i \where i \in I}$ are mutually disjoint sets such that $\abs{A_i} = \k_i$ for all $i \in I$.
    Similarly, suppose that $\angles{B_i \where i \in I}$ are mutually disjoint sets such that $\abs{B_i} = \l_i$ for all $i \in I$.
    It then follows by definition that
    \ali{
      \sum_{i \in I} \k_i &= \abs{\un A_i} &
      \sum_{i \in I} \l_i &= \abs{\un B_i} \,.
    }
    Now, we have $\abs{A_i} = \k_i \leq \l_i = \abs{B_i}$ so that there is an injective function $f_i : A_i \to B_i$ for all $i \in I$.
    With the help of the Axiom of Choice, we can choose one of these functions for each $i \in I$ and form the system of functions $\braces{f_i}_{i \in I}$.

    We claim that $\braces{f_i}_{i \in I}$ is a compatible system of functions.
    To see this, consider any $i_1$ and $i_2$ in $I$.
    If $i_1 = i_2$ then consider any $x \in \dom(f_{i_1}) \cap \dom(f_{i_2}) = A_{i_1} \cap A_{i_2} = A_{i_1} \cap A_{i_1} = A_{i_1}$.
    Then clearly $f_{i_1}(x) = f_{i_2}(x)$ since $i_1 = i_2$ and $f_{i_1} = f_{i_2}$ is a function.
    On the other hand, if $i_1 \neq i_2$, then we have that $\dom(f_{i_1}) \cap \dom(f_{i_2}) = A_{i_1} \cap A_{i_2} = \es$ since $\angles{A_i \where i \in I}$ are mutually disjoint and $i_1 \neq i_2$.
    Hence it is vacuously true that $f_{i_1}(x) = f_{i_2}(x)$ for all $x \in \dom(f_{i_1}) \cap \dom(f_{i_2})$ since there is no such $x$.
    Since $i_1$ and $i_2$ were arbitrary, this shows that $\braces{f_i}_{i \in I}$ is a compatible system (see Definition 2.3.10).
    It then follows from Theorem~2.3.12 that $f = \un f_i$ is a function with domain $\un \dom(f_i) = \un A_i$.

    Though perhaps it may seem obvious, we show formally that $f(x) = f_i(x)$ for any $x \in A_i$ (for any $i \in I$).
    So consider any such $i \in I$ and $x \in A_i$.
    Then $(x, f(x)) \in f = \un f_i$ so that there is a $j \in I$ where $(x, f(x)) \in f_j$.
    Suppose for a moment that $j \neq i$ so that $x \in \dom(f_j) = A_j$.
    Since $x \in A_i$ and $x \in A_j$ but $i \neq j$, this contradicts the fact that $\angles{A_i \where i \in I}$ are mutually disjoint.
    Hence it must be that $j = i$ so that $(x, f(x)) \in f_j = f_i$.
    From this of course it follows that $f_i(x) = f(x)$ as desired.
    
    We also claim that $f(x) \in \un B_i$ for any $x \in \un A_i$ so that $\un B_i$ can be the codomain of $f$.
    This is easy to show: consider any $x \in \un A_i$ so that there is an $i \in I$ where $x \in A_i$.
    It then follows that $f(x) = f_i(x) \in B_i$ since $f_i$ is a function from $A_i$ to $B_i$.
    Therefore we clearly have $f(x) \in \un B_i$.
    This shows the result since $x$ was arbitrary.

    We also claim that $f$ is injective.
    So consider any $x_1$ and $x_2$ in $\un A_i$ where $x_1 \neq x_2$.
    Then there are $i_1$ and $i_2$ such that $x_1 \in A_{i_1}$ and $x_2 \in A_{i_2}$.
    If $i_1 = i_2$ then $f(x_1) = f_{i_1}(x_1) \neq f_{i_1}(x_2) = f_{i_2}(x_2) = f(x_2)$ since $f_{i_1} = f_{i_2}$ is injective.
    If $i_1 \neq i_2$ then $f(x_1) = f_{i_1}(x_1) \in B_{i_1}$ whereas $f(x_2) = f_{i_2}(x_2) \in B_{i_2}$.
    Since $\braces{B_i \where i \in I}$ are mutually disjoint and $i_1 \neq i_2$ it follows that $f(x_1) \neq f(x_2)$.
    Therefore $f$ is an injective function from $\un A_i$ to $\un B_i$ so that
    \gath{
      \sum_{i \in I} \k_i = \abs{\un A_i} \leq \abs{\un B_i} = \sum_{i \in I} \l_i
    }
    as desired.
  }
}

\def\sumn{\sum_{n=0}^\infty}
\exercise{3}{
  Find some cardinals $\k_n$, $\l_n$ ($n \in \nats$) such that $k_n < \l_n$ for all $n$, but $\sumn \k_n = \sumn \l_n$.
}
\sol{
  Let $\k_n = 1$ and $\l_n = 2$ for all $n \in \nats$.
  We claim that these satisfy the required properties.
  \qproof{
    Clearly we have $\k_n = 1 < 2 = \l_n$ for all $n \in \nats$.
    It then follows from Exercise~9.1.4 (and also the more general Theorem~9.1.3) that
    \gath{
      \sumn \k_n = \sumn 1 = \sum_{n \in \nats} 1 = \sum_{n < \al_0} 1 = \al_0 \cdot 1 = \al_0
      = \al_0 \cdot 2 = \sum_{n < \al_0} 2 = \sum_{n \in \nats} 2 = \sumn 2 = \sumn \l_n
    }
    as desired.
  }
}

\exercise{4}{
  Prove that $\k + \k + \cdots \text{($\l$ times)} = \l \cdot \k$.
}
\sol{
  \def\una{\bigcup_{\a < \l}}
  \qproof{
    First, if $\l = 0$ then, by convention, we have
    \gath{
      \k + \k + \cdots \text{($\l$ times)} = \k + \k + \cdots \text{(0 times)} = 0 = 0 \cdot \k = \l \cdot \k
    }
    regardless of what $\k$ is.
    So assume in what follows that $\l > 0$ so $\l \geq 1$.
    
    For each $\a < \l$, define $A_\a = \braces{(\a, \b) \where \b \in \k}$.
    Consider then any $\a_1 < \l$ and $\a_2 < \l$ where $\a_1 \neq \a_2$.
    Suppose that both $(x,y) \in A_{\a_1}$ and $(x,y) \in A_{\a_2}$.
    It then follows that $x = \a_1$ and $x = \a_2$ so that $x = \a_1 = \a_2$, which contradicts our assumption that $\a_1 \neq \a_2$!
    So it must be that no such $(x,y)$ exists so that $A_{\a_1}$ and $A_{\a_2}$ are disjoint.
    Since $\a_1$ and $\a_2$ were arbitrary, this shows that $\angles{A_\a \where \a < \l}$ are mutually disjoint sets.
    We also clearly have that $\abs{A_\a} = \k$ for each $\a < \l$.
    It therefore follows from the definition of cardinal summation that
    \gath{
      \k + \k + \cdots \text{($\l$ times)} = \sum_{\a < \l} \k = \abs{\una A_\a} \,.
    }

    Now we show that $\una A_\a = \l \times \k$.
    First consider any $(x,y) \in \una$ so that there is an $\a < \l$ where $(x,y) \in A_\a$.
    We then have that $x = \a$ and $y \in \k$.
    Therefore $x = \a < \l$ so that $x \in \l$ by the definition of $<$ for ordinal numbers.
    Hence $x \in \l$ and $y \in \k$ so that $(x,y) \in \l \times \k$, which shows that $\una A_\a \ss \l \times \k$ since $(x,y)$ was arbitrary.
    Now consider any $(x,y) \in \l \times \k$ so that $x \in \l$ and $y \in \k$.
    Let $\a = x \in \l$ so that $\a < \l$.
    Hence $(x,y) = (\a, y)$ for $\a < \l$ and $y \in \k$, which shows that $(x,y) \in A_\a$ so that clearly $(x,y) \in \una A_\a$.
    This shows that $\l \times \k \ss \una A_\a$ since again $(x,y)$ was arbitrary.
    Thus $\una A_\a = \l \times \k$ as desired.

    Putting all this together, we have
    \gath{
      \k + \k + \cdots \text{($\l$ times)} = \sum_{\a < \l} \k = \abs{\una A_\a} = \abs{\l \times \k} 
      = \l \cdot \k
    }
    by the definition of cardinal multiplication since obviously $\abs{\l} = \l$ and $\abs{\k} = \k$.
  }
}

\def\sumi{\sum_{i \in I}}
\def\uni{\bigcup_{i \in I}}
\exercise{5}{
  Prove the \emph{distributive} law:
  \gath{
    \l \cdot \parens{\sumi \k_i} = \sumi \parens{\l \cdot \k_i} \,.
  }
}
\sol{
  \qproof{
    Suppose that $\angles{A_i \where i \in I}$ are mutually disjoint sets such that $\abs{A_i} = \k_i$ for all $i \in I$.
    Then by definition $\sumi \k_i = \abs{\uni A_i}$.
    Also suppose that $B$ is a set such that $\abs{B} = \l$.

    We claim first that $B \times \uni A_i = \uni \parens{B \times A_i}$.
    This is easy to show since, for any $x$ and $y$, we have
    \ali{
      (x, y) \in B \times \uni A_i &\bic x \in B \land \exists i \in I (y \in A_i) \\
      &\bic \exists i \in I (x \in B \land y \in A_i) \\
      &\bic \exists i \in I ((x,y) \in B \times A_i) \\
      &\bic (x,y) \in \uni (B \times A_i) \,.
    }
    We then have
    \ali{
      \l \cdot \sumi \k_i &= \l \cdot \abs{\uni A_i} & \text{(by the definition of cardinal summation)} \\
      &= \abs{B \times \uni A_i} & \text{(by the definition of cardinal multiplication)} \\
      &= \abs{\uni \parens{B \times A_i}} & \text{(by what was just shown above)} \\
      &= \sumi \abs{B \times A_i} & \text{(by the definition of cardinal summation)} \\
      &= \sumi \parens{\l \cdot \k_i} & \text{(by the definition of cardinal multiplication)}
    }
    as desired.
    We note that $\abs{\uni (B \times A_i)} = \sumi \abs{B \times A_i}$ works since the sets $\angles{B \times A_i \where i \in I}$ are mutually disjoint.
    This is easy to see by considering $i$ and $j$ in $I$ where $i \neq j$.
    Then, if $(x,y) \in B \times A_i$ and also $(x,y) \in B \times A_j$, it follows that $y \in A_i$ and $y \in A_j$, which cannot be since $i \neq j$ and $\angles{A_i \where i \in I}$ are mutually disjoint.
    Hence it must be that there is no such ordered pair $(x,y)$ so that $B \times A_i$ and $B \times A_j$ are disjoint, which proves the result since $i$ and $j$ were arbitrary.
  }
}

\def\sumi{\sum_{i \in I}}
\def\uni{\bigcup_{i \in I}}
\exercise{6}{
  $\abs{\uni A_i} \leq \sumi \abs{A_i}$.
}
\sol{
  \qproof{
    First let $\angles{B_i \where i \in I}$ be mutually disjoint sets where $\abs{B_i} = \abs{A_i}$ for every $i \in I$.
    It then follows that $\sumi \abs{A_i} = \abs{\uni B_i}$ by definition.
    For each $i \in I$ we can choose a bijection $f_i$ from $A_i$ to $B_i$ by the Axiom of Choice since $\abs{A_i} = \abs{B_i}$.
    We construct a function $f : \uni A_i \to \uni B_i$.
    For each $x \in \uni A_i$ we have that $x \in A_j$ for some $j \in I$.
    We choose one such $j$, which requires the Axiom of Choice, and set $f(x) = f_j(x)$.
    Clearly $f(x) = f_j(x) \in B_j$ so that then $f(x) \in \uni B_i$, which shows that $\uni B_i$ can be the codomain for $f$.

    We show that $f$ is injective.
    So consider any $x_1$ and $x_2$ in $\uni A_i$ where $x_1 \neq x_2$.
    Then we have chosen unique $j_1$ and $j_2$ where $f(x_1) = f_{j_1}(x)$ and $f(x_2) = f_{j_2}(x)$.
    If $j_1 = j_2$ then we have $f(x_1) = f_{j_1}(x_1) \neq f_{j_1}(x_2) = f_{j_2}(x_2) = f(x_2)$ since $f_{j_1} = f_{j_2}$ is injective and $x_1 \neq x_2$.
    If $j_1 \neq j_2$ then $f(x_1) = f_{j_1}(x_1) \in B_{j_1}$ whereas $f(x_2) = f_{j_2}(x_2) \in B_{j_2}$.
    Since $j_1 \neq j_2$ and $\angles{B_i \where i \in I}$ are mutually disjoint, it follows that $f(x_1) \neq f(x_2)$.
    Since this is true in both cases and $x_1$ and $x_2$ were arbitrary, it shows that $f$ is injective.

    Since $f$ is injective we, we have
    \gath{
      \abs{\uni A_i} \leq \abs{\uni B_i} = \sumi \abs{A_i}
    }
    as desired.
  }
}

\def\uni{\bigcup_{i \in I}}
\def\prodi{\prod_{i \in I}}
\def\prodj{\prod_{j \in J}}
\exercise{7}{
  If $J_i$ ($i \in I)$ are mutually disjoint sets and $J = \uni J_i$, and if $\k_j$ ($j \in J$) are cardinals, then
  \gath{
    \prodi \parens{\prod_{j \in J_i} \k_j} = \prod_{j \in J} \k_j
  }
  (\emph{associativity} of $\prod$).
}
\sol{
  \qproof{
    Suppose that $\angles{A_j \where j \in J}$ are sets where $\abs{A_j} = \k_j$ for each $j \in J$.
    It then follows that
    \begin{gather}
      \prodj \k_j = \abs{\prodj A_j} \,. \label{eqn:infsp:prodJ}
    \end{gather}
    Now, for any $i \in I$, set $B_i = \prod_{j \in J_i} A_j$ so that
    \begin{gather}
      \prod_{j \in J_i} \k_j = \abs{\prod_{j \in J_i} A_j} = \abs{B_i} \label{eqn:infsp:prodJi}
    \end{gather}
    by definition.

    Now we construct a bijection $f$ from $\prodj A_j$ to $\prodi B_i$.
    So consider any $a \in \prodj A_j$ so that $a = \angles{a_j \where j \in J}$ where $a_j \in A_j$ for every $j \in J$.
    Now, for each $i \in I$, set $a_i' = \angles{a_j \where j \in J_i}$, noting that clearly $j \in J = \bigcup_{i \in I} J_i$ for each $j \in J_i$ so that $a_j$ has been defined as in the range of $a$.
    We then have that $a_j \in A_j$ for $j \in J_i$ so that $a_i' \in \prod_{j \in J_i} A_j = B_i$.
    So set $b = \angles{a_i' \where i \in I}$ so that clearly $b \in \prodi B_i$, and set $f(a) = b$.
    Since $f(a) = b \in \prodi B_i$ for any $a \in \prodj A_j$, we have that $f$ is indeed a function from $\prodj A_j$ into $\prodi B_i$.

    We claim first that $f$ is injective.
    So consider any $\a$ and $\b$ in $\prodj A_j$ where $\a \neq \b$.
    It then follows that $\a = \angles{\a_j \where j \in J}$ and $\b = \angles{\b_j \where j \in J}$ where both $\a_j$ and $\b_j$ are in $A_j$ for any $j \in J$.
    Now, since $\a \neq \b$ it follows that there is a $j_0 \in J$ such that $\a_{j_0} \neq \b_{j_0}$.
    Also there is an $i_0 \in I$ such that $j_0 \in J_{i_0}$ since $j_0 \in J = \bigcup_{i \in I} J_i$.
    Now let $\a_i' = \angles{\a_j \where j \in J_i}$ and $\b_i' = \angles{\b_j \where j \in J_i}$ for $i \in I$.
    We then have that $\a_{i_0}' \neq \b_{i_0}'$ since $j_0 \in J_{i_0}$ and $\a_{j_0} \neq \b_{j_0}$.
    Clearly then $f(\a) = \angles{\a_i' \where i \in I}$ and $f(\b) = \angles{\b_i' \where i \in I}$ by definition so that $f(\a) \neq f(\b)$ since $i_0 \in I$ and $\a_{i_0}' \neq \b_{i_0}'$.
    This shows that $f$ is injective since $\a$ and $\b$ were arbitrary.

    We also claim that $f$ is onto.
    Consider any $b \in \prodi B_i$ so that $b = \angles{b_i \where i \in I}$ where each $b_i \in B_i$ for $i \in I$.
    So, for any $i \in I$, we have that $b_i \in B_i = \prod_{j \in J_i} A_j$ so that $b_i = \angles{a_{ij} \where j \in J_i}$ where each $a_{ij} \in A_j$ for $j \in J_i$.
    Now we construct a function $g$.
    So consider any $j_0 \in J$ so that there is a unique $i_0 \in I$ such that $j_0 \in J_{i_0}$, where the uniqueness clearly follows from the fact that $\angles{J_i \where i \in I}$ are mutually disjoint.
    Then simply set $g(j_0) = a_{i_0 j_0} \in A_{j_0}$ so that clearly $g \in \prodj A_j$.
    If we then set $a_i' = \angles{g(j) \where j \in J_i} = \angles{a_{ij} \where j \in J_i} = b_i$ for all $i \in I$, then $f(g) = \angles{a_i' \where i \in I} = \angles{b_i \where i \in I} = b$.
    Since $b$ was arbitrary this shows that $f$ is indeed onto.

    It may not have been obvious, but the uniqueness of $i_0 \in I$ for any $j_0 \in J$ (such that $j_0 \in J_{i_0}$) when constructing $g$ was critical for this proof.
    To see why, suppose that for some $j_0 \in J$ there are distinct $i_1$ and $i_2$ in $I$ such that $j_0 \in J_{i_1}$ and $j_0 \in J_{i_2}$.
    Then it could very well be that $a_{i_1 j_0} \neq a_{i_2 j_0}$ (though they would both be in $A_{j_0}$) and we would have to choose one to be $g(j_0)$.
    Supposing we choose $g(j_0) = a_{i_1 j_0}$ then we would have $a_{i_2}' = \angles{g(j) \where j \in J_{i_2}}$ so that $a_{i_2}' \neq b_{i_2}$ since $a_{i_2}'(j_0) = g(j_0) = a_{i_1 j_0} \neq a_{i_2 j_0} = b_{i_2}(j_0)$.
    If we had set $g(j_0) = a_{i_2 j_0}$ instead then, by the same argument, we would have $a_{i_1}' \neq b_{i_1}$.
    Clearly in either case this would break the proof since we would have $f(g) = \angles{a_i' \where i \in I} \neq \angles{b_i \where i \in I} = b$.
    In fact, in this case there would be no $g \in \prodj A_j$ such that $f(g) = b$ since we cannot choose a value for $g(j_0)$ for which $a_i' = b_i$ for all $i \in I$.
    
    Returning from our digression, we have shown that $f$ is a bijection from $\prodj A_j$ to $\prodi B_i$ so that
    \begin{gather}
      \abs{\prodj A_j} = \abs{\prodi B_i} \,. \label{eqn:infsp:pJpI}
    \end{gather}
    Therefore we have
    \ali{
      \prodi \parens{\prod_{j \in J_i} \k_j} &= \prodi \abs{B_i} & \text{(by \eqref{eqn:infsp:prodJi})} \\
      &= \abs{\prodi B_i} & \text{(by the definition of cardinal product)} \\
      &= \abs{\prodj A_j} & \text{(by \eqref{eqn:infsp:pJpI})} \\
      &= \prodj \k_j & \text{(by \eqref{eqn:infsp:prodJ})}
    }
    as desired.
  }
}

\def\prodi{\prod_{i \in I}}
\exercise{8}{
  If $\k_i \leq \l_i$ for all $i \in I$, then
  \gath{
    \prodi \k_i \leq \prodi \l_i \,.
  }
}
\sol{
  \qproof{
    Suppose sets $\angles{A_i \where i \in I}$ and $\angles{B_i \where i \in I}$ where $\abs{A_i} = \k_i$ and $\abs{B_i} = \l_i$ for all $i \in I$.
    Then, by the definition of the cardinal product, we have that
    \ali{
      \prodi \k_i &= \abs{\prodi A_i} &
      \prodi \l_i &= \abs{\prodi B_i} \,.
    }
    Now, for any $i \in I$, we also have
    \gath{
      \abs{A_i} = \k_i \leq \l_i = \abs{B_i}
    }
    so that we can choose an injective $f_i : A_i \to B_i$ (with the help of the Axiom of Choice).

    We then construct a function $f$ from $\prodi A_i$ to $\prodi B_i$.
    So for any $a \in \prodi A_i$ we have that $a = \angles{a_i \where i \in I}$ where each $a_i \in A_i$.
    Thus $a_i \in A_i = \dom(f_i)$ for each $i \in I$ so that $f_i(a_i) \in B_i$.
    We then define $f(a) = \angles{f_i(a_i) \where i \in I}$ so that clearly $f(a) \in \prodi B_i$ and hence $f$ is a function into $\prodi B_i$.

    We claim that $f$ as defined above is injective.
    To this end, consider any $\a$ and $\b$ in $\prodi A_i$ where $\a \neq \b$.
    It then follows that $\a = \angles{\a_i \where i \in I}$ and $\b = \angles{\b_i \where i \in I}$ where both $\a_i$ and $\b_i$ are elements of $A_i$ for each $i \in I$.
    Since $\a \neq \b$ we must have that there is an $i_0 \in I$ such that $\a_{i_0} \neq \b_{i_0}$.
    It then follows that $f_{i_0}(\a_{i_0}) \neq f_{i_0}(\b_{i_0})$ since $f_{i_0}$ is injective.
    Therefore clearly $f(\a) = \angles{f_i(\a_i) \where i \in I} \neq \angles{f_i(\b_i) \where i \in I} = f(\b)$.
    This shows that $f$ is injective since $\a$ and $\b$ were arbitrary.

    Finally, since $f$ is an injective function from $\prodi A_i$ to $\prodi B_i$, we have that
    \gath{
      \prodi \k_i = \abs{\prodi A_i} \leq \abs{\prodi B_i} = \prodi \l_i
    }
    as desired.
  }
}

\def\prodn{\prod_{n=0}^\infty}
\exercise{9}{
  Find some cardinals $\k_n$, $\l_n$ ($n \in \nats$) such that $\k_n < \l_n$ for all $n$, but $\prodn \k_n = \prodn \l_n$.
}
\sol{
  Let $\k_n = 2$ and $\l_n = \al_0$ for all $n \in \nats$.
  We claims that these satisfy the requirements.
  \qproof{
    Clearly $\k_n = 2 < \al_0 = \l_n$ for every $n \in \nats$.
    However, by Exercise~9.1.10 and Theorem~5.2.2c, we then have
    \gath{
      \prodn \k_n = \prodn 2 = \prod_{n < \al_0} 2 = 2^{\al_0}
      = \al_0^{\al_0} = \prod_{n < \al_0} \al_0 = \prodn \al_0 = \prodn \l_n
    }
  }
}

\def\proda{\prod_{\a < \l}}
\exercise{10}{
  Prove that $\k \cdot \k \cdot \cdots \text{($\l$ times)} = \k^\l$.
}
\sol{
  \qproof{
    First, we clearly have that
    \gath{
      \k \cdot \k \cdot \cdots \text{($\l$ times)} = \proda \k \,.
    }
    So let $A$ and $B$ be sets such that $\abs{A} = \k$ and $\abs{B} = \l$.
    Then by definition we have $\proda \k = \abs{\proda A}$ and $\k^\l = \abs{A^B}$.
    We also have that there is a bijective $g: \l \to B$ since $\abs{\l} = \l = \abs{B}$.
    We construct a bijection $f$ from $A^B$ to $\proda A$.
    So, for any $h \in A^B$ let $f(h) = h \circ g$.
    Clearly, since $g: \l \to B$ and $h: B \to A$, it follows that $f(h) : \l \to A$ and hence clearly $f(h) \in \proda A$ (since $f(h)(\a) \in A$ for each $\a < \l$) so that $f$ is a function from $A^B$ to $\proda A$.

    We also claim that $f$ is injective.
    So consider any $h_1$ and $ h_2$ in $A^B$ where $h_1 \neq h_2$.
    It then follows that  there must be a $b \in B$ where $h_1(b) \neq h_2(b)$.
    Then let $\a = \inv{g}(b)$, noting that $\inv{g}$ is a bijective function from $B$ to $\l$ since $g$ is bijective.
    We then have that
    \ali{
      f(h_1)(\a) &= (h_1 \circ g)(\a) = h_1(g(\a)) = h_1(g(\inv{g}(b))) = h_1(b) \\
      &\neq h_2(b) = h_2(g(\inv{g}(b))) = h_2(g(\a)) = (h_2 \circ g)(\a) = f(h_2)(\a)
    }
    so that $f(h_1) \neq f(h_2)$.
    Since $h_1$ and $h_2$ were arbitrary this shows that $f$ is indeed injective.

    We also claim that $f$ is onto.
    So consider any $h' \in \proda A$ so that $h'$ is a function from $\l$ to $A$ (since $h'(\a) \in A$ for each $\a < \l$).
    Then let $h = h' \circ \inv{g}$ so that clearly $h$ is a function from $B$ to $A$ since $\inv{g} : B \to \l$ and $h' : \l \to A$.
    We then have that
    \gath{
      f(h) = h \circ g = (h' \circ \inv{g}) \circ g = h' \circ (\inv{g} \circ g) = h' \circ i_\l = h'
    }
    where $i_\l$ is the identity function from $\l$ to $\l$.
    Since $h'$ was arbitrary, this shows that $f$ is indeed onto.

    Thus, since we have shown that $f$ is a bijection, it follows that
    \gath{
      \k \cdot \k \cdot \cdots \text{($\l$ times)} = \proda \k = \abs{\proda A} = \abs{A^B} = \k^\l
    }
    as desired.
  }
}
