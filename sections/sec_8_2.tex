\subsection{The Axiom of Choice in Mathematics}

\exercise{1}{
  Without using the Axiom of Choice, prove that the two definitions of closure points are equivalent if $A$ is an open set.
    [Hint: $X_n$ is open, so $X_n \cap \rats \neq \es$, and $\rats$ can be well-ordered.]
}
\sol{
  \begin{lem}\label{lem:aoc:intopen}
    If $A$ and $B$ are open subsets of $\reals$ then $A \cap B$ is open.
  \end{lem}
  \qproof{
    First note that, if $A \cap B = \es$, then this is vacuously open.
    Otherwise, consider any $x \in A \cap B$.
    Then, since $A$ is open and $x \in A$, there is a real $\d_1 > 0$ such that $\abs{y - x} < \d_1$ implies that $y \in A$ for all $y \in \reals$.
    Similarly, $x \in B$ and $B$ is open so that there is a $\d_2 > 0$ where $\abs{y - x} < \d_2$ implies $y \in B$ for all $y \in \reals$
    Let $\d = \min\braces{\d_1, \d_2}$ so that $\d \leq \d_1$ and $\d \leq \d_2$.
    Then consider any $y \in \reals$ where $\abs{y - x} < \d$.
    Then we have $\abs{y - x} < \d \leq \d_1$ so that $y \in A$.
    Similarly, $\abs{y - x} < \d \leq \d_2$ so that $y \in B$ as well.
    Hence $y \in A \cap B$.
    Since $y$ was arbitrary and $\d > 0$, this shows that $A \cap B$ is open.
  }

  \begin{lem}\label{lem:aoc:openrat}
    If $A$ is a nonempty open subset of $\reals$ then $A \cap \rats \neq \es$.
  \end{lem}
  \qproof{
    Suppose that $A \ss \reals$ is nonempty and open.
    Then there is an $x \in A$ and, since $A$ is open, there is a real $\d > 0$ such that $(x-\d, x+\d) \ss A$.
    Now, since $\d > 0$, we clearly have
    \gath{
      -\d < 0 < \d \\
      x-\d < x+\d.
    }
    Since $\rats$ is dense in $\reals$ with respect to order, there is a $q \in \rats$ such that $x-\d < q < x+\d$.
    Then we have $q \in (x-\d, x+\d)$ so that also $q \in A$ since $(x-\d, x+\d) \ss A$.
    Thus $q \in \rats$ and $q \in A$ so that $q \in A \cap \rats$ as desired.
  }

  \mainprob
  \qproof{
    A proof of this equivalence is presented in the text as Theorem~8.2.1.
    Recall that, in the proof that (b) implies (a), $a \in \reals$ is the closure point of $A \ss \reals$ and $X_n = \braces{x \in A \where \abs{x-a} < 1/n}$, and we know from (b) that each $X_n$ is nonempty.
    Note that we assume that $X_0 = A \cap (-\infty, \infty) = A \cap \reals = A$.
    Per the remarks after this proof, it suffices to show that the system of nonempty sets $\braces{X_n}_{n \in \nats}$ has a choice function when $A$ is open.

    First we show that each $X_n$ is an open set.
    Clearly $X_0 = A$ is open, so consider any natural $n > 0$ and let $I_n = (a-1/n, a+1/n)$.
    We claim that $X_n = A \cap I_n$.
    To this end we have
    \ali{
      x \in X_n &\bic x \in A \land \abs{x-a} < 1/n \\
      &\bic x \in A \land -1/n < x - a < 1/n \\
      &\bic x \in A \land a-1/n < x < a+1/n \\
      &\bic x \in A \land x \in I_n \\
      &\bic x \in A \cap I_n.
    }
    for any real $x$, which of course shows that $X_n = A \cap I_n$.
    Then, since $A$ is open and $I_n$ is clearly an open interval, it follows from Lemma~\ref{lem:aoc:intopen} that $X_n$ is open as well.

    Now, since each $X_n$ is open and nonempty, it follows that $X_n \cap \rats \neq \es$ from Lemma~\ref{lem:aoc:openrat}.
    Then, since $\rats$ is countable, it can clearly be well-ordered.
    So choose a well-ordering of $\rats$.
    Since $X_n \cap \rats$ is clearly a nonempty subset of $\rats$, it then has a least element $x_n$ according to our well-ordering.
    We then define a function $g$ on $\braces{X_n}_{n \in \nats}$ by $g(X_n) = x_n$, which is clearly a choice function.
  }
}

\exercise{2}{
  Prove that every continuous additive function $f$ is equal to $f_a$ for some $a \in \reals$.
}
\sol{
  \qproof{
    Consider any arbitrary continuous additive function $f : \reals \to \reals$.
    Then, by what was shown in the text, there is a real $a$ such that $f(q) = f_a(q) = a \cdot q$ for all $q \in \rats$; in particular $a = f(1)$.

    Now suppose to the contrary that $f \neq f_a$ so that there is an $x \in \reals$ where $f(x) \neq f_c(x)$.
    So let $\e = \abs{f(x) - f_a(x)}/2$, noting that clearly $\e > 0$ since $f(x) \neq f_a(x)$.
    Since $f$ is continuous there is a real $\d_1 > 0$ such that $\abs{f(y) - f(x)} < \e$ for all $y \in \reals$ where $\abs{y - x} < \d_1$.
    Also clearly $f_a$ is also continuous so that there is a real $\d_2 > 0$ where $\abs{f_a(y) - f_a(x)} < \e$ for all $y \in \reals$ where $\abs{y - x} < \d_2$.
    So let $\d = \min\braces{\d_1, \d_2}$.
    Then, since $\d > 0$ it follows that $x - \d < x + \d$ so that there is a $q \in \rats$ where $x - \d < q < x + \d$ since $\rats$ is order dense in $\reals$.
    It then clearly follows that $\abs{q - x} < \d$ so that $\abs{q - x} < \d \leq \d_1$ and $\abs{q - x} < \d \leq \d_2$.
    Therefore $\abs{f(q) - f(x)} < \e$ and $\abs{f_a(q) - f_a(x)} < \e$.

    We then have
    \ali{
      \abs{f(x) - f_a(x)} &\leq \abs{f(x) - f(q)} + \abs{f(q) - f_a(x)} \\
      &\leq \abs{f(x) - f(q)} + \abs{f(q) - f_a(q)} + \abs{f_a(q) - f_a(x)} \\
      &< \e + 0 + \e = 2\e = \abs{f(x) - f_a(x)},
    }
    which is a contradiction, noting that $\abs{f(q) - f_a(q)} = 0$ since $f(q) = f_a(q)$ since $q \in \rats$.
    So it must be that in fact $f = f_a$ as desired.
  }

  This proof is similar to that of Theorem~10.3.11 later in the text.
  That theorem is certainly more general, and this can be easily proved from it.
  In particular it was shown in the text that, for an arbitrary additive and continuous $f: \reals \to \reals$,  $f(q) = f_a(q)$ for all $q \in \rats$ for some $a \in \reals$ so that $f \rest \rats = f_a \rest \rats$.
  Since $f$ and $f_a$ are both continuous and $\rats$ is order dense in $\reals$, it follows from Theorem~10.3.11 that $f = f_a$.
}

\exercise{3}{
  Assume that $\mu$ has properties 0)-ii).
  Prove properties iv) and v).
  Also prove:
  \begin{enumerate}
    \item[\ilab{vi}] $\mu(A \cup B) = \mu(A) + \mu(B) - \mu(A \cap B)$.
    \item[\ilab{vii}] $\mu\parens{\bigcup_{n=0}^\infty A_n} \leq \sum_{n=0}^\infty \mu(A_n)$.
  \end{enumerate}
}
\sol{
  First, for reference, we assume the following properties of $\mu$:
  \begin{enumerate}
    \item[\ilab{0}] $\mu([a,b]) = b - a$ for any $a$ and $b$ in $\reals$ where $a < b$.
    \item[\ilab{i}] $\mu(\es) = 0$ and $\mu(\reals) = \infty$.
    \item[\ilab{ii}] If $\braces{A_n}_{n=0}^\infty$ is a collection of mutually disjoint subsets of $\reals$, then
      \gath{
        \mu\parens{\bigcup_{n=0}^\infty A_n} = \sum_{n=0}^\infty \mu(A_n).
      }
  \end{enumerate}
  First we show
  \begin{enumerate}
    \item[\ilab{iv}] If $A \cap B = \es$ then $\mu(A \cup B) = \mu(A) + \mu(B)$.
  \end{enumerate}
  \qproof{
    Assume that $A \cap B = \es$ and define $A_0 = A$, $A_1 = B$, and $A_n = \es$ for all natural $n > 1$.
    Then clearly each of the sets in $\braces{A_n}_{n=0}^\infty$ are mutually disjoint.
    It is also trivial to show that $\bigcup_{n=0}^\infty A_n = A \cup B$.
    We then have by property ii) that
    \ali{
      \mu(A \cup B) &= \mu\parens{\bigcup_{n=0}^\infty A_n} = \sum_{n=0}^\infty \mu(A_n) = \mu(A_0) + \mu(A_1) + \sum_{n=2}^\infty \mu(A_n) \\
      &= \mu(A) + \mu(B) + \sum_{n=2}^\infty \mu(\es) = \mu(A) + \mu(B) + \sum_{n=2}^\infty 0 \\
      &= \mu(A) + \mu(B),
    }
    noting that we have also used property i) according to which $\mu(\es) = 0$.
    This shows the desired result.
  }

  \begin{lem}\label{lem:aoc:measmin}
    For a measure $\mu$, if $A \ss B \ss \reals$ then $\mu(B-A) = \mu(B) - \mu(A)$.
  \end{lem}
  \qproof{
    Clearly $A$ and $B-A$ are disjoint sets such that $A \cup (B-A) = B$ so that $\mu(A) + \mu(B-A) = \mu(B)$ by property iv).
    The result then clearly follows by subtracting $\mu(A)$ from both sides.
  }

  Next we show
  \begin{enumerate}
    \item[\ilab{v}] If $A \ss B$ then $\mu(A) \leq \mu(B)$.
  \end{enumerate}
  \qproof{
    Suppose that $A \ss B$ so that $\mu(A) + \mu(B-A) = \mu(B)$ by Lemma~\ref{lem:aoc:measmin}.
    Since $\mu$ is a function into $\clop{0, \infty} \cup \braces{\infty}$, it follows that
    \gath{
      0 \leq \mu(B-A) \\
      \mu(A) \leq \mu(B-A) + \mu(A) = \mu(A) + \mu(B-A) \\
      \mu(A) \leq \mu(B)
    }
    as desired.
  }

  Now we show
  \begin{enumerate}
    \item[\ilab{vi}] $\mu(A \cup B) = \mu(A) + \mu(B) - \mu(A \cap B)$.
  \end{enumerate}
  \qproof{
    Let $C = A \cap B$, $A' = A - C$, and $B' = B - C$.
    It is then trivial to show that $A'$, $B'$, and $C$ are mutually disjoint sets such that $A' \cup B' \cup C = A \cup B$.
    We then have by a straightforward extension of property iv) that
    \gath{
      \mu(A') + \mu(B') + \mu(C) = \mu(A \cup B) \\
      \mu(A - C) + \mu(B - C) + \mu(C) = \mu(A \cup B) \\
      \mu(A) - \mu(C) + \mu(B) - \mu(C) + \mu(C) = \mu(A \cup B) \\
      \mu(A) + \mu(B) - \mu(C) = \mu(A \cup B) \\
      \mu(A) + \mu(B) - \mu(A \cap B) = \mu(A \cup B)
    }
    as desired.
    Note that we have also used Lemma~\ref{lem:aoc:measmin} since clearly $C \ss A$ and $C \ss B$ so that $\mu(A-C) = \mu(A) - \mu(C)$ and $\mu(B-C) = \mu(B) - \mu(C)$.
  }

  Lastly we show
  \begin{enumerate}
    \item[\ilab{vii}] $\mu\parens{\bigcup_{n=0}^\infty A_n} \leq \sum_{n=0}^\infty \mu(A_n)$.
  \end{enumerate}
  \qproof{
    Supposing that we have a system of sets $\braces{A_n}_{n=0}^\infty$, first we define a sequence of corresponding sets recursively:
    \ali{
      A_0' &= A_0 \\
      A_{n+1}' &= A_{n+1} - \bigcup_{k=0}^n A_k'
    }
    Note that it is clear that $A_n' \ss A_n$ for any $n \in \nats$ so that $\mu(A_n') \leq \mu(A_n)$ by property v).

    We now show that each of these sets are mutually disjoint.
    So consider any natural $m$ and $n$ where $m \neq n$.
    Without loss of generality, we can then assume that $m < n$.
    Suppose that $A_m'$ and $A_n'$ are \emph{not} disjoint so that there is an $x \in A_m' \cap A_n'$.
    Thus $x \in A_n' = A_n - \bigcup_{k=0}^{n-1} A_k'$ so that $x \notin \bigcup_{k=0}^{n-1} A_k'$.
    However since also $x \in A_m'$ and $0 \leq m \leq n-1$, we also can conclude that $x \in \bigcup_{k=0}^{n-1} A_k'$.
    Since this is a contradiction, it must be that $A_m'$ and $A_n'$ are in fact disjoint, which shows mutual disjointedness since $m$ and $n$ were arbitrary.

    Next we show that $\bigcup_{n=0}^\infty A_n' = \bigcup_{n=0}^\infty A_n$.
    The $\ss$ direction is clear since, for any $x \in \bigcup_{n=0}^\infty A_n'$, there a natural $n$ where $x \in A_n'$.
    Since $A_n' \ss A_n$ it follows that $x \in A_n$ so that clearly $x \in \bigcup_{n=0}^\infty A_n$.
    Now consider any $x \in \bigcup_{n=0}^\infty A_n$ so that there is a natural $n$ where $x \in A_n$.
    Clearly if $n=0$ then $x \in A_0 = A_0'$.
    So assume that $n > 0$ so that $\bigcup_{k=0}^{n-1} A_k'$ is defined.
    If $x \in \bigcup_{k=0}^{n-1} A_k'$ then there is a $0 \leq k \leq n-1$ where $x \in A_k'$.
    On the other hand, if $x \notin \bigcup_{k=0}^{n-1} A_k'$ then clearly $x \in A_n - \bigcup_{k=0}^{n-1} A_k' = A_n'$.
    Thus in all cases there is a natural $k$ such that $x \in A_k'$ so that $x \in \bigcup_{n=0}^\infty A_n'$ as desired.

    We therefore have
    \ali{
      \mu\parens{\bigcup_{n=0}^\infty A_n} &= \mu\parens{\bigcup_{n=0}^\infty A_n'} \\
      &= \sum_{n=0}^\infty \mu(A_n') & \text{(by property ii) since the sets $\braces{A_n'}_{n=0}^\infty$ are mutually disjoint)} \\
      &\leq \sum_{n=0}^\infty \mu(A_n) & \text{(since $0 \leq \mu(A_n') \leq \mu(A_n)$ for all natural $n$)}
    }
    as desired.
  }
}

\def\defsa{
  \begin{enumerate}
    \item[(a)] $\es \in \fs$ and $S \in \fs$.
    \item[(b)] If $X \in \fs$ then $S - X \in \fs$.
    \item[(c)] If $X_n \in \fs$ for all $n$, then $\bigcup_{n=0}^\infty X_n \in \fs$ and $\bigcap_{n=0}^\infty X_n \in \fs$.
  \end{enumerate}
}
\def\fs{\mathfrak{S}}
\exerciseapp{4}{*}{
  Let $\fs = \braces{X \ss S \where \abs{X} \leq \al_0 \text{ or } \abs{S-X} \leq \al_0}$.
  Prove that $\fs$ is a $\s$-algebra.
}
\sol{
  \begin{lem}\label{lem:aoc:setdm}
    If $A$, $B$, and $C$ are sets then $A - (B - C) = (A - B) \cup (A \cap C)$.
  \end{lem}
  \qproof{
    For any $x$ we have
    \ali{
      x \in A - (B - C) &\bic x \in A \land x \notin B - C \\
      &\bic x \in A \land \lnot (x \in B \land x \notin C) \\
      &\bic x \in A \land (x \notin B \lor x \in C) \\
      &\bic (x \in A \land x \notin B) \lor (x \in A \land x \in C) \\
      &\bic X \in A - B \lor x \in A \cap C \\
      &\bic x \in (A - B) \cup (A \cap C).
    }
  }

  \begin{lem}\label{lem:aoc:smss}
    If $A \ss B$ and $S$ is another set, then $S - B \ss S - A$.
  \end{lem}
  \qproof{
    Consider any $x \in S - B$ so that $x \in S$ and $x \notin B$.
    Then it has to be that also $x \notin A$ since otherwise it would not be that $A \ss B$.
    Hence $x \in S - A$, which shows the result since $x$ was arbitrary.
  }

  \mainprob
  \qproof{
    We must show that the above definition of $\fs$ satisfies the three parts of the definition of a $\s$-algebra:
    \defsa

    For (a) clearly $\abs{\es} = 0 \leq \al_0$ so that $\es \in \fs$, and $S - S = \es$ so that $\abs{S - S} = \abs{\es} = 0 \leq \al_0$.
    Hence $S \in \fs$ as well.

    For (b) suppose that $X \ss S$ and $X \in \fs$.
    Then either $\abs{X} \leq \al_0$ or $\abs{S - X} \leq \al_0$.
    If $\abs{X} \leq \al_0$ then by Lemma~\ref{lem:aoc:setdm} we have
    \gath{
      S - (S - X) = (S - S) \cup (S \cap X) = \es \cup (S \cap X) = S \cap X = X
    }
    since $X \ss S$.
    Therefore $\abs{S - (S - X)} = \abs{X} \leq \al_0$ so that $S - X \in \fs$.
    On the other hand, if $\abs{S - X} \leq \al_0$, then obviously $S - X \in \fs$ by definition.

    Lastly, regarding (c), suppose that $\braces{X_n}_{n=0}^\infty$ is a system of sets where each $X_n$ is in $\fs$.
    Thus $\abs{X_n} \leq \al_0$ or $\abs{S - X_n} \leq \al_0$ for each natural $n$.

    Now we show that $\bigcup_{n=0}^\infty X_n \in \fs$.
    First, if $\abs{X_n} \leq \al_0$ for all natural $n$, then it follows from Theorem~8.1.7 that $\abs{\bigcup_{n=0}^\infty X_n} \leq \al_0$, which of course uses the Axiom of Choice.
    If, on the other hand, there is a natural $m$ such that $\abs{X_m} \not\leq \al_0$, then it has to be that $\abs{S - X_m} \leq \al_0$ since $X_m \in \fs$.
    Then, since clearly $X_m \ss \bigcup_{n=0}^\infty X_n$, it follows from Lemma~\ref{lem:aoc:smss} that $S - \bigcup_{n=0}^\infty X_n \ss S - X_m$ so that clearly $\abs{S - \bigcup_{n=0}^\infty X_n} \leq \abs{S - X_m} \leq \al_0$.
    Thus in all cases we have that either $\abs{\bigcup_{n=0}^\infty X_n} \leq \al_0$ or $\abs{S - \bigcup_{n=0}^\infty X_n} \leq \al_0$ so that $\bigcup_{n=0}^\infty X_n \in \fs$.

    Lastly we show that $\bigcap_{n=0}^\infty X_n \in \fs$ as well.
    If it is the case $\abs{S - X_n} \leq \al_0$ for all natural $n$, then clearly we have that $\bigcup_{n=0}^\infty (S - X_n) \leq \al_0$, again by Theorem~8.1.7.
    It also follows from Exercise~2.3.11 that $S - \bigcap_{n=0}^\infty X_n = \bigcup_{n=0}^\infty (S - X_n)$ so that we have $\abs{S - \bigcap_{n=0}^\infty X_n} = \abs{\bigcup_{n=0}^\infty (S - X_n)} \leq \al_0$.
    Now, on the other hand, if there is a natural $m$ such that $\abs{S - X_m} \not\leq \al_0$, then it has to be that $\abs{X_m} \leq \al_0$ since $X_m \in \fs$.
    Since clearly $\bigcap_{n=0}^\infty X_n \ss X_m$, we then have $\abs{\bigcap_{n=0}^\infty X_n} \leq \abs{X_m} \leq \al_0$.
    Hence in either case we have that $\abs{\bigcap_{n=0}^\infty X_n} \leq \al_0$ or $\abs{S - \bigcap_{n=0}^\infty X_n} \leq \al_0$ so that $\bigcap_{n=0}^\infty X_n \in \fs$ by definition.

    We have therefore shown parts (a), (b), and (c) so that $\fs$ is a $\s$-algebra as desired.
  }
}

\def\fc{\mathfrak{C}}
\def\ft{\mathfrak{T}}
\exercise{5}{
  Let $\fc$ be any collection of subsets of $S$.
  Let $\fs = \bigcap \braces{\ft \where \fc \ss \ft \text{ and $\ft$ is a $\s$-algebra of subsets of $S$}}$.
  Prove that $\fs$ is a $\s$-algebra (it is called the \emph{$\s$-algebra generated} by $\fc$).
}
\sol{
  \qproof{
    First, let $T = \braces{\ft \where \fc \ss \ft \text{ and $\ft$ is a $\s$-algebra of subsets of $S$}}$ so that $\fs = \bigcap T$.
    Then we must show that $\fs$ meets the definition of a $\s$-algebra:
    \defsa

    Regarding (a), consider any $\ft \in T$.
    Since $\ft$ is then a $\s$-algebra it follows that both $\es \in \ft$ and $S \in \ft$ by (a).
    Then, since $\ft \in T$ was arbitrary, it follows that both $\es$ and $S$ are in $\bigcap T = \fs$.

    For (b) suppose that $X \in \fs = \bigcap T$ so that $X \in \ft$ for all $\ft \in T$.
    So consider any such $\ft \in T$ so that clearly $X \in \ft$.
    Then, since $\ft$ is then a $\s$-algebra, it follows that $S - X \in \ft$ by (b).
    Since $\ft \in T$ was arbitrary, we have that $S - X \in \bigcap T = \fs$.

    Lastly, for part (c) of the definition, suppose that $X_n \in \fs = \bigcap T$ for all $n \in \nats$.
    Let $\ft$ be any element of $T$ so that $X_n \in \ft$ for all natural $n$.
    Since $\ft$ is a $\s$-algebra, it then follows from (c) that both $\bigcup_{n=0}^\infty X_n$ and $\bigcap_{n=0}^\infty X_n$ are in $\ft$.
    Since $\ft \in T$ was arbitrary we have that $\bigcup_{n=0}^\infty X_n$ and $\bigcap_{n=0}^\infty X_n$ are in $\bigcap T = \fs$.

    Hence we have shown all three parts of the definition so that $\fs$ is indeed a $\s$-algebra.
  }
}

\def\defsam{
  \begin{enumerate}
    \item[\ilab{i}] $\mu(\es) = 0$, $\mu(S) > 0$.
    \item[\ilab{ii}] If $\braces{X_n}_{n=0}^\infty$ is a collection of mutually disjoint sets from $\fs$, then
      \gath{
        \mu\parens{\bigcup_{n=0}^\infty X_n} = \sum_{n=0}^\infty \mu(X_n).
      }
  \end{enumerate}
}
\exercise{6}{
  Fix $a \in S$ and define $\mu$ on $\pset{S}$ by: $\mu(A) = 1$ if $a \in A$, $\mu(A) = 0$ if $a \notin A$.
  Show that $\mu$ is a $\s$-additive measure on $S$.
}
\sol{
  \qproof{
    Let $\fs = \pset{S}$, which we know is the largest $\s$-algebra of subsets of $S$.
    We must show that $\mu$ as defined above satisfies the properties of $\s$-additive measure on $S$:
    \defsam

    To show \ilab{i}, we clearly have that $a \notin \es$ so that by definition $\mu(\es) = 0$.
    Also, clearly $a \in S$ so that $\mu(S) = 1 > 0$.

    Regarding \ilab{ii}, suppose that $\braces{X_n}_{n=0}^\infty$ is a collection of mutually disjoint sets in $\fs = \pset{S}$.

    Case: $a \in \bigcup_{n=0}^\infty X_n$.
    Then by definition $\mu\parens{\bigcup_{n=0}^\infty X_n} = 1$.
    There is also an $n \in \nats$ such that $a \in X_n$, and since the sets $\braces{X_k}_{k=0}^\infty$ are mutually disjoint, it follows that $a \notin X_m$ for any natural $m \neq n$ (since otherwise $X_n$ and $X_m$ would not be disjoint).
    Thus we have $\mu(X_n) = 1$ while $\mu(X_m) = 0$ for all natural $m \neq n$.
    Hence
    \gath{
      \sum_{k=0}^\infty \mu(X_k) = \sum_{k \in \nats} \mu(X_k) = \sum_{k=0}^{n-1} \mu(X_k) + \mu(X_n) + \sum_{k=n+1}^\infty \mu(X_k)
      = \sum_{k=0}^{n-1} 0 + 1 + \sum_{k=n+1}^\infty 0 = 1.
    }
    Thus clearly $\mu\parens{\bigcup_{n=0}^\infty X_n} = \sum_{n=0}^\infty \mu(X_n) = 1$.

    Case: $a \notin \bigcup_{n=0}^\infty X_n$.
    Then $\mu\parens{\bigcup_{n=0}^\infty X_n} = 0$ by definition.
    It also follows that $a \notin X_n$ for every natural $X_n$ so that $\mu(X_n) = 0$.
    Hence clearly
    \gath{
      \mu\parens{\bigcup_{n=0}^\infty X_n} = 0 = \sum_{n=0}^\infty 0 = \sum_{n=0}^\infty \mu(X_n).
    }

    Thus \ilab{ii} is shown in both cases so that $\mu$ is indeed a $\s$-additive measure on $S$ since we also showed \ilab{i}.
  }
}

\exercise{7}{
  For $A \ss S$ let $\mu(A) = 0$ if $A = \es$, $\mu(A) = \infty$ otherwise.
  Show that $\mu$ is a $\s$-additive measure on $S$.
}
\sol{
  \qproof{
    Let $\fs = \pset{S}$, which we know is the largest $\s$-algebra of subsets of $S$.
    We must show that $\mu$ as defined above satisfies the properties of $\s$-additive measure on $S$:
    \defsam

    For \ilab{i} we clearly have $\mu(\es) = 0$ by definition and $\mu(S) = \infty >0$ since $S$ is nonempty, which follows from the fact that $\pset{S}$ is a $\s$-algebra of subsets of $S$.

    Regarding \ilab{ii}, suppose that $\braces{X_n}_{n=0}^\infty$ is a collection of disjoint sets in $\fs = \pset{S}$.

    Case: $\bigcup_{n=0}^\infty X_n = \es$.
    Then by definition $\mu\parens{\bigcup_{n=0}^\infty X_n} = 0$, and it also has to be that $X_n = \es$ for all $n \in \nats$ so that $\mu(X_n) = 0$.
    Therefore
    \gath{
      \sum_{n=0}^\infty \mu(X_n) = \sum_{n=0}^\infty 0 = 0 = \mu\parens{\bigcup_{n=0}^\infty X_n}.
    }

    Case: $\bigcup_{n=0}^\infty X_n \neq \es$.
    Then by definition $\mu\parens{\bigcup_{n=0}^\infty X_n} = \infty$.
    It also follows that $X_n \neq \es$ for at least one $n \in \nats$ so that $\mu(X_n) = \infty$.
    We then have
    \ali{
      \sum_{k=0}^\infty \mu(X_k) &= \sum_{k=0}^{n-1} \mu(X_k) + \mu(X_n) + \sum_{k=n+1}^\infty \mu(X_k)
      = \sum_{k=0}^{n-1} \mu(X_k) + \infty + \sum_{k=n+1}^\infty \mu(X_k) \\
      &= \infty = \mu\parens{\bigcup_{n=0}^\infty X_n}
    }
    since each $\mu(X_m) \in \braces{0, \infty}$ for all natural $m \neq n$.

    Thus \ilab{ii} is shown in both cases so that $\mu$ is indeed a $\s$-additive measure on $S$ as desired.
  }
}

\def\unxn{\bigcup_{n=0}^\infty X_n}
\exercise{8}{
  For $A \ss S$ let $\mu(A) = \abs{A}$ if $A$ is finite, $\mu(A) = \infty$ if $A$ is infinite.
  $\mu$ is a $\s$-additive measure on $S$; it is called the \emph{counting measure} on $S$.
}
\sol{
  \qproof{
    Let $\fs = \pset{S}$, which we know is the largest $\s$-algebra of subsets of $S$.
    We must show that $\mu$ as defined above satisfies the properties of $\s$-additive measure on $S$:
    \defsam

    For \ilab{i} we have that $\mu(\es) = \abs{\es} = 0$ since $\es$ is finite.
    If $S$ is finite then $\mu(S) = \abs{S} > 0$ since $S$ is nonempty.
    If $S$ is infinite then $\mu(S) = \infty > 0$ as well so that in either case $\mu(S) > 0$ as desired.

    To show \ilab{ii} suppose that $\braces{X_n}_{n=0}^\infty$ is a collection of disjoint sets in $\fs = \pset{S}$.

    Case: There is an $m \in \nats$ where $X_m$ is infinite.
    Then obviously $\mu(X_m) = \infty$ by definition.
    We also clearly have that $\unxn$ is infinite since $X_m \ss \unxn$, and hence $\mu(\unxn) = \infty$.
    Then
    \gath{
      \sum_{n=0}^\infty \mu(X_n) = \sum_{n=0}^{m-1} \mu(X_n) + \mu(X_m) + \sum_{n=m+1}^\infty \mu(X_n)
      = \sum_{n=0}^{m-1} \mu(X_n) + \infty + \sum_{n=m+1}^\infty \mu(X_n) = \infty
    }
    since $\mu(X_n) \in \nats \cup \braces{\infty}$ for every $n \neq m$.
    Therefore $\mu(\unxn) = \sum_{n=0}^\infty \mu(X_n) = \infty$ as desired.

    Case: $X_n$ is finite for every $n \in \nats$.
    Clearly then $\mu(X_n) = \abs{X_n}$ for every natural $n$.
    First, if there is a natural $N$ such that $X_n = \es$ for all $n > N$, then clearly $\unxn = \bigcup_{n=0}^N X_n$, i.e. the union is finite.
    It then follows that $\bigcup_{n=0}^N X_n$ is finite by Theorem~4.2.7 so that $\mu\parens{\bigcup_{n=0}^N X_n} = \abs{\bigcup_{n=0}^N X_n}$.
    Then, since the sets $\braces{X_n}_{n=0}^N$ are mutually disjoint, we have
    \gath{
      \mu\parens{\unxn} = \mu\parens{\bigcup_{n=0}^N X_n} = \abs{\bigcup_{n=0}^N X_n}
      = \sum_{n=0}^N \abs{X_n} = \sum_{n=0}^N \mu(X_n) = \sum_{n=0}^\infty \mu(X_n)
    }
    by the definition of cardinal addition, noting that clearly the last step follows from the fact that $\mu(X_n) = \abs{\es} = 0$ for all $n > N$.

    On the other hand, if there is no such $N$, then it follows that, for every $N \in \nats$, there is a natural $n > N$ such that $X_n \neq \es$.
    At this point we need two facts from real analysis, supposing that $\angles{a_n}_{n=0}^\infty$ is a real sequence:
    \begin{enumerate}
      \item By definition, the sequence converges to a real $a$ if, for every real $\e > 0$, there is an $N \in \nats$ such that $\abs{a_n - a} < \e$ for every natural $n \geq N$.
      \item If the infinite series $\sum_{n=0}^\infty a_n$ converges (to a finite value) then the sequence itself must converge to zero.
    \end{enumerate}
    We shall show that $\sum_{n=0}^\infty \mu(X_n)$ diverges by showing that the sequence $\angles{\mu(X_n)}_{n=0}^\infty$ does \emph{not} converge to zero (i.e. the contrapositive of 2).
    So let $\e = 1/2$, noting that clearly $\e = 1/2 > 0$.
    Then consider any natural $N$ so that there is a natural $n > N$ such that $X_n \neq \es$.
    It then follows that, since $X_n$ is finite but nonempty, $\abs{\mu(X_n) - 0} = \abs{\mu(X_n)} = \abs{\abs{X_n}} = \abs{X_n} \geq 1 \geq 1/2 = \e$.
    Therefore we have shown
    \gath{
      \exists \e > 0 \forall N \in \nats \exists n \geq N \parens{\abs{\mu(X_n) - 0} \geq \e} \\
      \lnot \forall \e > 0 \exists N \in \nats \forall n \geq N \parens{\abs{\mu(X_n) - 0} < \e},
    }
    which shows by definition that $\angles{\mu(X_n)}_{n=0}^\infty$ does not converge to zero.
    Hence $\sum_{n=0}^\infty \mu(X_n)$ diverges so that by convention $\sum_{n=0}^\infty \mu(X_n) = \infty$.

    Lastly, we show that $\unxn$ must be infinite.
    Suppose to the contrary that $\unxn$ is finite.
    We then construct a function $f : \unxn \to \nats$ as follows: for each $x \in \unxn$ there is a unique $m \in \nats$ where $x \in X_m$.
    Clearly such an $m$ exists since $x \in \unxn$, and it is unique because the sets $\braces{X_n}_{n=0}^\infty$ are mutually disjoint (if it was not unique then there would be distinct $n$ and $m$ where $x \in X_n$ and $x \in X_m$ so that $X_n \cap X_m \neq \es$).
    We then simply set $f(x) = m$.

    It then follows from Theorem~4.2.5 that $\ran(f)$ is finite since $\dom(f) = \unxn$ is.
    Since $\ran(f)$ is then a finite set of natural numbers, it has greatest natural number $N$.
    But we know that there is an $m > N$ such that $X_m \neq \es$ so that there is an $x \in X_m$.
    It then follows that clearly $x \in \unxn$ and that $f(x) = m$.
    However, then $m$ would be in $\ran(f)$ so that $m \leq N$ since $N$ is the greatest element of $\ran(f)$.
    But we already know that $m > N$, which is a contradiction.
    So it has to be that $\unxn$ is in fact infinite as desired.

    We therefore have $\mu\parens{\unxn} = \infty = \sum_{n=0}^\infty \mu(X_n)$ and hence ii) has been shown in every case and sub-case so that $\mu$ is indeed a $\s$-additive measure on $S$ by definition.
  }
}
