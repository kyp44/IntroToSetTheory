% Common sets
\def\nats{{\boldsymbol{N}}}
\def\ints{{\boldsymbol{Z}}}
\def\rats{{\boldsymbol{Q}}}
\def\prats{\rats^+}
\def\reals{{\boldsymbol{R}}}
\def\es{\varnothing}

% Common variables/symbols
\def\vphi{\varphi}
\def\a{\alpha}
\def\b{\beta}
\def\g{\gamma}
\def\d{\delta}
\def\e{\varepsilon}
\def\z{\zeta}
\def\k{\kappa}
\def\l{\lambda}
\def\n{\nu}
\def\r{\rho}
\def\s{\sigma}
\def\t{\tau}
\def\x{\xi}
\def\w{\omega}
\def\W{\Omega}
\def\al{\aleph}

% Logic
\def\imp{\rightarrow}
\def\bic{\leftrightarrow}
\newcommand{\propP}[1]{\prop{P}(#1)}

% For set-buider notation
\def\where{\mid}

% Other stuff
\def\dom{\mathrm{dom}\,}
\def\ran{\mathrm{ran}\,}

% Cardinality shortcuts
\def\cnats{{\al_0}}
\def\ccont{{2^\cnats}}

% Equivalence classes
\newcommand\eclass[2]{\squares{#1}_{#2}}

% Shortcuts that make writing easier
\newcommand\parens[1]{\left( #1 \right)}
\newcommand\squares[1]{\left[ #1 \right]}
\newcommand\braces[1]{\left\{ #1 \right\}}
\newcommand\angles[1]{\left\langle #1 \right\rangle}
\newcommand\ceil[1]{\left\lceil #1 \right\rceil}
\newcommand\floor[1]{\left\lfloor #1 \right\rfloor}
\newcommand\abs[1]{\left| #1 \right|}
\newcommand\dabs[1]{\left\| #1 \right\|}
\newcommand\vect[1]{\mathrm{\mathbf{#1}}}
\newcommand\conj[1]{\overline{#1}}
\newcommand\pset[1]{\mathcal{P}\left(#1\right)}
\newcommand\inv[1]{#1^{-1}}
\newcommand\prop[1]{\mathbf{#1}}
\def\rest{\restriction}
\newcommand\tet[2]{{^{#1}#2}}

% Families of set
\def\famF{\mathcal{F}}

% These are needed so that half-open intervals do not cause auto-indentation issues due to unmatched brackets
\newcommand\clop[1]{[#1)}
\newcommand\ilab[1]{#1)}

% Other miscellaneous stuff
\def\ss{\subseteq}
\def\pss{\subset}
\def\Seq{\mathrm{Seq}}
\def\prece{\preccurlyeq}
\def\sd{\,\triangle\,}

% Environment shortcuts
\newcommand\gath[1]{\begin{gather*} #1 \end{gather*}}
\newcommand\ali[1]{\begin{align*} #1 \end{align*}}
\newcommand\qproof[1]{\begin{proof} #1 \end{proof}}

% Environment for indenting nested paragraphs (useful in case trees)
\newenvironment{indpar}
{
    \begin{adjustwidth}{1cm}{}
}{
    \end{adjustwidth}
}

% Exercise, Theorem, and solution shortcuts
\newcommand\exercise[2]{{
    \renewcommand\label[1]{} % Needed to suppress multiply-defined label warning since the same question numbers are used in different sections
    \setcounter{subsubsection}{#1-1} % Subsubsections are used to that lemmas have the full nested numbering
    \stepcounter{subsubsection} % Need to increment this so that the lemma counter gets reset
    \setcounter{question}{#1-1} % Manually set the question number since we always know the exercise number
    \question{#2} % The actual exam class question
}}
\newcommand\exerciseapp[3]{
    \setqf{#2}
    \exercise{#1}{#3}
    \setqf{}
}   
\newcommand\theorem[2]{{
    \renewcommand\label[1]{} % Needed to suppress multiply-defined label warning since the same question numbers are used in different sections
    \setcounter{subsubsection}{#1-1} % Subsubsections are used to that lemmas have the full nested numbering
    \stepcounter{subsubsection} % Need to increment this so that the lemma counter gets reset
    \setcounter{question}{#1-1} % Manually set the question number since we always know the theorem number
    \question{#2} % The actual exam class question
}}
\newcommand\theoremapp[3]{
    \setqf{#2}
    \theorem{#1}{#3}
    \setqf{}
}
\newcommand\sol[1]{
  \begin{solution}

    #1
  \end{solution}
}

% Main problem and theorem labels
\def\mainprob{\textbf{Main Problem.}}
\def\mainthrm{\textbf{Main Theorem.}}

% Saves some typing
\def\cbthrm{Cantor-Bernstein Theorem}

% Book title
\def\booktitle{
  Introduction to Set Theory \\
  Third Edition, Revised and Expanded \\
  by Karel Hrbacek and Thomas Jech
}
