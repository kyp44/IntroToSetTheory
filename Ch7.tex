\def\ex{7.1.1}
\setcounter{itm}{0}
\question{\ex}

\begin{solution}
    \begin{statement}{Lemma~\ex.\itm{lem:infwo:succ}}
        If $\a$ is an infinite ordinal then $\abs{\a} = \abs{\a+1}$, i.e. $\a$ and $\a+1$ are equipotent.
    \end{statement}

    \proof{
      First we note that since $\a$ is inifinite we have $\a + 1 > \a \geq \w$.
      We then construct a bijection from $\a+1$ to $\a$.
      So define $f: \a+1 \to \a$ by
      $$
      f(\b) =
      \begin{cases}
        \b+1 & \b < \w \\
        \b & \b \geq \w \land \b \neq \a \\
        0 & \b = \a
      \end{cases}
      $$
      for $\b \in \a+1$.

      First we show that $f$ is injective.
      So consider any $\b$ and $\g$ in $\a+1$ where $\b \neq \g$.
      Without loss of generality we can assume that $\b < \g$.
      We then have the following:

      Case: $\b < \w$.
      Then clearly $f(\b) = \b + 1 < \w$ since $\b < \w$ and $\w$ is a limit ordinal, but we also clearly have that $0 < \b + 1 = f(\b)$.
      Now, if also $\g < \w$ then clearly $f(\b) = \b + 1 < \g + 1 = f(\g)$ since $\b < \g$.
      If $\g \geq \w$ and $\g \neq \a$ then we have $f(\b) < \w \leq \g = f(\g)$.
      Lastly if $\g = \a$ then we have $f(\g) = 0 < f(\b)$.

      Case: $\b \geq \w$ and $\b \neq \a$.
      Here since $\b < \g$ we have $\w \leq \b < \g$.
      Thus if also $\g \neq \a$ then clearly we have $f(\b) = \b < \g = f(\g)$.
      On the other hand if $\g = \a$ then $f(\g) = 0 < \w \leq \b = f(\b)$.

      Thus in every case we have $f(\b) \neq f(\g)$, thereby showing that $f$ is injective.
      We note that the case in which $\b = \a$ is impossible since $\a$ is the greatest element of $\a+1$ but $\g > \b$ and $\g \in \a+1$.

      Next we show that $f$ is surjective.
      So consider any $\b \in \a$.

      Case: $\b < \w$.
      If $\b = 0$ then clearly $f(\a) = 0 = \b$.
      On the other hand if $0 < \b < \w$ then $\b$ is a successor ordinal, say $\b = \g+1$, so that $\g < \b < \w$ hence clearly $\g \in \a+1$ and $f(\g) = \g+1 = \b$.

      Case: $\b \geq \w$.
      Then since $\b \in \a$ we have $\b < \a < \a+1$ so that $\b \neq \a$ but $\b \in \a+1$.
      Then clearly $f(\b) = \b$.

      Hence in all cases there is a $\g \in \a+1$ such that $f(\g) = \b$ so that $f$ is injective.
      Therefore we have shown that $f$ is a bijection so that by definition $\a+1$ and $\a$ are equipotent. \qedsymbol

      \begin{statement}{Lemma~\ex.\itm{lem:infwo:reord}}
        If an infinite set $A$ with order $\prec$ is isomorphic to an ordinal $\a$ then it can also be re-ordered to be isomorphic to $\a + 1$.
      \end{statement}

      \proof{
        Since $A$ is infinite and the isomorphism from $A$ to $\a$ is a bijective function, it follows that they are equipotent so that $\a$ is also infinite.
        Then by Lemma~\ex.\ref{lem:infwo:succ} $\a$ is equipotent to $\a+1$ so that $A$ is also equipotent to $\a+1$.
        Hence there is an $f:A \to \a+1$ that is bijective.
        We then simply re-order $A$ according to $\a+1$, i.e. we create the following order on $A$:
        $$
        R = \braces{(a,b) \in A \times A \where f(a) < f(b)}
        $$
        so that clearly $(A, R)$ is isomorphic to $(\a+1, <)$. \qedsymbol
      }

      \emph{Main Problem.}

      For an infinite well-orderable set $X$ we show that $X$ has an infinite number of non-isomorphic well-orderings.
      So let $\prec$ be a well-ordering of $X$ so that by Theorem~6.3.1 $(X, \prec)$ is isomorphic to some ordinal $\a$.
      We then show by induction that, for any natural number $n$, there is an ordering $R_n$ of $X$ such that it is isomorphic to $\a + n$.
      For $n=0$ we have that, for $R_0 = \prec$, clearly $(X, R_0)$ is isomorphic to $(\a, <)$ by what has already been established.
      Now suppose that there is an ordering $R_n$ of $X$ such that $(X, R_n)$ is isomorphic to $(\a + n, <)$.
      Then since $X$ is an inifinite set it follows from Lemma~\ex.\ref{lem:infwo:reord} that there is an ordering $R_{n+1}$ such that $X$ is isomorphic to $(\a + n) + 1 = \a + (n+1)$.
      This completes the inductive proof.
      We note that clearly each of these re-orderings are mutually non-isomorphic since different ordinals are not isomorphic to each other. \qedsymbol
    }
\end{solution}

\question{7.1.2}

\begin{solution}
  First we show that $\a + \b$ is at most countable.
  First we define two sets:
  \ali{
    W_1 &= \braces{(0,\g) \where \g \in \a} &
    W_2 &= \braces{(1,\g) \where \g \in \b} \,.
  }
  We also define the order $<_1$ on $W_1$ so that $(0,\g) <_1 (0,\d)$ if and only if $\g < \d$ for $(0, \g)$ and $(0, \d)$ in $W_1$ (so that $\g$ and $\d$ are in $\a$).
  Similarly we define the order $<_2$ on $W_2$ so that $(1,\g) <_2 (1,\d)$ if and only if $\g < \d$ for $(1, \g)$ and $(1, \d)$ in $W_2$ (so that $\g$ and $\d$ are in $\b$).

  Clearly $W_1$ and $W_2$ are disjoint, $(W_1, <_1)$ is isomorphic to $\a$, and $(W_2, <_2)$ is isomorphic to $\b$.
  It then follows from Theorem~6.5.3 that the sum $(W, <)$ is isomorphic to $\a + \b$.

  Now, since they are isomorphic, clearly $W_1$ is equipotent to $\a$ and therefore is at most countable.
  Similarly $W_2$ is at most countable by virtue of being isomorphic to $\b$.
  It then follows from Theorem~4.2.6 and Theorem~4.3.5 that $W = W_1 \cup W_2$ is at most countable.
  Then, since $(W, <)$ is isomorphic to $\a + \b$, $W$ and $\a + \b$ are equipotent so that $\a + \b$ must be at most countable too. \qedsymbol

  Next we show that $\a \cdot \b$ is at most countable.
  Since $\a$ and $\b$ are at most countable it follows from Exercise~4.2.2 and Theorem~4.3.7 that $\a \times \b$ is at most countable.
  Then, since $\a \cdot \b$ is isomorphic to the antilexicographic ordering of $\a \times \b$ by Theorem~6.5.8, it follows that $\a \cdot \b$ is equipotent to $\a \times \b$ and there for at most countable. \qedsymbol

  Lastly we show that $\a^\b$ is at most countable.
  First if we have that $\a = 0$ then then either $\a^\b = 0^\b = 1$ (if $\b = 0$) or $\a^\b = 0^\b = 0$ (if $\b > 0$).
  Clearly both $0$ and $1$ are both at most countable so in the following we assume that $\a \neq 0$.

  \def\seqab{\Seq(\a \cdot \b)}
  \def\sab{S(\a, \b)}
  Now, let $\seqab$ be the set of all finite sequences of elements of $\a \cdot \b$, and $\sab$ be the set as defined in Exercise~6.5.16 such that $\sab$ with the order defined there is isomorphic (and therefore equipotent) to $\a^\b$.
  We shall construct a function $g: \sab \to \seqab$.

  So consider any $f \in \sab$ so that $f:\b \to \a$ and $s(f) = \braces{\x < \b \where f(\x) \neq 0}$ (as defined in the exercise) is finite.
  Hence there is a natural number $n$ such that $s(f)$ can be expressed as an increasing sequence $h: n \to s(f)$.
  We now define another sequence $t: n \to \a \cdot \b$ by
  $$
  t(k) = \a \cdot h(k) + f(h(k))
  $$
  for $k \in n$.
  We then set $g(f) = t$.

  The first thing we show is that $g(f)$ is really a sequence whose elements are in $\a \cdot \b$ for any $f \in \sab$.
  Hence for any such $f$ again let $h$ be the increasing finite sequence whose range is $s(f)$ and let $t = g(f)$.
  Then for any $k \in n$ we we note that $h(k) < \b$ since $h(k) \in s(f)$ so that $h(k) + 1 \leq \b$.
  We also note that $f(h(k)) \in \a$ since $f: \b \to \a$ so that $f(h(k)) < \a$.
  Thus we have by definition that
  \ali{
    t(k) &= \a \cdot h(k) + f(h(k)) \\
    &< \a \cdot h(k) + \a & \text{(by Lemma~6.54)} \\
    &= \a \cdot (h(k) + 1) & \text{(by Definition~6.5.6b)} \\
    &\leq \a \cdot \b \,. & \text{(by Exercise~6.5.7 since $\a \neq 0$)}
  }
  Hence $t(k) < \a \cdot \b$ so that $t(k) \in \a \cdot \b$.
  Thus $t: n \to \a \cdot \b$ and since clearly this sequence is finite it follows that $t \in \seqab$.

  Now we show that $g$ is injective.
  So consider $f_1$ and $f_2$ in $\sab$ such that $f_1 \neq f_2$.
  Let $h_1$, $n_1$ and $h_2$, $n_2$ be the increasing sequences and natural numbers for $f_1$ and $f_2$, respectively, as defined above.
  Also let $t_1 = g(f_1)$ and $t_2 = g(f_2)$.

  Case: $n_1 \neq n_2$.
  In this case the sequences are different sizes so that clearly $t_1 \neq t_2$ since $t_1$ and $t_2$ are sequences of sizes $n_1$ and $n_2$, respectively.

  Case: $n_1 = n_2$.
  Since $f_1 \neq f_2$ there must be a $\g \in \b$ such that $f_1(\g) \neq f_2(\g)$.
  Without loss of generality we can assume that $f_1(\g) < f_2(\g)$.

  If $f_1(\g) = 0$ then by definition $\g \notin s(f_1)$ but $\g \in s(f_2)$.
  Hence there is a $k \in n_2$ such that $h_2(k) = \g$.
  Also since $n_1 = n_2$ clearly $k \in n_1$.
  However, it must be that $h_1(k) \neq \g = h_2(k)$ since otherwise it would be that $\g \in s(f_1)$.
  Suppose that $h_1(k) < h_2(k)$ so that $h_1(k) + 1 \leq h_2(k)$ and we have
  \ali{
    t_1(k) &= \a \cdot h_1(k) + f_1(h_1(k)) \\
    &< \a \cdot h_1(k) + \a & \text{(by Lemma~6.54)} \\
    &= \a \cdot (h_1(k) + 1) & \text{(by Definition~6.5.6b)} \\
    &\leq \a \cdot h_2(k) & \text{(by Exercise~6.5.7 since $\a \neq 0$)} \\
    &\leq \a \cdot h_2(k) + f_2(h_2(k)) \\
    &= t_2(k)
  }
  so that $t_1(k) \neq t_2(k)$ and therefore $t_1 \neq t_2$.
  The case in which $h_1(k) > h_2(k)$ is analagous.

  On the other hand if $f_1(\g) \neq 0$ then $0 < f_1(\g) < f_2(\g)$ so that $\g \in s(f_1)$ and $\g \in s(f_2)$.
  Thus there are $k_1$ and $k_2$ in $n_1 = n_2$ such that $h_1(k_1) = h_2(k_2) = \g$.
  If $k_1 = k_2$ then we have
  \ali{
    t_1(k_1) &= \a \cdot h_1(k_1) + f_1(h_1(k_1)) \\
    &= \a \cdot \g + f_1(\g) \\
    &\neq \a \cdot \g + f_2(\g) & \text{(by Lemma~6.5.4b since $f_1(\g) \neq f_2(\g)$)} \\
    &= \a \cdot h_2(k_2) + f_2(h_2(k_2)) \\
    &= t_2(k_2) = t_2(k_1)
  }
  so that $t_1 \neq  t_2$.
  If $k_1 < k_2$ then since $h_1$ is increasing we have $h_2(k_2) = h_1(k_1) < h_1(k_2)$ so that $h_2(k_2) + 1 \leq h_1(k_2)$ and so
  \ali{
    t_2(k_2) &= \a \cdot h_2(k_2) + f_2(h_2(k_2)) \\
    &< \a \cdot h_2(k_2) + \a & \text{(by Lemma~6.54)} \\
    &= \a \cdot (h_2(k_2) + 1) & \text{(by Definition~6.5.6b)} \\
    &\leq \a \cdot h_1(k_2) & \text{(by Exercise~6.5.7 since $\a \neq 0$)} \\
    &\leq \a \cdot h_1(k_2) + f_1(h_1(k_2)) \\
    &= t_1(k_2)
  }
  and $t_1 \neq t_2$.
  The final sub-case in which $k_1 > k_2$ is analagous.
  
  Hence in all cases and sub-cases $g(f_1) = t_1 \neq t_2 = g(f_2)$, which shows that $g$ is injective.
  From this it follows from Definition~4.1.4 that $\abs{\sab} \leq \abs{\seqab}$.
  However, from what was shown above it follows that $\a \cdot \b$ is at most countable since $\a$ and $\b$ are.
  Thus $\seqab$ is also at most countable by Exercise~4.2.4 and Theorem~4.3.10 so that $\sab$ must be at most countable since it was just shown that $\abs{\sab} \leq \abs{\seqab}$.
  Lastly, since $\sab$ is equipotent to $\a^\b$ it follows that $\a^\b$ is at most countable as well. \qedsymbol
\end{solution}

\def\ex{7.1.3}
\setcounter{itm}{0}
\question{\ex}

\begin{solution}
    \begin{statement}{Lemma~\ex.\itm{lem:pset:sslt}}
        For a set $A$ and ordinal $\a$, if $\a$ is equipotent to a subset of $A$ then $\a < h(A)$.
    \end{statement}

    \proof{
        Suppose that $\a$ is equipotent to $X \ss A$ but that $\a \geq h(A)$.
        Clearly if $\a = h(A)$ then $h(A)$ is equipotent to $X \ss A$ (since $\a$ is), which contradicts the definition of the Hartogs number.
        On the other hand if $\a > h(A)$ then let $f$ be a bijection from $\a$ to $X$.
        Then, since $h(A) < \a$ we have that $h(A) \in \a$ and $h(A) \ss \a$ since ordinals are transitive.
        It then follows that $f \rest h(A)$ is a bijection from $h(A)$ to $f[h(A)] \ss X \ss A$.
        Hence again $h(A)$ is equipotent to a subset of $A$, contradicting the definition of the Hartogs number.
        So it has to be that $\a < h(A)$ as desired. \qedsymbol
    }

    \emph{Main Problem.}

	We define a function $f : \pset{A \times A} \to h(A)$.
    So for any $R \in \pset{A \times A}$ clearly $R \ss A \times A$ so that $R$ is a relation on $A$.
    If $R$ is a well-ordering of some $X \ss A$ then by Theorem~6.3.1 there is a unique ordinal $\a$ such that $(X, R)$ is isomorphic to $\a$.
    We then set
    $$
    f(R) = \begin{cases}
         \a & \text{$R$ is a well-ordering of some $X \ss A$} \\
         0 & \text{$R$ is not a well-ordering of any $X \ss A$} \\
    \end{cases} \,.
    $$

    First we show that $f(R)$ really is in $h(A)$ for any $R \in \pset{A \times A}$.
    So for any such $R$, if $R$ is not a well-ordering of some $X \ss A$ then clearly $f(R) = 0$.
    Then, since clearly $\es \ss A$ and $\es$ is equipotent to 0 it follows that $0 < h(A)$ by Lemma~\ex.\ref{lem:pset:sslt}.
    Hence $f(R) = 0 \in h(A)$.\
    On the other hand if $R$ is a well-ordering of some $X \ss A$ then let $\a$ be the ordinal isomorphic to $(X,R)$ so that $f(R) = \a$.
    Since this means that $\a$ is equipotent to $X \ss A$, it again follows from Lemma~\ex.\ref{lem:pset:sslt} that $\a < h(A)$ so that $f(R) = \a \in h(A)$.

    To show that $f$ is surjective consider any $\a \in h(A)$ so that $\a < h(A)$.
    Since by definition $h(A)$ is the \emph{least} ordinal that is not equipotent to a subset of $A$ it follows that $\a$ has to be equipotent to an $X \ss A$.
    Then let $R$ be the well-ordering of $X$ such that $(X,R)$ is isomorphic to $\a$.
    Clearly $R$ is a relation on $X$ and therefore also a relation on $A$ since $X \ss A$.
    Thus $R \ss A \times A$ so that $R \in \pset{A \times A}$.
    Clearly also $f(R) = \a$ and since $\a$ was arbitrary this shows that $f$ is surjective. \qedsymbol

    Note that this does not mean that $h(A) \leq \abs{\pset{A \times A}}$ unless the Axiom of Choice is employed.
\end{solution}

\def\ex{7.1.4}
\setcounter{itm}{0}
\question{\ex}

\begin{solution}
    \begin{statement}{Lemma~\ex.\itm{lem:lthart:lthart}}
    $\abs{h(A)} \nleq \abs{A}$ for any set $A$.
    \end{statement}

    \proof{
        Suppose to the contrary that $\abs{h(A)} \leq \abs{A}$ so that there is an injective $f$ from $h(A)$ to $A$.
        Then let $X = f[h(A)]$ so that clearly $X \ss A$.
        But then $f$ considered as function from $h(A)$ to $X$ is a bijection so that $h(A)$ is equipotent to a subset of $A$, which contradicts the definition of the Hartogs number.
        Hence it must be that $\abs{h(A)} \nleq \abs{A}$ as desired. \qedsymbol
        
        Note that this does not imply that $h(a) > \abs{A}$ without using the Axiom of Choice.
    }

    \emph{Main Problem.}
    
	Since we are only interested in the cardinalities of $A$ and $h(A)$ and their cardinal sum we can assume that they are disjoint.
    Clearly $f: A \to A \cup h(A)$ defined by $f(x) = x$ for $x \in A$ is an injective function so that $\abs{A} \leq \abs{A} + h(A)$ by the definition of cardinal addition.
    So suppose that $\abs{A} = \abs{A} + h(A)$.
    Then there is a bijective function $f$ from $A \cup h(A)$ into $A$ so that $f \rest h(A)$ is an injective function from $h(A)$ to $A$.
    By definition this means that $h(A) \leq \abs{A}$, but this contradicts Lemma~\ex.\ref{lem:lthart:lthart}.
    So it must be that $\abs{A} < \abs{A} + h(A)$ as desired. \qedsymbol
\end{solution}

\question{7.1.5}

\begin{solution}
    \def\pha{\pset{h(A)}}
    \def\paa{\pset{\pset{A \times A}}}
	First we show that $\abs{\pha} \leq \abs{\paa}$ by constructing an injective $f : \pha \to \paa$.
    So consider any $X \in \pha$ so that $X \ss h(A)$.
    Then let $Y$ be the set of well-orderings $R \ss A \times A$ (so that $R \in \pset{A \times A}$) of subsets $B \ss A$ such that $(B,R)$ is isomorphic to some $\a \in X$.
    We then set $f(X) = Y$, noting that clearly $f(X) = Y \in \paa$ since for any $R \in Y$ we have that $R \in \pset{A \times A}$ so that $Y \ss \pset{A \times A}$ hence $Y \in \paa$.
    Note also that $Y \neq \es$ because every $\a \in h(A)$ is equipotent to some subset $B \ss A$ (by the definition of the Hartogs number) so that the well-ordering of $B$ according to $\a$ will be in $Y$.

    We claim that $f$ is injective.
    So consider $X_1$ and $X_2$ in $\pha$ (so that $X_1 \ss h(A)$ and $X_2 \ss h(A)$) such that $f(X_1) = f(X_2)$.
    Then consider any $\a \in X_1$.
    Then since $f(X_1) \neq \es$ there is a well-ordering $R \in f(X_1)$ of a subset of $A$ that is isomorphic to $\a$.
    Then since $f(X_1) = f(X_2)$ we have $R \in f(X_2)$ as well.
    It follows from this that $\a \in X_2$.
    Thus $X_1 \ss X_2$ since $\a$ was arbitrary.
    A similar argument shows that $X_2 \ss X_1$ so that we conclude that $X_1 = X_2$.
    This shows that $f$ is injective.

    Hence we have shown that $\abs{\pha} \leq \abs{\paa}$.
    We also have by Cantor's Theorem (Theorem~5.1.8 in the text) that $\abs{h(A)} < \abs{\pha}$.
    It therefore follows from Exercise~4.1.2a that $\abs{h(A)} < \abs{\paa}$ as desired. \qedsymbol
\end{solution}

\def\ex{7.1.6}
\setcounter{itm}{0}
\question{\ex}

\begin{solution}
    \def\hs{h^*}

    \begin{statement}{Lemma~\ex.\itm{lem:hs:onto}}
        If $A$ is a well-orderable set, $B$ is any other set, and there is a function from $A$ onto $B$ then $\abs{B} \leq \abs{A}$.
    \end{statement}

    \proof{
        Let $f$ be a function from $A$ onto $B$ and suppose that $R$ is a well-ordering of $A$.
        For each $b \in B$ define the set $A_b = \braces{a \in A \where f(a) = b}$, which clearly not empty since $f$ is onto.
        Then since $R$ is a well-ordering of $A$ and $A_b \ss A$ there is a unique least element of $a_b$ of $A_b$ according to $R$.
        We then define $g: B \to A$ by simply setting $g(b)  = a_b$ for any $b \in B$.

        We then claim that $g$ is injective.
        So consider $b_1$ and $b_2$ in $B$ where $b_1 \neq b_2$.
        Then since $a_{b_1} \in A_{b_1}$ clearly $f(a_{b_1}) = b_1$.
        Similarly $f(a_{b_2}) = b_2$ so that clearly $a_{b_1} \neq a_{b_2}$ since $f$ is a function and $b_1 \neq b_2$.
        Hence $g(b_1) = a_{b_1} \neq a_{b_2} = g(b_2)$, which shows that $g$ is injective since $b_1$ and $b_2$ were arbitrary.
        Then by definition $\abs{B} \leq \abs{A}$ as desired. \qedsymbol
    }

    \emph{Main Problem.}

    First we note that presumably $\hs(A)$ for a set $A$ is the least \emph{nonzero} ordinal $\a$ such that there is no function from $A$ onto $\a$.
    The fact that $\hs(A)$ is nonzero is important since, for a non-empty set $A$, there is no function from $A$ onto $\es = 0$ so that $0$ is actually the least ordinal such that such a function does not exist!
    We also make note of the fact that, even for $A=\es$, the empty function $f = \es$ is a vacuously a function from $A$ onto $\es = 0$ so that $\hs(A) = 1$ anyway since there is no function from $A = \es$ onto $1 = \braces{0}$.

	(a) Suppose to the contrary that there \emph{is} a function $f$ from $A$ onto $\a$.
    Then since $0 < \hs(A) \leq \a$ clearly $0 \in \hs(A)$ and $\hs(A) \ss \a$.
    So we define a function $g : A \to \hs(A)$ by
    $$
    g(a) = \begin{cases}
         f(a) & f(a) \in \hs(A) \\
         0 & f(a) \notin \hs(A)
    \end{cases} \,.
    $$
    for any $a \in A$.
    Clearly each $g(a) \in \hs(A)$ but we also claim that $g$ is onto.
    To this end consider any $\b \in \hs(A)$.
    Since $\hs(A) \ss \a$ we have that $\b \in \a$ also.
    Then, since $f$ is onto $\a$ there is an $a \in A$ such that $f(a) = \b$.
    Since $f(a) = \b \in \hs(A)$ it follows by definition that $g(a) = f(a) = \b$.
    Since $\b$ was arbitrary this shows that $g$ is onto.
    However, the existence of $g$  contradicts the definition of $\hs(A)$ so that it must be that there is no such function from $A$ onto $\a$. \qedsymbol

    (b) Suppose to the contrary that $\hs(A)$ is \emph{not} an initial ordinal so that there is an $\a < \hs(A)$ such that $\abs{\a} = \abs{\hs(A)}$.
    Let $f$ then be a bijection from $\a$ onto $\hs(A)$.
    Also since $\a < \hs(A)$ it follows from the definition of $\hs(A)$ that there is a function $g$ from $A$ onto $\a$ (since otherwise $\hs(A)$ would not be the \emph{least} such ordinal for which such a function does not exist).
    But then $f \circ g$ is a function from $A$ onto $\hs(A)$, which contradicts the definition of $\hs(A)$.
    Hence it has to be that $\hs(A)$ is in fact an initial ordinal.

    (c) Suppose to the contrary that $h(A) > \hs(A)$.
    Then by the definition of $h(A)$ there is a subset $X \ss A$ such that $\hs(A)$ is equipotent to $X$.
    So let $f$ be a bijection from $\hs(A)$ to $X$.
    We then define a function $g : A \to \hs(A)$ by
    $$
    g(a) = \begin{cases}
         \inv{f}(a) & a \in X \\
         0 & a \notin X
    \end{cases}
    $$
    for any $a \in A$, noting that $0 < \hs(A)$ so that $0 \in \hs(A)$.
    Clearly $g$ is into $\hs(A)$ but we also claim that it is onto.
    So consider any $\a \in \hs(A)$ and let $a = f(\a)$ so that $a \in X$ and therefore $a \in A$ since $X \ss A$.
    We then have $g(a) = \inv{f}(a) = \inv{f}(f(\a)) = \a$ since $a \in X$.
    Since $\a$ was arbitrary this shows that $g$ is onto.
    However, the existence of $g$ contradicts the definition of $\hs(A)$ so that it must be that in fact $h(A) \leq \hs(A)$ as desired. \qedsymbol

    (d) We show that $\hs(A) \leq h(A)$, from which the result clearly follows since also $\hs(A) \geq h(A)$ by part (c).
    So suppose to the contrary that $\hs(A) > h(A)$.
    Then by the definition of $\hs(A)$ it follows that there is a function from $A$ onto $h(A)$.
    It then  follows from Lemma~\ex.\ref{lem:hs:onto} that $\abs{h(A)} \leq \abs{A}$ since $A$ is well-orderable.
    However, this contradict Lemma~7.1.4.\ref{lem:lthart:lthart} so that it must be that $\hs(A) \leq h(A)$ so that the result follows.

    (e) Consider any set $A$ and let $S$ denote the set of well-orderings of some partition of $A$ into equivalence classes, noting that it could be that $S = \es$.
    Since each $R \in S$ is isomorphic to a unique ordinal, let $H$ be the set of ordinals that are isomorphic to some $R \in S$, which exists by the Axiom Schema of Replacement.
    Then let $\a = \braces{0} \cup H$ and we claim that $\a = \hs(A)$.

    First we show that $\a$ is indeed an ordinal number.
    Since $\a$ is a set of ordinals clearly it is well-ordered by Theorem~6.2.6d.
    We also must show that $\a$ is transitive, so consider any $\b \in \a$ 
\end{solution}
