\def\ex{7.1.1}
\setcounter{itm}{0}
\question{\ex}

\begin{solution}
    \begin{statement}{Lemma~\ex.\itm{lem:infwo:succ}}
        If $\a$ is an infinite ordinal then $\abs{\a} = \abs{\a+1}$, i.e. $\a$ and $\a+1$ are equipotent.
    \end{statement}

    \proof{
      First we note that since $\a$ is inifinite we have $\a + 1 > \a \geq \w$.
      We then construct a bijection from $\a+1$ to $\a$.
      So define $f: \a+1 \to \a$ by
      $$
      f(\b) =
      \begin{cases}
        \b+1 & \b < \w \\
        \b & \b \geq \w \land \b \neq \a \\
        0 & \b = \a
      \end{cases}
      $$
      for $\b \in \a+1$.

      First we show that $f$ is injective.
      So consider any $\b$ and $\g$ in $\a+1$ where $\b \neq \g$.
      Without loss of generality we can assume that $\b < \g$.
      We then have the following:

      Case: $\b < \w$.
      Then clearly $f(\b) = \b + 1 < \w$ since $\b < \w$ and $\w$ is a limit ordinal, but we also clearly have that $0 < \b + 1 = f(\b)$.
      Now, if also $\g < \w$ then clearly $f(\b) = \b + 1 < \g + 1 = f(\g)$ since $\b < \g$.
      If $\g \geq \w$ and $\g \neq \a$ then we have $f(\b) < \w \leq \g = f(\g)$.
      Lastly if $\g = \a$ then we have $f(\g) = 0 < f(\b)$.

      Case: $\b \geq \w$ and $\b \neq \a$.
      Here since $\b < \g$ we have $\w \leq \b < \g$.
      Thus if also $\g \neq \a$ then clearly we have $f(\b) = \b < \g = f(\g)$.
      On the other hand if $\g = \a$ then $f(\g) = 0 < \w \leq \b = f(\b)$.

      Thus in every case we have $f(\b) \neq f(\g)$, thereby showing that $f$ is injective.
      We note that the case in which $\b = \a$ is impossible since $\a$ is the greatest element of $\a+1$ but $\g > \b$ and $\g \in \a+1$.

      Next we show that $f$ is surjective.
      So consider any $\b \in \a$.

      Case: $\b < \w$.
      If $\b = 0$ then clearly $f(\a) = 0 = \b$.
      On the other hand if $0 < \b < \w$ then $\b$ is a successor ordinal, say $\b = \g+1$, so that $\g < \b < \w$ hence clearly $\g \in \a+1$ and $f(\g) = \g+1 = \b$.

      Case: $\b \geq \w$.
      Then since $\b \in \a$ we have $\b < \a < \a+1$ so that $\b \neq \a$ but $\b \in \a+1$.
      Then clearly $f(\b) = \b$.

      Hence in all cases there is a $\g \in \a+1$ such that $f(\g) = \b$ so that $f$ is injective.
      Therefore we have shown that $f$ is a bijection so that by definition $\a+1$ and $\a$ are equipotent. \qedsymbol

      \begin{statement}{Lemma~\ex.\itm{lem:infwo:reord}}
        If an infinite set $A$ with order $\prec$ is isomorphic to an ordinal $\a$ then it can also be re-ordered to be isomorphic to $\a + 1$.
      \end{statement}

      \proof{
        Since $A$ is infinite and the isomorphism from $A$ to $\a$ is a bijective function, it follows that they are equipotent so that $\a$ is also infinite.
        Then by Lemma~\ex.\ref{lem:infwo:succ} $\a$ is equipotent to $\a+1$ so that $A$ is also equipotent to $\a+1$.
        Hence there is an $f:A \to \a+1$ that is bijective.
        We then simply re-order $A$ according to $\a+1$, i.e. we create the following order on $A$:
        $$
        R = \braces{(a,b) \in A \times A \where f(a) < f(b)}
        $$
        so that clearly $(A, R)$ is isomorphic to $(\a+1, <)$. \qedsymbol
      }

      \emph{Main Problem.}

      For an infinite well-orderable set $X$ we show that $X$ has an infinite number of non-isomorphic well-orderings.
      So let $\prec$ be a well-ordering of $X$ so that by Theorem~6.3.1 $(X, \prec)$ is isomorphic to some ordinal $\a$.
      We then show by induction that, for any natural number $n$, there is an ordering $R_n$ of $X$ such that it is isomorphic to $\a + n$.
      For $n=0$ we have that, for $R_0 = \prec$, clearly $(X, R_0)$ is isomorphic to $(\a, <)$ by what has already been established.
      Now suppose that there is an ordering $R_n$ of $X$ such that $(X, R_n)$ is isomorphic to $(\a + n, <)$.
      Then since $X$ is an inifinite set it follows from Lemma~\ex.\ref{lem:infwo:reord} that there is an ordering $R_{n+1}$ such that $X$ is isomorphic to $(\a + n) + 1 = \a + (n+1)$.
      This completes the inductive proof.
      We note that clearly each of these re-orderings are mutually non-isomorphic since different ordinals are not isomorphic to each other. \qedsymbol
    }
\end{solution}

\question{7.1.2}

\begin{solution}
  First we show that $\a + \b$ is at most countable.
  First we define two sets:
  \ali{
    W_1 &= \braces{(0,\g) \where \g \in \a} &
    W_2 &= \braces{(1,\g) \where \g \in \b} \,.
  }
  We also define the order $<_1$ on $W_1$ so that $(0,\g) <_1 (0,\d)$ if and only if $\g < \d$ for $(0, \g)$ and $(0, \d)$ in $W_1$ (so that $\g$ and $\d$ are in $\a$).
  Similarly we define the order $<_2$ on $W_2$ so that $(1,\g) <_2 (1,\d)$ if and only if $\g < \d$ for $(1, \g)$ and $(1, \d)$ in $W_2$ (so that $\g$ and $\d$ are in $\b$).

  Clearly $W_1$ and $W_2$ are disjoint, $(W_1, <_1)$ is isomorphic to $\a$, and $(W_2, <_2)$ is isomorphic to $\b$.
  It then follows from Theorem~6.5.3 that the sum $(W, <)$ is isomorphic to $\a + \b$.

  Now, since they are isomorphic, clearly $W_1$ is equipotent to $\a$ and therefore is at most countable.
  Similarly $W_2$ is at most countable by virtue of being isomorphic to $\b$.
  It then follows from Theorem~4.2.6 and Theorem~4.3.5 that $W = W_1 \cup W_2$ is at most countable.
  Then, since $(W, <)$ is isomorphic to $\a + \b$, $W$ and $\a + \b$ are equipotent so that $\a + \b$ must be at most countable too. \qedsymbol

  Next we show that $\a \cdot \b$ is at most countable.
  Since $\a$ and $\b$ are at most countable it follows from Exercise~4.2.2 and Theorem~4.3.7 that $\a \times \b$ is at most countable.
  Then, since $\a \cdot \b$ is isomorphic to the antilexicographic ordering of $\a \times \b$ by Theorem~6.5.8, it follows that $\a \cdot \b$ is equipotent to $\a \times \b$ and there for at most countable. \qedsymbol

  Lastly we show that $\a^\b$ is at most countable.
  First if we have that $\a = 0$ then then either $\a^\b = 0^\b = 1$ (if $\b = 0$) or $\a^\b = 0^\b = 0$ (if $\b > 0$).
  Clearly both $0$ and $1$ are both at most countable so in the following we assume that $\a \neq 0$.

  \def\seqab{\Seq(\a \cdot \b)}
  \def\sab{S(\a, \b)}
  Now, let $\seqab$ be the set of all finite sequences of elements of $\a \cdot \b$, and $\sab$ be the set as defined in Exercise~6.5.16 such that $\sab$ with the order defined there is isomorphic (and therefore equipotent) to $\a^\b$.
  We shall construct a function $g: \sab \to \seqab$.

  So consider any $f \in \sab$ so that $f:\b \to \a$ and $s(f) = \braces{\x < \b \where f(\x) \neq 0}$ (as defined in the exercise) is finite.
  Hence there is a natural number $n$ such that $s(f)$ can be expressed as an increasing sequence $h: n \to s(f)$.
  We now define another sequence $t: n \to \a \cdot \b$ by
  $$
  t(k) = \a \cdot h(k) + f(h(k))
  $$
  for $k \in n$.
  We then set $g(f) = t$.

  The first thing we show is that $g(f)$ is really a sequence whose elements are in $\a \cdot \b$ for any $f \in \sab$.
  Hence for any such $f$ again let $h$ be the increasing finite sequence whose range is $s(f)$ and let $t = g(f)$.
  Then for any $k \in n$ we we note that $h(k) < \b$ since $h(k) \in s(f)$ so that $h(k) + 1 \leq \b$.
  We also note that $f(h(k)) \in \a$ since $f: \b \to \a$ so that $f(h(k)) < \a$.
  Thus we have by definition that
  \ali{
    t(k) &= \a \cdot h(k) + f(h(k)) \\
    &< \a \cdot h(k) + \a & \text{(by Lemma~6.54)} \\
    &= \a \cdot (h(k) + 1) & \text{(by Definition~6.5.6b)} \\
    &\leq \a \cdot \b \,. & \text{(by Exercise~6.5.7 since $\a \neq 0$)}
  }
  Hence $t(k) < \a \cdot \b$ so that $t(k) \in \a \cdot \b$.
  Thus $t: n \to \a \cdot \b$ and since clearly this sequence is finite it follows that $t \in \seqab$.

  Now we show that $g$ is injective.
  So consider $f_1$ and $f_2$ in $\sab$ such that $f_1 \neq f_2$.
  Let $h_1$, $n_1$ and $h_2$, $n_2$ be the increasing sequences and natural numbers for $f_1$ and $f_2$, respectively, as defined above.
  Also let $t_1 = g(f_1)$ and $t_2 = g(f_2)$.

  Case: $n_1 \neq n_2$.
  In this case the sequences are different sizes so that clearly $t_1 \neq t_2$ since $t_1$ and $t_2$ are sequences of sizes $n_1$ and $n_2$, respectively.

  Case: $n_1 = n_2$.
  Since $f_1 \neq f_2$ there must be a $\g \in \b$ such that $f_1(\g) \neq f_2(\g)$.
  Without loss of generality we can assume that $f_1(\g) < f_2(\g)$.

  If $f_1(\g) = 0$ then by definition $\g \notin s(f_1)$ but $\g \in s(f_2)$.
  Hence there is a $k \in n_2$ such that $h_2(k) = \g$.
  Also since $n_1 = n_2$ clearly $k \in n_1$.
  However, it must be that $h_1(k) \neq \g = h_2(k)$ since otherwise it would be that $\g \in s(f_1)$.
  Suppose that $h_1(k) < h_2(k)$ so that $h_1(k) + 1 \leq h_2(k)$ and we have
  \ali{
    t_1(k) &= \a \cdot h_1(k) + f_1(h_1(k)) \\
    &< \a \cdot h_1(k) + \a & \text{(by Lemma~6.54)} \\
    &= \a \cdot (h_1(k) + 1) & \text{(by Definition~6.5.6b)} \\
    &\leq \a \cdot h_2(k) & \text{(by Exercise~6.5.7 since $\a \neq 0$)} \\
    &\leq \a \cdot h_2(k) + f_2(h_2(k)) \\
    &= t_2(k)
  }
  so that $t_1(k) \neq t_2(k)$ and therefore $t_1 \neq t_2$.
  The case in which $h_1(k) > h_2(k)$ is analagous.

  On the other hand if $f_1(\g) \neq 0$ then $0 < f_1(\g) < f_2(\g)$ so that $\g \in s(f_1)$ and $\g \in s(f_2)$.
  Thus there are $k_1$ and $k_2$ in $n_1 = n_2$ such that $h_1(k_1) = h_2(k_2) = \g$.
  If $k_1 = k_2$ then we have
  \ali{
    t_1(k_1) &= \a \cdot h_1(k_1) + f_1(h_1(k_1)) \\
    &= \a \cdot \g + f_1(\g) \\
    &\neq \a \cdot \g + f_2(\g) & \text{(by Lemma~6.5.4b since $f_1(\g) \neq f_2(\g)$)} \\
    &= \a \cdot h_2(k_2) + f_2(h_2(k_2)) \\
    &= t_2(k_2) = t_2(k_1)
  }
  so that $t_1 \neq  t_2$.
  If $k_1 < k_2$ then since $h_1$ is increasing we have $h_2(k_2) = h_1(k_1) < h_1(k_2)$ so that $h_2(k_2) + 1 \leq h_1(k_2)$ and so
  \ali{
    t_2(k_2) &= \a \cdot h_2(k_2) + f_2(h_2(k_2)) \\
    &< \a \cdot h_2(k_2) + \a & \text{(by Lemma~6.54)} \\
    &= \a \cdot (h_2(k_2) + 1) & \text{(by Definition~6.5.6b)} \\
    &\leq \a \cdot h_1(k_2) & \text{(by Exercise~6.5.7 since $\a \neq 0$)} \\
    &\leq \a \cdot h_1(k_2) + f_1(h_1(k_2)) \\
    &= t_1(k_2)
  }
  and $t_1 \neq t_2$.
  The final sub-case in which $k_1 > k_2$ is analagous.
  
  Hence in all cases and sub-cases $g(f_1) = t_1 \neq t_2 = g(f_2)$, which shows that $g$ is injective.
  From this it follows from Definition~4.1.4 that $\abs{\sab} \leq \abs{\seqab}$.
  However, from what was shown above it follows that $\a \cdot \b$ is at most countable since $\a$ and $\b$ are.
  Thus $\seqab$ is also at most countable by Exercise~4.2.4 and Theorem~4.3.10 so that $\sab$ must be at most countable since it was just shown that $\abs{\sab} \leq \abs{\seqab}$.
  Lastly, since $\sab$ is equipotent to $\a^\b$ it follows that $\a^\b$ is at most countable as well. \qedsymbol
\end{solution}

\def\ex{7.1.3}
\setcounter{itm}{0}
\question{\ex}

\begin{solution}
    \begin{statement}{Lemma~\ex.\itm{lem:pset:sslt}}
        For a set $A$ and ordinal $\a$, if $\a$ is equipotent to a subset of $A$ then $\a < h(A)$.
    \end{statement}

    \proof{
        Suppose that $\a$ is equipotent to $X \ss A$ but that $\a \geq h(A)$.
        Clearly if $\a = h(A)$ then $h(A)$ is equipotent to $X \ss A$ (since $\a$ is), which contradicts the definition of the Hartogs number.
        On the other hand if $\a > h(A)$ then let $f$ be a bijection from $\a$ to $X$.
        Then, since $h(A) < \a$ we have that $h(A) \in \a$ and $h(A) \ss \a$ since ordinals are transitive.
        It then follows that $f \rest h(A)$ is a bijection from $h(A)$ to $f[h(A)] \ss X \ss A$.
        Hence again $h(A)$ is equipotent to a subset of $A$, contradicting the definition of the Hartogs number.
        So it has to be that $\a < h(A)$ as desired. \qedsymbol
    }

    \emph{Main Problem.}

	We define a function $f : \pset{A \times A} \to h(A)$.
    So for any $R \in \pset{A \times A}$ clearly $R \ss A \times A$ so that $R$ is a relation on $A$.
    If $R$ is a well-ordering of some $X \ss A$ then by Theorem~6.3.1 there is a unique ordinal $\a$ such that $(X, R)$ is isomorphic to $\a$.
    We then set
    $$
    f(R) = \begin{cases}
         \a & \text{$R$ is a well-ordering of some $X \ss A$} \\
         0 & \text{$R$ is not a well-ordering of any $X \ss A$} \\
    \end{cases} \,.
    $$

    First we show that $f(R)$ really is in $h(A)$ for any $R \in \pset{A \times A}$.
    So for any such $R$, if $R$ is not a well-ordering of some $X \ss A$ then clearly $f(R) = 0$.
    Then, since clearly $\es \ss A$ and $\es$ is equipotent to 0 it follows that $0 < h(A)$ by Lemma~\ex.\ref{lem:pset:sslt}.
    Hence $f(R) = 0 \in h(A)$.\
    On the other hand if $R$ is a well-ordering of some $X \ss A$ then let $\a$ be the ordinal isomorphic to $(X,R)$ so that $f(R) = \a$.
    Since this means that $\a$ is equipotent to $X \ss A$, it again follows from Lemma~\ex.\ref{lem:pset:sslt} that $\a < h(A)$ so that $f(R) = \a \in h(A)$.

    To show that $f$ is surjective consider any $\a \in h(A)$ so that $\a < h(A)$.
    Since by definition $h(A)$ is the \emph{least} ordinal that is not equipotent to a subset of $A$ it follows that $\a$ has to be equipotent to an $X \ss A$.
    Then let $R$ be the well-ordering of $X$ such that $(X,R)$ is isomorphic to $\a$.
    Clearly $R$ is a relation on $X$ and therefore also a relation on $A$ since $X \ss A$.
    Thus $R \ss A \times A$ so that $R \in \pset{A \times A}$.
    Clearly also $f(R) = \a$ and since $\a$ was arbitrary this shows that $f$ is surjective. \qedsymbol

    Note that this does not mean that $h(A) \leq \abs{\pset{A \times A}}$ unless the Axiom of Choice is employed.
\end{solution}

\def\ex{7.1.4}
\setcounter{itm}{0}
\question{\ex}

\begin{solution}
    \begin{statement}{Lemma~\ex.\itm{lem:lthart:lthart}}
    $\abs{h(A)} \nleq \abs{A}$ for any set $A$.
    \end{statement}

    \proof{
        Suppose to the contrary that $\abs{h(A)} \leq \abs{A}$ so that there is an injective $f$ from $h(A)$ to $A$.
        Then let $X = f[h(A)]$ so that clearly $X \ss A$.
        But then $f$ considered as function from $h(A)$ to $X$ is a bijection so that $h(A)$ is equipotent to a subset of $A$, which contradicts the definition of the Hartogs number.
        Hence it must be that $\abs{h(A)} \nleq \abs{A}$ as desired. \qedsymbol
        
        Note that this does not imply that $h(a) > \abs{A}$ without using the Axiom of Choice.
    }

    \emph{Main Problem.}
    
	Since we are only interested in the cardinalities of $A$ and $h(A)$ and their cardinal sum we can assume that they are disjoint.
    Clearly $f: A \to A \cup h(A)$ defined by $f(x) = x$ for $x \in A$ is an injective function so that $\abs{A} \leq \abs{A} + h(A)$ by the definition of cardinal addition.
    So suppose that $\abs{A} = \abs{A} + h(A)$.
    Then there is a bijective function $f$ from $A \cup h(A)$ into $A$ so that $f \rest h(A)$ is an injective function from $h(A)$ to $A$.
    By definition this means that $h(A) \leq \abs{A}$, but this contradicts Lemma~\ex.\ref{lem:lthart:lthart}.
    So it must be that $\abs{A} < \abs{A} + h(A)$ as desired. \qedsymbol
\end{solution}

\question{7.1.5}

\begin{solution}
    \def\pha{\pset{h(A)}}
    \def\paa{\pset{\pset{A \times A}}}
	First we show that $\abs{\pha} \leq \abs{\paa}$ by constructing an injective $f : \pha \to \paa$.
    So consider any $X \in \pha$ so that $X \ss h(A)$.
    Then let $Y$ be the set of well-orderings $R \ss A \times A$ (so that $R \in \pset{A \times A}$) of subsets $B \ss A$ such that $(B,R)$ is isomorphic to some $\a \in X$.
    We then set $f(X) = Y$, noting that clearly $f(X) = Y \in \paa$ since for any $R \in Y$ we have that $R \in \pset{A \times A}$ so that $Y \ss \pset{A \times A}$ hence $Y \in \paa$.
    Note also that $Y \neq \es$ because every $\a \in h(A)$ is equipotent to some subset $B \ss A$ (by the definition of the Hartogs number) so that the well-ordering of $B$ according to $\a$ will be in $Y$.

    We claim that $f$ is injective.
    So consider $X_1$ and $X_2$ in $\pha$ (so that $X_1 \ss h(A)$ and $X_2 \ss h(A)$) such that $f(X_1) = f(X_2)$.
    Then consider any $\a \in X_1$.
    Then since $f(X_1) \neq \es$ there is a well-ordering $R \in f(X_1)$ of a subset of $A$ that is isomorphic to $\a$.
    Then since $f(X_1) = f(X_2)$ we have $R \in f(X_2)$ as well.
    It follows from this that $\a \in X_2$.
    Thus $X_1 \ss X_2$ since $\a$ was arbitrary.
    A similar argument shows that $X_2 \ss X_1$ so that we conclude that $X_1 = X_2$.
    This shows that $f$ is injective.

    Hence we have shown that $\abs{\pha} \leq \abs{\paa}$.
    We also have by Cantor's Theorem (Theorem~5.1.8 in the text) that $\abs{h(A)} < \abs{\pha}$.
    It therefore follows from Exercise~4.1.2a that $\abs{h(A)} < \abs{\paa}$ as desired. \qedsymbol
\end{solution}

\def\ex{7.1.6}
\setcounter{itm}{0}
\question{\ex}
\begin{solution}
    \def\hs{h^*}

    \begin{statement}{Lemma~\ex.\itm{lem:hs:onto}}
        If $A$ is a well-orderable set and $B$ is any other set, there is a function from $A$ onto $B$ if and only if $\abs{B} \leq \abs{A}$.
    \end{statement}

    \proof{
        Suppose that $R$ is a well-ordering of $A$.

        $(\to)$ Suppose that $f$ is a function from $A$ onto $B$.
        If $A$ is empty then clearly $B$ must be as well or else $f$ could not be onto.
        Thus we have $\abs{B} = \abs{\es} = 0 \leq 0 = \abs{\es} = \abs{A}$.
        So we can assume that $A$ is nonempty so that if $B$ is empty then $\abs{B} = \abs{\es} = 0 < \abs{A}$.
        Hence we can assume that $B$ is nonempty as well.

        Then, for each $b \in B$, define the set $A_b = \braces{a \in A \where f(a) = b}$, which clearly not empty since $f$ is onto.
        Then, since $R$ is a well-ordering of $A$ and $A_b \ss A$, there is a unique least element of $a_b$ of $A_b$ according to $R$.
        We then define $g: B \to A$ by simply setting $g(b)  = a_b$ for any $b \in B$.

        We then claim that $g$ is injective.
        So consider $b_1$ and $b_2$ in $B$ where $b_1 \neq b_2$.
        Then since $a_{b_1} \in A_{b_1}$ clearly $f(a_{b_1}) = b_1$.
        Similarly $f(a_{b_2}) = b_2$ so that clearly $a_{b_1} \neq a_{b_2}$ since $f$ is a function and $b_1 \neq b_2$.
        Hence $g(b_1) = a_{b_1} \neq a_{b_2} = g(b_2)$, which shows that $g$ is injective since $b_1$ and $b_2$ were arbitrary.
        Then by definition $\abs{B} \leq \abs{A}$ as desired.

        $(\leftarrow)$ Now suppose that $\abs{B} \leq \abs{A}$ so that there is an injective $f: B \to A$.
        If $B$ is empty then clearly it has to be that $\abs{B} = \abs{\es} = 0 \leq \abs{A}$ regardless of $A$.
        So we can assume that $B$ is nonempty so that there is a $b \in B$.
        Since $f$ is injective, the inverse $\inv{f}$ is a function from $\ran(f)$ onto $B$.
        Now we construct a function $g: A \to B$ by setting
        $$
        g(x) = \begin{cases}
             \inv{f}(x) & x \in \ran(f) \\
             b & x \notin \ran(f)
        \end{cases}
        $$
        for any $x \in A$.
        It should be clear that $g$ maps $A$ onto $B$ since, for any $b \in B$, we have $f(b) \in \ran(f)$ (hence also $f(b) \in A$) so that $g(f(b)) = \inv{f}(f(b)) = b$.
        \qedsymbol
    }

    \emph{Main Problem.}

    First we note that presumably $\hs(A)$ for a set $A$ is the least \emph{nonzero} ordinal $\a$ such that there is no function from $A$ onto $\a$.
    The fact that $\hs(A)$ is nonzero is important since, for a non-empty set $A$, there is no function from $A$ onto $\es = 0$ so that $0$ is actually the least ordinal such that such a function does not exist!
    We also make note of the fact that, even for $A=\es$, the empty function $f = \es$ is a vacuously a function from $A$ onto $\es = 0$ so that $\hs(A) = 1$ anyway since there is no function from $A = \es$ onto $1 = \braces{0}$.

	(a) Suppose to the contrary that there \emph{is} a function $f$ from $A$ onto $\a$.
    Then since $0 < \hs(A) \leq \a$ clearly $0 \in \hs(A)$ and $\hs(A) \ss \a$.
    So we define a function $g : A \to \hs(A)$ by
    $$
    g(a) = \begin{cases}
         f(a) & f(a) \in \hs(A) \\
         0 & f(a) \notin \hs(A)
    \end{cases} \,.
    $$
    for any $a \in A$.
    Clearly each $g(a) \in \hs(A)$ but we also claim that $g$ is onto.
    To this end consider any $\b \in \hs(A)$.
    Since $\hs(A) \ss \a$ we have that $\b \in \a$ also.
    Then, since $f$ is onto $\a$ there is an $a \in A$ such that $f(a) = \b$.
    Since $f(a) = \b \in \hs(A)$ it follows by definition that $g(a) = f(a) = \b$.
    Since $\b$ was arbitrary this shows that $g$ is onto.
    However, the existence of $g$  contradicts the definition of $\hs(A)$ so that it must be that there is no such function from $A$ onto $\a$. \qedsymbol

    (b) Suppose to the contrary that $\hs(A)$ is \emph{not} an initial ordinal so that there is an $\a < \hs(A)$ such that $\abs{\a} = \abs{\hs(A)}$.
    Let $f$ then be a bijection from $\a$ onto $\hs(A)$.
    Also since $\a < \hs(A)$ it follows from the definition of $\hs(A)$ that there is a function $g$ from $A$ onto $\a$ (since otherwise $\hs(A)$ would not be the \emph{least} such ordinal for which such a function does not exist).
    But then $f \circ g$ is a function from $A$ onto $\hs(A)$, which contradicts the definition of $\hs(A)$.
    Hence it has to be that $\hs(A)$ is in fact an initial ordinal.

    (c) Suppose to the contrary that $h(A) > \hs(A)$.
    Then by the definition of $h(A)$ there is a subset $X \ss A$ such that $\hs(A)$ is equipotent to $X$.
    So let $f$ be a bijection from $\hs(A)$ to $X$.
    We then define a function $g : A \to \hs(A)$ by
    $$
    g(a) = \begin{cases}
         \inv{f}(a) & a \in X \\
         0 & a \notin X
    \end{cases}
    $$
    for any $a \in A$, noting that $0 < \hs(A)$ so that $0 \in \hs(A)$.
    Clearly $g$ is into $\hs(A)$ but we also claim that it is onto.
    So consider any $\a \in \hs(A)$ and let $a = f(\a)$ so that $a \in X$ and therefore $a \in A$ since $X \ss A$.
    We then have $g(a) = \inv{f}(a) = \inv{f}(f(\a)) = \a$ since $a \in X$.
    Since $\a$ was arbitrary this shows that $g$ is onto.
    However, the existence of $g$ contradicts the definition of $\hs(A)$ so that it must be that in fact $h(A) \leq \hs(A)$ as desired. \qedsymbol

    (d) We show that $\hs(A) \leq h(A)$, from which the result clearly follows since also $\hs(A) \geq h(A)$ by part (c).
    So suppose to the contrary that $\hs(A) > h(A)$.
    Then by the definition of $\hs(A)$ it follows that there is a function from $A$ onto $h(A)$.
    It then  follows from Lemma~\ex.\ref{lem:hs:onto} that $\abs{h(A)} \leq \abs{A}$ since $A$ is well-orderable.
    However, this contradicts Lemma~7.1.4.\ref{lem:lthart:lthart} so that it must be that $\hs(A) \leq h(A)$ so that the result follows.

    (e) Consider any set $A$ and let $S$ denote the set of well-orderings of some partition of $A$ into equivalence classes, noting that it could be that $S = \es$.
    Since each $R \in S$ is isomorphic to a unique ordinal, let $H$ be the set of ordinals that are isomorphic to some $R \in S$, which exists by the Axiom Schema of Replacement.
    Then let $\a = \braces{0} \cup H$ and we claim that $\a = \hs(A)$.

    First we show that $\a$ is indeed an ordinal number.
    Since $\a$ is a set of ordinals clearly it is well-ordered by Theorem~6.2.6d.
    We also must show that $\a$ is transitive, so consider any $\b \in \a$.
    Then either $\b = 0$ or $\b \in H$.
    If $\b = 0 = \es$ then clearly $\b \ss \a$.
    On the other hand if $\b \in H$ then there is a partition $P$ of $A$ and a well-ordering $R$ of $P$ such that $(P, R)$ is isomorphic to $(\b, <)$.
    Now consider any $\g \in \b$ so that $\g < \b$.
    It then follows that $\g$ is isomorphic to an intial segment of $\b$ and therefore also to an initial segment $P'$ of $P$ ordered by $R$.
    Let $L$ be the least element of $P$ (which is also the least element of $P'$), which exists since $R$ is a well-ordering.
    Then let
    $$
    L' = L \cup \parens{A - \bigcup P'} \,,
    $$
    i.e. $L'$ is the set containing the elements of $L$ and any elements of $A$ that are not covered in the initial segment $P'$.
    Then let
    $$
    P'' = \braces{L'} \cup \parens{P' - \braces{L}} \,,
    $$
    i.e. $P''$ is $P'$ but with $L$ replaced with $L'$.
    It is easy to show that $P''$ is a partition of $A$ and that it is isomorphic to $\g$ with the same ordering as $R$ except with $L$ replaced by $L'$.
    Hence by definition we have that $\g \in \a$.
    Since $\g \in \b$ was arbitrary this shows that $\b \ss \a$, and since $\b \in \a$ was arbitrary this shows that $\a$ is transitive and hence an ordinal number.

    Now we show that there is no function from $A$ onto $\a$.
    So suppose to the contrary that there is such a function $f$.
    We then define the set
    $$
    E = \braces{(a,b) \in A \times A \where f(a) = f(b)} \,.
    $$
    It is trivial to show that this is an equivalence relation on $A$ so that $A/E$ is a partition of $A$ by Theorem~4.4.7.
    Moreover let $g$ be the mapping from $A/E$ to $\a$ defined as follows: for any $B \in A/E$ let $g(B)$ be the least element of $\braces{f(x) \where x \in B}$, noting that $\braces{f(x) \where x \in B}$ contains only a single element since $B$ is an equivalence class where $f(x) = f(y)$ for any $x$ and $y$ in $B$.
    It is trivial to show that $g$ is a bijective function so that we can well-order $A/E$ according to the ordinal $\a$ since $\a$ is the range of $g$.
    However, it then follows by definition that $\a \in \a$, which contradicts Lemma~6.2.7.
    Hence it must be that there is no function from $A$ onto $\a$.

    Lastly we show that there is a function from $A$ onto $\b$ for every nonzero $\b < \a$.
    So consider any such $\b$ so that $\b \in \a$.
    Then either $\b = 0$ or $\b \in H$, but since $\b$ is nonzero it must be that $\b \in H$.
    Then by definition there is is a partition $P$ of $A$ and well-ordering $R$ of $P$ such that $(P,R)$ is isomorphic to $(\b,<)$.
    Let $f$ then be the isomorphism from $P$ to $\b$.
    We then define the mapping $g : A \to \b$ as follows: for any $a \in A$ there is a unique $B \in P$ such that $a \in B$ since $P$ is a partition of $A$.
    We then set $g(a) = f(B)$.
    It is easy to show that $g$ is onto.

    It follows from what has been shown that indeed $\a = \hs(A)$. \qedsymbol
\end{solution}

\def\ex{7.2.1}
\setcounter{itm}{0}
\question{\ex}
\begin{solution}
    \begin{statement}{Lemma~\ex.\itm{lem:alephadd:ordgt}}
        If $\a$ and $\b$ are ordinals and $\a > \b$ then there is a $\g  <\a$ such that $\abs{\b} \leq \abs{\g}$.
    \end{statement}

    \proof{
        Clearly for $\g = \b$ we have $\g = \b < \a$ and $\abs{\b} = \abs{\g}$ so that $\abs{\b} \leq \abs{\g}$ is true. \qedsymbol
    }

    \begin{statement}{Corollary~\ex.\itm{cor:alephadd:setgt}}
        If a set $(A, \prec)$ is isomorphic to ordinal $\a$ and $\a > \b$ for another ordinal $\b$ then there is an $a \in A$ such that $\abs{\b} \leq \abs{X}$ for the set
        $$
        X = \braces{x \in A \where x \prec a} \,.
        $$
    \end{statement}
    \proof{
        First let $f$ be the isomorphism from $\a$ to $A$.
        Clearly by Lemma~\ex.\ref{lem:alephadd:ordgt} there is an ordinal $\g < \a$ such that $\abs{\b} \leq \abs{\g}$.
        Now let $a = f(\g)$ so that clearly $a \in A$, and let
        $$
        X = \braces{x \in A \where x \prec a} \,.
        $$
        Now, we claim that $X = f[\g]$.
        So consider any $x \in X$ so that $x \prec a$.
        It then follows that $\inv{f}(x) < \inv{f}(a) = \g$ since $\inv{f}$ is an isomorphism since $f$ is.
        Hence $\inv{f}(x) \in \g$ so that clearly $x = f(\inv{f}(x)) \in f[\g]$.
        Since $x$ was arbitrary it follows that $X \ss f[\g]$.
        Now consider any $x \in f[\g]$ so that there is a $\d \in \g$ such that $x = f(\d)$.
        Then $\d < \g$ so that $x = f(\d) \prec f(\g) = a$ since $f$ is an isomorphism so that by definition $x \in X$.
        Hence $f[\g] \ss X$.
        This shows that $X = f[\g]$.
        Then, since $f$ is bijective, it follows that $\abs{\b} \leq \abs{\g} = \abs{f[\g]} = \abs{X}$. \qedsymbol
    }

    \begin{statement}{Lemma~\ex.\itm{lem:alephadd:alephlt}}
        For ordinal $\a > 0$ and an ordinal $\b < \w_\a$ there is an ordinal $\g < \a$ such that $\abs{\b} \leq \al_\g$.
    \end{statement}
    \proof{
        First, if $\b$ is finite then clearly $\b < \w$ so that $\b \in \w$.
        Then $\b \ss \w$ since $\w$ is transitive (since it is an ordinal number).
        Hence $\abs{\b} \leq \abs{\w} = \al_0$ (i.e. $\g = 0$ so that $\g < \a$).
        On the other hand if $\b$ is infinite then by Theorem~7.1.3 $\b$ is equipotent to some initial ordinal $\W$.
        Clearly $\W$ is infinite since $\b$ is and clearly $\W < \w_\a$ since $\b < \w_\a$ and $\w_\a$ is an initial ordinal.
        It then follows from Lemma~7.1.9.4.T that there is a $\g < \a$ such that $\W = \w_\g$.
        Then we have $\abs{\b} = \abs{\W} = \abs{\w_\g} = \al_\g$ so that $\abs{\b} \leq \al_\g$ is true. \qedsymbol
    }

    \emph{Main Problem.}

    This proof is similar to the proof of Theorem~7.2.1.

	Suppose that $A_1$ and $A_2$ are disjoint sets that are both equipotent to $\w_\a$ for some ordinal $\a$.
    Then there are bijections $f_1$ and $f_2$ from $A_1$ and $A_2$, respectively, to $\w_\a$.
    We define a well-ordering $\prec$ of $A = A_1 \cup A_2$ as follows: for $a$ and $b$ in $A$ we let $a \prec b$ if and only if
    \begin{itemize}
        \item $a$ and $b$ are in $A_1$ and $f_1(a) < f_1(b)$, or
        \item $a$ and $b$ are in $A_2$ and $f_2(a) < f_2(b)$, or
        \item $a \in A_1$ and $b \in A_2$ and $f_1(a) \leq f_2(b)$, or
        \item $a \in A_2$ and $b \in A_1$ and $f_2(a) < f_1(b)$.
    \end{itemize}

    First we show that $\prec$ is transitive.
    So consider $a$, $b$, and $c$ in $A$ such that $a \prec b$ and $b \prec c$.

    Case: $a \in A_1$
    \begin{indpar}
        Case: $b \in A_1$
        \begin{indpar}
            Case: $c \in A_1$.
            Then $f_1(a) < f_1(b) < f_1(c)$ so that $f_1(a) < f_1(c)$ and hence $a \prec c$.

            Case: $c \in A_2$.
            Then $f_1(a) < f_1(b) \leq f_2(c)$ so that $f_1(a) \leq f_2(c)$ is true and hence $a \prec c$.
        \end{indpar}

        Case: $b \in A_2$
        \begin{indpar}
            Case: $c \in A_1$.
            Then $f_1(a) \leq f_2(b) < f_1(c)$ so that $f_1(a) < f_1(c)$ and hence $a \prec c$.

            Case: $c \in A_2$.
            Then $f_1(a) \leq f_2(b) < f_2(c)$ so that $f_1(a) \leq f_2(c)$ is true and hence $a \prec c$.
        \end{indpar}
    \end{indpar}

    Case: $a \in A_2$
    \begin{indpar}
        Case: $b \in A_1$
        \begin{indpar}
            Case: $c \in A_1$.
            Then $f_2(a) < f_1(b) < f_1(c)$ so that $f_2(a) < f_1(c)$ and hence $a \prec c$.

            Case: $c \in A_2$.
            Then $f_2(a) < f_1(b) \leq f_2(c)$ so that $f_2(a) < f_2(c)$ and hence $a \prec c$.
        \end{indpar}

        Case: $b \in A_2$
        \begin{indpar}
            Case: $c \in A_1$.
            Then $f_2(a) < f_2(b) < f_1(c)$ so that $f_2(a) < f_1(c)$ and hence $a \prec c$.

            Case: $c \in A_2$.
            Then $f_2(a) < f_2(b) < f_2(c)$ so that $f_2(a) < f_2(c)$ and hence $a \prec c$.
        \end{indpar}
    \end{indpar}
    Since all cases imply that $a \prec c$ and $a$, $b$, and $c$ were arbitrary this shows that $\prec$ is transitive.

    Now we show that for any $a$ and $b$ in $A$, that either $a \prec b$, $a = b$, or $b \prec a$ and that only one of these is true.
    So consider any $a$ and $b$ in $A$.
    Then we have

    Case: $a \in A_1$
    \begin{indpar}
        Case: $b \in A_1$.
        Then clearly exactly one of the following is true: $a \prec b$ if $f_1(a) < f_1(b)$, $a = b$ if $f_1(a) = f_1(b)$ since $f_1$ is a bijection, and $b \prec a$ if $f_1(b) < f_1(a)$.

        Case: $b \in A_2$.
        Clearly $a = b$ is not possible since $A_1$ and $A_2$ are disjoint.
        Then if $f_1(a) \leq f_2(b)$ then $a \prec b$ and if $f_1(a) > f_2(b)$ then $b \prec a$, noting that these are mutually exclusive.
    \end{indpar}

    Case: $a \in A_2$
    \begin{indpar}
        Case: $b \in A_1$.
        Then again clearly $a = b$ is not possible since $A_1$ and $A_2$ are disjoint.
        Then if $f_2(a) < f_1(b)$ then $a \prec b$ and if $f_2(a) \geq f_1(b)$ then $b \prec a$, noting that these are mutually exclusive.

        Case: $b \in A_2$.
        Then clearly exactly one of the following is true: $a \prec b$ if $f_2(a) < f_2(b)$, $a = b$ if $f_2(a) = f_2(b)$ since $f_2$ is a bijection, and $b \prec a$ if $f_2(b) < f_2(a)$.
    \end{indpar}
    Thus we have shown that $\prec$ is a total (strict) order on $A$.

    Now we show that $\prec$ is also a well-ordering.
    So let $X$ be a nonempty subset of $A$.
    Let $B = f_1[X \cap A_1] \cup f_2[X \cap A_2]$, noting that this is a nonempty set of ordinals.
    Then let $B$ has a least element $\a$.

    Case: $\a \in f_1[X \cap A_1]$.
    Then we claim that $x = \inv{f_1}(\a)$ is the $\prec$-least element of $X$, noting that $x \in A_1$.
    Note also that clearly then $f_1(x) = f_1(\inv{f_1}(\a)) = \a$.
    So consider any $y \in X$.
    \begin{indpar}
        Case: $y \in A_1$.
        Then $y \in X \cap A_1$ so that $f_1(y) \in f_1[X \cap A_1]$ so $f_1(y) \in B$.
        Thus $f_1(x) = \a \leq f_1(y)$ since $\a$ is the least element of $B$.
        Clearly if $f_1(x) = f_1(y)$ then $x = y$ since $f_1$ is bijective.
        On the other hand if $f_1(x) < f_1(y)$ then by definition $x \prec y$.
        Hence in either case we have $x \preceq y$.

        Case: $y \in A_2$.
        Then $y \in X \cap A_2$ so that $f_2(y) \in f_2[X \cap A_2]$ so that $f_2(y) \in B$.
        Thus $f_1(x) = \a \leq f_2(y)$ so that by definition $x \prec y$.
        Hence again $x \preceq y$ is true.
    \end{indpar}

    Case: $\a \notin f_1[X \cap A_1]$.
    Then it has to be that $\a \in f_2[X \cap A_2]$.
    Then we claim that $x = \inv{f_2}(\a)$ is the $\prec$-least element of $X$, noting that $x \in A_2$.
    Note also that clearly then $f_2(x) = f_2(\inv{f_2}(\a)) = \a$.
    So consider any $y \in X$.
    \begin{indpar}
        Case: $y \in A_1$.
        Then $y \in X \cap A_1$ so that $f_1(y) \in f_1[X \cap A_1]$ so $f_1(y) \in B$.
        Thus $f_2(x) = \a \leq f_1(y)$ since $\a$ is the least element of $B$.
        Now, it cannot be that $f_2(x) = \a = f_1(y)$ for then $\a$ would be in $f_1[X \cap A_1]$.
        So it must be that $f_2(x) = \a < f_1(y)$ so that by definition $x \prec y$ so that $x \preceq y$ is true.

        Case: $y \in A_2$.
        Then $y \in X \cap A_2$ so that $f_2(y) \in f_2[X \cap A_2]$ so that $f_2(y) \in B$.
        Thus $f_2(x) = \a \leq f_2(y)$.
        Then if $f_2(x) = \a = f_2(y)$ then $x = y$ since $f_2$ is bijective.
        On the other hand if $f_2(x) = \a < f_2(y)$ then $x \prec y$ by definition.
        In either case we have $x \preceq y$.
    \end{indpar}
    Hence in all cases we have shown that $X$ has a $\prec$-least element so that $\prec$ is a well-ordering of $A$.

    Now we show by transfinite induction that $\al_\a + \al_\a = \al_\a$ for all ordinals $\a$.
    First it was already shown in a previous chapter that $\al_0 + \al_0 = \al_0$.
    So now consider any $\a > 0$ and suppose $\al_\g + \al_\g = \al_\g$ for all $\g < \a$.

    Then consider two disjoint sets $A_1$ and $A_2$ that are both equipotent to $\w_\a$ and the well-ordering $\prec$ on $A = A_1 \cup A_2$ as defined above, also again letting $f_1$ and $f_2$ be the isomophisms from $A_1$ and $A_2$, respectively, to $\w_\a$.
    Now let $a$ be any element of $A$ and define
    $$
    X = \braces{x \in A \where x \prec a} \,.
    $$
    Let $X_1 = X \cap A_1$ and $X_2 = X \cap A_2$ so that clearly $X_1$ and $X_2$ are disjoint and $X = X_1 \cup X_2$.
    From this it follows from the definition of cardinal addition that $\abs{X} = \abs{X_1} + \abs{X_2}$.

    If $a \in A_1$ then define $\b = f_1(a) \in \w_\a$ so that $\b < \w_\a$.
    It follows from this and Lemma~\ex.\ref{lem:alephadd:alephlt} that there is a $\g < \a$ such that $\abs{\b} \leq \al_\g$ since $\a > 0$, noting also that $\al_\g < \al_\a$ by the remarks following Definition~7.1.8.
    Now consider any $x_1 \in X_1$ so that also $x_1 \in X$.
    Then by definition $x_1 \prec a$ so that by the definition of $\prec$ we have $f_1(x_1) < f_1(a) = \b$ since $x_1 \in A_1$ and $a \in A_1$.
    Hence $f_1(x_1) \in \b$ so that $f_1[X_1] \ss \b$ since $x_1$ was arbitrary.
    Hence, since $f_1$ is bijective, we have $\abs{X_1} = \abs{f_1[X_1]} \leq \abs{\b} \leq \al_\g$.
    Next, consider any $x_2 \in X_2$ so that $x_2 \in X$ and hence $x_2 \prec a$.
    Then, again by the definition of $\prec$, we have that $f_2(x_2) < f_1(a) = \b$ since $x_2 \in A_2$ and $a \in A_1$.
    Hence $f_2(x_2) \in \b$ so that $f_2[X_2] \ss \b$ since $x_2$ was arbitrary.
    Thus we have $\abs{X_2} = \abs{f_2[X_2]} \leq \abs{\b} \leq \al_\g$ since $f_2$ is bijective.

    A similar argument shows that $\abs{X_1} \leq \al_\g$ and $\abs{X_2} \leq \al_\g$ for some $\g < \a$ in the case when $a \in A_2$.
    However, in this case we must set $\b = f_2(a) + 1$, noting that $\b \in \w_\a$ since $f_2(a) \in \w_\a$ and $\w_\a$ is a limit ordinal by Lemma~7.1.9.1.T.

    Thus in all cases we have
    \ali{
        \abs{X} &= \abs{X_1} + \abs{X_2} \\
        &\leq \al_\g + \al_\g & \text{(by property (d) of cardinal numbers in section~5.1)} \\
        &= \al_\g & \text{(by the induction hypothesis since $\g < \a$)} \\
        &< \al_\a \,.
    }
    Thus we have shown that $\abs{X} < \al_\a = \abs{\w_\a}$ for any $a \in A$, and hence $\abs{\w_\a} \not\leq \abs{X}$ by Corollary~7.1.8.2.T since $\w_\a$ and $X$ are both well-ordered.
    If $\d$ is the ordinal isomorphic to $(A, \prec)$ (which exists by Theorem~6.3.1 since we have shown that $\prec$ is a well-ordering), then it follows from the contrapositive of Lemma~\ex.\ref{cor:alephadd:setgt} that $\d \leq \w_\a$ and hence $\abs{A} = \abs{\d} \leq \abs{\w_\a} = \al_\a$.
    Thus we have
    $$
    \al_\a + \al_\a = \abs{A_1} + \abs{A_2} = \abs{A} \leq \al_\a \,.
    $$
    Since obviously $0 \leq \al_\a$ it follows again from property (d) in section~5.1 that
    $$
    \al_\a = \al_\a + 0 \leq \al_\a + \al_\a \,.
    $$
    Hence, by the Cantor-Bernstein Theorem we have that $\al_\a = \al_\a + \al_\a$, which completes the inductive step. \qedsymbol
\end{solution}

\def\ex{7.2.2}
\setcounter{itm}{0}
\question{\ex}
\begin{solution}
    \begin{statement}{Lemma~\ex.\itm{lem:natal:natlim}}
        If $\a$ is a limit ordinal and $n$ is a natural number then $n \cdot \a = \a$.
    \end{statement}
    \proof{
        First, clearly we have by definition that $n \cdot \w = \sup \braces{n \cdot k \where k < \w} = \w$.
        Then, since $\a$ is a limit ordinal, we have from Exercise~6.5.10 that $\a = \w \cdot \b$ for some ordinal $\beta$.
        Thus we have
        $$
        n \cdot \a = n \cdot (\w \cdot \b) = (n \cdot \w) \cdot \b = \w \cdot \b = \a \,.
        $$
    }

    \emph{Main Problem.}

    Consider any natural number $n > 0$ and any ordinal number $\a$.
    We then show that $n \cdot \al_\a = \al_\a$ by constructing a bijective $f: \w_\a \to n \times \w_\a$, which clearly shows the result by the definition of cardinal multiplication.

    So consider any $\b \in \w_\a$.
    Then, since $n > 0$, we have by Theorem~6.6.3 that there is a unique ordinal $\g$ and unique natural number $k < n$ such that
    $$
    \b = n \cdot \g + k \,,
    $$
    where $k < n$ and clearly
    $$
    \g = 1 \cdot \g \leq n \cdot \g = n \cdot \g + 0 \leq n \cdot \g + k = \b
    $$
    since $1 \leq n$ and $0 \leq k$.
    Hence we have $k \in n$ and $\g \leq \b < \w_\a$ so that $\g \in \w_\a$.
    We then set $f(\b) = (k, \g) \in n \times \w_\a$.

    First we show that $f$ is injective.
    So consider $\b_1$ and $\b_2$ in $\w_\a$ where $f(\b_1) = f(\b_2)$.
    If we have $f(\b_1) = (k_1, \g_1)$ and $f(\b_2) = (k_2, \g_2)$ then clearly this means that $k_1 = k_2$ and $\g_1 = \g_2$.
    It then clearly follows that
    $$
    \b_1 = n \cdot \g_1 + k_1 = n \cdot \g_2 + k_2 = \b_2 \,,
    $$
    which shows that $f$ is injective.

    Now we show that $f$ is also surjective.
    So consider any $(k, \g) \in n \times \w_\a$ so that $k \in n$ and $\g \in \w_\a$.
    Then let $\b = n \cdot \g + k$ so that clearly $f(\b) = (k, \g)$.
    However, we must show that $\b$ is actually in $\w_\a$.
    To see this, first we note that since $\g < \w_\a$ is an ordinal we have $\g = \d + m$ for some limit ordinal $\d$ and natural number $m$ by Exercise~6.5.4, where clearly $\d \leq \g$.
    Hence we have
    $$
    \b = n \cdot \g + k = n \cdot (\d + m) + k = (n \cdot \d + n \cdot m) + k = n \cdot \d + (n \cdot m + k) = \d + (n \cdot m + k) \,,
    $$
    where $n \cdot \d = \d$ by Lemma~\ex.\ref{lem:natal:natlim} since $\d$ is a limit ordinal.
    Then since also $\d \leq \g < \w_\a$ and $n \cdot m + k$ is a natural number we clearly have that $\b = \d + (n \cdot m + k)< \w_\a$ as well since $\w_\a$ is a limit ordinal (by Theorem~7.1.9b).
    Hence $\b \in \w_\a$.

    Thus we have shown that $f$ is bijective so that by definition $n \cdot \al_\a = \abs{n \times \w_\a} = \abs{\w_\a} = \al_\a$ as desired. \qedsymbol
\end{solution}

\question{7.2.3}
\begin{solution}
    \newcommand\nss[1]{\squares{\al_\a}^{#1}}
    \def\wss{\squares{\al_\a}^{<\w}}
	(a) For any ordinal $\a$, we show this by induction on $n$, noting that we only need to show this for positive $n$ so that $n \geq 1$ (in fact it is untrue for $n=0$).
    First, for $n=1$ we clearly have $\al_\a^n = \al_\a^1 = \al_\a$ by what was shown in Exercise~5.1.2.
    Now assume that $\al_\a^n = \al_\a$.
    We then have
    \ali{
        \al_\a^{n+1} &= \al_\a^n \cdot \al_\a^1 & \text{(by Theorem~5.1.7a)} \\
        &= \al_\a^n \cdot \al_\a & \text{(again by Exercise~5.1.2)} \\
        &= \al_\a \cdot \al_\a & \text{(by the induction hypothesis)} \\
        &= \al_\a \,. & \text{(by Theorem~7.2.1)}
    }
    This completes the induction step. \qedsymbol

    (b) For ordinal $\a$ and natural number $n$, first we show that $\abs{\nss{n}} \leq \al_\a$ by constructing an injective $f : \nss{n} \to \w_\a^n$.
    For a set $X \in \nss{n}$ we have that $\abs{X} = n$.
    Thus there is a bijective $g$ from $n$ to $X$, and since clearly $X \ss \al_\a = \w_\a$ it follows that $g$ is a function from $n$ to $\w_\a$.
    Hence we simply set $f(X) = g$.
    Now consider any $X_1$ and $X_2$ in $\nss{n}$ where $X_1 \neq X_2$.
    Then let $g_1$ and $g_2$ be the corresponding bijections from $n$ to $X_1$ and $X_2$, respectively.
    Thus $f(X_1) = g_1$ and $f(X_2) = g_2$.
    Then, since clearly the range of $g_1$ is $X_1$, the range of $g_2$ is $X_2$, and $X_1 \neq X_2$ it follows that $f(X_1) = g_1 \neq g_2 = f(X_2)$, which shows that $f$ is injective.
    Hence it follows that $\abs{\nss{n}} \leq \abs{\w_\a^n} = \abs{\w_\a}^{\abs{n}} = \al_\a^n = \al_\a$ by what was shown in part (a).

    Now we show that also $\al_\a \leq \abs{\nss{n}}$ by constructing an injective $f : \w_\a \to \nss{n}$.
    So for any $\b \in \w_\a$ let $X = \braces{\b + k \where k \in n}$, noting that clearly $X \ss \w_\a = \al_\a$ since $\w_\a$ is a limit ordinal (since then $\b + k \in \w_\a$ for any natural number $k$).
    Also clearly $\abs{X} = n$ so that $X \in \nss{n}$.
    We then set $f(\b) = X$.
    Now consider any $\b_1$ and $\b_2$ in $\w_\a$ where $\b_1 \neq \b_2$ and let $X_1 = \braces{\b_1 + k \where k \in n}$ and $X_2 = \braces{\b_2 + k \where k \in n}$ so that $f(\b_1) = X_1$ and $f(\b_2) = X_2$.
    Since $\b_1 \neq \b_2$ we can assume that $\b_1 < \b_2$ without loss of generality.
    Now, clearly $\b_1 = \b_1 + 0$ is the least element of $X_1$ and $\b_2 = \b_2 + 0$ the least element of $X_2$.
    Since $\b_1 < \b_2$ it then follows that $\b_1 \notin X_2$, but since $\b_1 \in X_1$ this clearly implies that $f(\b_1) = X_1 \neq X_2 = f(\b_2)$.
    This shows that $f$ is injective so that $\al_\a = \abs{\w_\a} \leq \abs{\nss{n}}$.

    Since we have shown that both $\abs{\nss{n}} \leq \al_\a$ and $\al_\a \leq \abs{\nss{n}}$, it follows from the Cantor-Bernstein Theorem that $\abs{\nss{n}} = \al_\a$, which is what we intended to show. \qedsymbol

    (c) For any ordinal $\a$ we show first note that clearly $\wss = \bigcup_{n < \w} \nss{n}$.
    We show that $\abs{\wss} = \al_\a$ by constructing a bijective $f : \w \times \w_\a \to \bigcup_{n < \w} \nss{n}$.
    So consider any $n \in \w$ and $\b \in \w_\a$.
    Now, by what was shown in part (b), we have $\abs{\nss{n}} = \al_\a = \abs{\w_\a}$ so that there is a bijective $g_n : \w_\a \to \nss{n}$, i.e. $g_n$ is a transfinite enumeration of $\nss{n}$.
    We then set $f(n, \b) = g_n(\b)$, from which it should be clear that $f(n,\b) \in \bigcup_{k < \w} \nss{k}$ since $f(n, \b) = g_n(\b) \in \nss{n}$.

    First we show that $f$ is injective.
    To this end consider any $(n_1, \b_1)$ and $(n_2, \b_2)$ in $\w \times \w_\a$ where $(n_1, \b_1) \neq (n_2, \b_2)$.
    Then either $n_1 \neq n_2$ or $\b_1 \neq \b_2$ (or both).
    Let $g_{n_1}$ and $g_{n_2}$ be the corresponding bijections from $\w_\a$ to $\nss{n_1}$ and $\nss{n_2}$, respectively, as described above.
    Clearly if $n_1 \neq n_2$ then $f(n_1, \b_1) = g_{n_1}(\b_1) \in \nss{n_1}$ whereas $f(n_2, \b_2) = g_{n_2}(\b_2) \in \nss{n_2}$ so that $f(n_1, \b_1) \neq f(n_2, \b_2)$ since $\nss{n_1}$ and $\nss{n_2}$ are clearly disjoint (since $\nss{n_1}$ contains only sets with $n_1$ elements and $\nss{n_2}$ contains only sets with $n_2$ elements and $n_1 \neq n_2$).
    On the other hand, if $n_1 = n_2$ then it must be the case that $\b_1 \neq \b_2$.
    It also follows that $g_{n_1} = g_{n_2}$ since $n_1 = n_2$.
    Hence we have $f(n_1, \b_1) = g_{n_1}(\b_1) \neq g_{n_1}(\b_2) = g_{n_2}(\b_2) = f(n_2, \b_2)$ since $g_{n_1} = g_{n_2}$ is injective and $\b_1 \neq \b_2$.
    Thus in any case $f(n_1, \b_1) \neq f(n_2, \b_2)$ so that $f$ is injective.

    Next we show that $f$ is surjective.
    So consider any $X \in \bigcup_{n < \w} \nss{n}$ so that there is an $n < \w$ such that $X \in \nss{n}$.
    Then let $\b = \inv{g_n}(X)$ (where $g_n$ is the bijection from $\w_\a$ to $\nss{n}$ as described above).
    Then clearly $(n, \b) \in \w \times \w_\a$ and we have $f(n, \b) = g_n(\b) = g_n(\inv{g_n}(X)) = X$.
    Since $X$ was arbitrary this shows that $f$ is surjective.

    Hence $f$ is a bijection so that $\abs{\wss} = \abs{\bigcup_{n < \w} \nss{n}} = \abs{\w \times \w_\a} = \abs{\w} \cdot \abs{\w_\a} = \al_0 \cdot \al_\a = \al_\a$ by Corollary~7.2.2 since clearly $0 \leq \a$.
    This shows the desired result. \qedsymbol
\end{solution}

\def\maxab{\max\parens{\abs{\a}, \abs{\b}}}
\def\maxag{\max\parens{\abs{\a}, \abs{\g}}}
\def\ex{7.2.4}
\setcounter{itm}{0}
\question{\ex}
\begin{solution}
    \begin{statement}{Lemma~\ex.\itm{lem:orcarar:add}}
        If $\a$ and $\b$ are ordinals then $\abs{\a + \b} = \abs{\a} + \abs{\b}$.
    \end{statement}
    \proof{
        Suppose that $A$ and $B$ are disjoint sets where $\abs{A} = \abs{\a}$ and $\abs{B} = \abs{\b}$ so that, by the definition of cardinal addition, $\abs{\a} + \abs{\b} = \abs{A \cup B}$.
        We then show the result by constructing a bijection $f$ from $A \cup B$ to the ordinal $\a + \b$.
        First, since $\abs{A} = \abs{\a}$, there is a bijective $f_A : A \to \a$.
        Similarly there is a bijective $f_B : B \to \b$ since $\abs{B} = \abs{\b}$.
        Now consider any $x \in A \cup B$.
        We then set
        $$
        f(x) = \begin{cases}
             f_A(x) & x \in A \\
             \a + f_B(x) & x \in B
        \end{cases} \,,
        $$
        noting that this is unambiguous since $A$ and $B$ are disjoint.
        If $x \in A$ then we clearly have $f(x) = f_A(x) < \a = \a + 0 \leq \a + \b$ by Lemma~6.5.4 so that $f(x) \in \a + \b$.
        On the other hand if $x \in B$ then $f(x) = \a + f_B(x) < \a + \b$ again by Lemma~6.5.4 since $f_B(x) < \b$, and hence again $f(x) \in \a + \b$.
        This shows that $f$ really is a function into $\a + \b$.

        Next we show that $f$ is injective.
        So consider any $x$ and $y$ in $A \cup B$ where $x \neq y$.

        Case: $x$ and $y$ are both in $A$.
        Then $f(x) = f_A(x) \neq f_A(y) = f(y)$ since $f_A$ is injective and $x \neq y$.

        Case: $x \in A$ and $y \in B$.
        Then $f(x) = f_A(x) < \a = \a + 0 \leq \a + f_B(y) = f(y)$ by Lemma~6.5.4 so that clearly $f(x) \neq f(y)$.

        Case: $x \in B$ and $y \in A$.
        This is analagous to the previous case.

        Case: $x$ and $y$ are both in $A$.
        Then $f(x) = \a + f_B(x) \neq \a + f_B(y) = f(y)$ by Lemma~6.5.4b since $f_B$ is injective and $x \neq y$ so that $f_B(x) \neq f_B(y)$.

        Hence in all cases we have $f(x) \neq f(y)$, which shows that $f$ is injective.

        Lastly, we show that $f$ is surjective.
        So consider any $y$ in $\a + \b$.
        If $y < \a$ then $y \in \a$.
        Since $f_A$ is surjective there is an $x \in A$ such that $f_A(x) = y$.
        Then, since $x \in A$, clearly $f(x) = f_A(x) = y$.
        If $y \geq a$ then by Lemma~6.5.5 there is an ordinal $\x$ such that $\a + \x = y$.
        It then follows that $\a + \x = y < \a + \b$ so that $\x < \b$ by Lemma~6.5.4a.
        Hence $\x \in \b$ so that there is an $x \in B$ such that $f_B(x) = \x$ since $f_B$ is surjective.
        Then, since $x \in B$,  clearly we have $f(x) = \a + f_B(x) = \a + \x = y$.
        Hence in both cases there is an $x \in A \cup B$ where $f(x) = y$.
        Since $y$ was arbitrary, this shows that $f$ is surjective.

        Thus we have shown that $f$ is bijective so that $\abs{\a} + \abs{\b} = \abs{A \cup B} = \abs{\a + \b}$. \qedsymbol
    }

    \begin{statement}{Lemma~\ex.\itm{lem:orcarar:mult}}
        If $\a$ and $\b$ are ordinals then $\abs{\a \cdot \b} = \abs{\a} \cdot \abs{\b}$.
    \end{statement}
    \proof{
        First, if $\a = 0$ then
        $$
        \abs{\a \cdot \b} = \abs{0 \cdot \b} = \abs{0} = 0 = 0 \cdot \abs{\b} = \abs{0} \cdot \abs{\b} = \abs{\a} \cdot \abs{\b} \,.
        $$
        The result also holds when $\b = 0$.
        Hence going forward we can assume that $\a$ and $\b$ are nonzero and therefore nonempty.

        We then show the result by constructing a bijective $f : \a \times \b \to \a \cdot \b$.
        So consider any $(\g, \d) \in \a \times \b$.
        It then follows that $\g < \a$ and $\d < \b$ so that $\d+1 \leq \b$.
        We then set $f(\g, \d) = \a \cdot \d + \g$.
        We note that
        $$
        f(\g, \d) = \a \cdot \d + \g < \a \cdot \d + \a = \a \cdot \d + \a \cdot 1 = \a \cdot (\d + 1) \leq \a \cdot \b
        $$
        by Lemma~6.5.4, Exercise~6.5.2, and Exercise~6.5.7.
        Hence $f(\g, \d) \in \a \cdot \b$ so that $f$ is into $\a \cdot \b$.

        Now we show that $f$ is injective.
        So consider any $(\g_1, \d_1)$ and $(\g_2, \d_2)$ in $\a \times \b$ where $(\g_1, \d_1) \neq (\g_2, \d_2)$.
        Then clearly we have $\g_1 < \a$, $\g_2 < \a$, $\d_1 < \b$, and $\d_2 < \b$.
        We then have

        Case: $\d_1 = \d_2$.
        Then it must be that $\g_1 \neq \g_2$.
        We then have
        $$
        f(\g_1, \d_1) = \a \cdot \d_1 + \g_1 \neq \a \cdot \d_1 + \g_2 = \a \cdot \d_2 + \g_2 = f(\g_2, \d_2)\,,
        $$
        where we have used Lemma~6.5.4b and Exercise~6.5.7b since $\a \neq 0$.
        
        Case: $\d_1 \neq \d_2$.
        Without loss of generality we can assume that $\d_1 < \d_2$ so that clearly $\d_1 + 1 \leq \d_2$.
        Then we have
        $$
        f(\g_1, \g_2) = \a \cdot \d_1 + \g_1 < \a \cdot \d_1 + \a = \a \cdot (\d_1 + 1) \leq \a \cdot \d_2 \leq \a \cdot \d_2 + \g_2 = f(\g_2, \d_2) \,,
        $$
        where we have again used Lemma~6.5.4a, Exercise~6.5.2, and Exercise~6.5.7.

        Hence in all cases we have $f(\g_1, \d_1) \neq f(\g_2, \d_2)$, which shows that $f$ is injective.

        Lastly, we show that $f$ is surjective.
        So consider any ordinal $\x \in \a \cdot \b$ so that $\x < \a \cdot \b$.
        Since $\a \neq 0$, it follows from Theorem~6.6.3 that there is a unique $\d$ and unique $\g < \a$ such that $\x = \a \cdot \d + \g$.
        Note that we have $\d < \b$ since otherwise $\d \geq \b$ would imply that
        $$
        \x = \a \cdot \d + \g \geq \a \cdot \d \geq \a \cdot \b
        $$
        by Lemma~6.5.4 and Exercise~6.5.7, which is impossible since $\x < \a \cdot \b$.
        Hence we have that $(\g, \d) \in \a \times \b$, and clearly $f(\g, \d) = \x$.
        Since $\x$ was arbitrary this shows that $f$ is surjective.

        Thus we have shown that $f$ is bijective so that by the definition of cardinal multiplication we have $\abs{\a \cdot \b} = \abs{\a \times \b} = \abs{\a} \cdot \abs{\b}$ as desired. \qedsymbol
     }

     The following lemmas are generalizations of the some of the theorems proved in section~4.3.

     \iffalse
    \begin{statement}{Lemma~\ex.\itm{lem:orcarar:range}}
        If $\angles{a_\g}_{\g < \a}$ is a transfinite sequence indexed by some ordinal $\a$ then the range of the sequence $A = \braces{a_\g}_{\g < \a}$ is at most $\abs{\a}$, i.e. $\abs{A} \leq \abs{\a}$.
    \end{statement}
    \proof{
        We construct an injective $f: A \to \a$.
        For any $a \in A$ consider the set $B = \braces{\g \in \a \where a_\g = a}$, noting that this is a nonempty set of ordinals since $A$ is the range of the sequence $\angles{a_\g}_{\g < \a}$.
        It then follows that $B$ has a unique least element $\b$, so we set $f(a) = \b$.

        To see that $f$ is injective, consider $a_1$ and $a_2$ in $A$ such that $f(a_1) = f(a_2)$.
        By the definition of $f$ we then have that $a_1 = a_{f(a_1)} = a_{f(a_2)} = a_2$.

        Hence, since $f$ is injective, we have by definition that $\abs{A} \leq \abs{\a}$. \qedsymbol
    }
    \fi

    \def\sun{\bigcup_{\b<\a} A_\b}
    \begin{statement}{Lemma~\ex.\itm{lem:orcarar:union}}
        Suppose $\g$ is an ordinal and $\angles{A_\b}_{\b < \a}$ is a system of sets indexed by nonzero ordinal $\a$ where $\abs{\a} \leq \al_\g$.
        Also suppose that each set in the system is well-orderable and at most $\al_\g$.
        Then $\sun$ is at most $\al_\g$ as well.
    \end{statement}
    \proof{
        First, suppose that every set in the system is empty.
        Then clearly $\sun$ is empty since, if there is an $x \in \sun$, then there is a $\b < \a$ where $x \in A_\b$, which contradicts the fact that $A_\b = \es$.
        Hence clearly $\abs{\sun} = \abs{\es} = 0 < \al_\g$.

        So we assume that at least one set in the system is non-empty and so that there exists an $a$ in that set.
        Now, consider any $\b < \a$.
        Then we clearly have  $\abs{A_\b} \leq \al_\g$.
        Also, since $A_\b$ is well-orderable, it follow that there is an ordinal $\a_\b$ such that the elements of $A_\b$ can be written as a transfinite sequence $\angles{a_\d}_{\d < \a_\b}$.
        It also follows that $\abs{\a_\b} \leq \al_\g$.
    }

    \begin{statement}{Lemma~\ex.\itm{lem:orcarar:exp}}
        If $\a$ and $\b$ are ordinals where at least one is infinite then $\abs{\a^\b} \leq \maxab$.
    \end{statement}
    \proof{
        Clearly $\maxab$ is an infinite cardinal since at least one of $\a$ or $\b$ is infinite.
        First we show that $\maxab \cdot \abs{\a} = \maxab \cdot \abs{\b} = \maxab$.
        We need only consider the case when $\abs{\a} \leq \abs{\b}$ since otherwise we apply can the same argument.
        Hence $\abs{\b} = \al_\g$ for some ordinal $\g$ since $\b$ must be infinite, and $\maxab = \abs{\b} = \al_\g$.
        We then clearly have $\maxab \cdot \abs{\b} = \al_\g \cdot \al_\g = \al_\g$ by Theorem~7.2.1.
        If $\abs{\a} = n$ for some finite $n$ then clearly $\maxab \cdot \abs{\a} = \al_\g \cdot n = n \cdot \al_\g = \al_\g$ by Corollary~7.2.2.
        Finally, if $\abs{\a}$ is infinite, then $\abs{\a} = \al_\d$ for some ordinal $\d$, and clearly $\d \leq \g$ since $\al_\d = \abs{\a} \leq \abs{\b} = \al_\g$.
        We then have $\maxab \cdot \abs{\a} = \al_\g \cdot \al_\d = \al_\d \cdot \al_\g = \al_\g$ again by Corollary~7.2.2.
        Hence we have shown that $\maxab \cdot \abs{\a} = \maxab \cdot \abs{\b} = \maxab$ since $\maxab = \al_\g$.

        Now we show the main result by transfinite induction.
        First, for $\b = 0$ then it has be that $\a$ is infinite so that $\maxab = \abs{\a}$ is also infinite.
        Then, clearly $\abs{\a^\b} = \abs{\a^0} = \abs{1} = 1 \leq \maxab$.
        Now suppose that $\abs{\a^\b} \leq \maxab$.
        Then we have
        \ali{
            \abs{\a^{\b+1}} &= \abs{\a^\b \cdot \a} & \text{(by the definition of ordinal exponentiation)} \\
            &= \abs{\a^\b} \cdot \abs{\a} & \text{(by Lemma~\ex.\ref{lem:orcarar:mult})} \\
            &\leq \maxab \cdot \abs{\a} & \text{(by the induction hypothesis and property (i) after Lemma~5.1.4)} \\
            &= \maxab & \text{(by what was just shown above)} \,.
        }
        Now suppose that $\b$ is a nonzero limit ordinal and that $\abs{\a^\g} \leq \maxag$ for all $\g < \b$.
        Then, for any $\g < \b$, clearly $\g \ss \b$ (since $\beta$ is transitive and $\g \in \b$ since $\g < \b$) so that $\abs{\g} \leq \abs{\b}$.
        
    }
    
    \emph{Main Problem.}

    That $\abs{\a + \b} \leq \al_\g$ follows almost immediately from Lemma~\ex.\ref{lem:orcarar:add}.
    We have that $\abs{\a + \b} = \abs{\a} + \abs{\b} \leq \al_\g + \al_\g = \al_\g$, where we have also used property (c) of cardinal numbers after Lemma~5.1.2, and Corollary~7.2.3.

    Similarly, $\abs{\a \cdot \b} \leq \al_\g$ follows from Lemma~\ex.\ref{lem:orcarar:mult}.
    We have that $\abs{\a \cdot \b} = \abs{\a} \cdot \abs{\b} \leq \al_\g \cdot \al_\g = \al_\g$, where we have used property (i) of cardinal numbers following Lemma~5.1.4, and Theorem~7.2.1.

    The analagous lemma for ordinal exponentiation (i.e. that $\abs{\a^\b} = \abs{\a}^{\abs{\b}}$ for ordinals $\a$ and $\b$) is evidently not true.
    As a counterexample consider $\a = 2$ and $\b = \w$.
    We then have that $\abs{\a^\b} = \abs{2^\w} = \abs{\w} = \al_0$ is countable whereas we know that $\abs{\a}^{\abs{\b}} = \abs{2}^{\abs{\w}} = 2^{\al_0}$ is uncountable.

    However, the somewhat analagous Lemma~\ex.\ref{lem:orcarar:exp} will help us show the desired result.
    First, if both $\a$ and $\b$ are finite then clearly $\a^\b$ is also finite so that clearly $\abs{\a^\b} \leq \al_\g$.
    On the other hand, if at least one of $\a$ or $\b$ is infinite, then have we have that $\abs{\a^\b} \leq \maxab \leq \al_\g$ by Lemma~\ex.\ref{lem:orcarar:exp} as desired, noting that clearly $\maxab \leq \al_\g$ since both $\abs{\a} \leq \al_\g$ and $\abs{\b} \leq \al_\g$. \qedsymbol
\end{solution}

\question{7.2.5}
\begin{solution}
	Clearly $f$ is a function from $\w_\a$ onto its image $X$ so that it follows from Lemma~7.1.6.\ref{lem:hs:onto} that $\abs{X} \leq \abs{\w_\a} = \al_\a$ as desired. \qedsymbol
    
    Note that the proof of Lemma~7.1.6.\ref{lem:hs:onto} uses exactly the technique given in the hint to argue its conclusion.
\end{solution}
